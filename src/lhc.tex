\chapter{The LHC}\label{chapter:lhc}

Maybe try narrative direction of particles travelling through LHC?
To be even more specific, perhaps use the journey of a di-higgs production process
(which you can detail the beginning of in the LHC section)
to explain the purpose of each part of the detector in identifying the consituent decay products of the HH pair
(maybe use some of the other final state channels as an excuse to explain components that may not matter to 4b):




Here I should justify why this chapter needs to exist. As said in intro, this is basically the "experimental setup" bit. This is where I explain the tools and setup I am using to do my experiment, so people have context and in theory could reproduce these results in the future.




% probably should explain the Run 1/2/3/etc stuff, since things change for that. I can incorporate this into the LHC's overall history: when did construction start, finish, when was it first turned on
% also luminosity (I think I might include a discussion of wtf luminosity/cross-section means in the appendices)
% as well as integrated luminosity per run

% ok, so why tf am i writing this bit? This is for someone who either doesn't know wtf the lhc/atlas are, or has just joined the group and needs to taught the basics of how this experiment even works. 
% what does it do
% technical specs (size, energy* (discuss this with the "Run" system, since it changes), number of detectors, what does it collide)
% more specs (collision rate, what is a 'bunch crossing', particles in a bunch, strength of magnets?)

% where is it, why was it even built, who built it.
The Large Hadron Collider (LHC) is a massive particle accelerator located near Geneva, Switzerland.
It was constructed by CERN (European Organization for Nuclear Research, from the French \textit{Conseil Européen pour la Recherche Nucléaire}) for the purpose of studying high-energy physics well above energies observed in any previous experiment.

Construction took place between 1995 and 2007, 





\section{Interaction Region} %TODO
    beam crossing point,
    how do bunches cross each other and interact
    interaction rate: ~1000 particles every 25 ns w/in |eta| < 2.5.
    the beampipe focusing magnets,



