\chapter{The LHC}\label{chapter:lhc}
%START:
%    Theoretically, the dihiggs production is a thing that should exist, and which we would like to observe. 
%    The higgs does not exist in an observable state naturally, so it must be artificially created.
%    For this we need the LHC.
%
%IN BETWEEN:?
%    What's the LHC? A very large proton-proton particle collider.
%
%    What's a particle collider? A machine which accelerates charged particles to very high velocities/energies using electro-magnetic fields.
%
%    Why protons? Protons are less susceceptible to synchrotron radiation. Also protons consist of strongly-interacting particles.
%
%        Why are protons less susceceptible to synchrotron radiation? TODO
%
%        Why does strong interaction matter? (does it?) TODO
%
%        Also note that protons have the downside of PDFs
%
%    Why are we smashing particles together? Because (as should have been shown in the last section?) the cross section of a particle interaction increases proportional to the particles' energies
%
%    How do we make particles go fast? Injection system + magnets on main ring, also discussion of centripital force/mag-field/radius energy limitations  TODO
%
%    How do you smash particles together? Interaction region TODO
%
%    How often can it make a higgs? or a dihiggs? This can be found by multiplying the cross-section of the desired interaction by luminosity
%
%        What is luminosity? A combination of many things:
%            tightness of beam at IR,
%            center of mass energy machine is running at,
%            how many interactions per second (interaction rate and bunch crossing),
%            how long the machine is running;
%            All of the above have changed over time, per Run 1,2,3
%
%END:
%    We have succesfully produced a di-higgs event.
%
%  ||||||||||
%  VVVVVVVVVV
%
%LAYOUT:
%    INTRO:
%        Theoretically, the dihiggs production is a thing that should exist, and which we would like to observe. 
%        The higgs does not exist in an observable state naturally, so it must be artificially created.
%        For this we need the LHC.
%
%        What's the LHC? A very large proton-proton particle collider.
%        What's a particle collider? A machine which accelerates charged particles to very high velocities/energies using electro-magnetic fields.
%        Why are we smashing particles together? Because (as should have been shown in the last section?) the cross section of a particle interaction increases proportional to the particles' energies
%
%    ACCELERATOR RING
%        How do we make particles go fast? Injection system + magnets on main ring, also discussion of centripital force/mag-field/radius energy limitations  TODO
%        Why protons? Protons are less susceceptible to synchrotron radiation. Also protons consist of strongly-interacting particles.
%            Why are protons less susceceptible to synchrotron radiation? TODO
%            Why does strong interaction matter? (does it?) TODO
%            Also note that protons have the downside of PDFs
%
%
%    INTERACTION REGION
%        How do you smash particles together? Interaction region TODO
%        How often can it make a higgs? or a dihiggs? This can be found by multiplying the cross-section of the desired interaction by luminosity
%            What is luminosity? A combination of many things:
%                tightness of beam at IR,
%                center of mass energy machine is running at,
%                how many interactions per second (interaction rate and bunch crossing),
%                how long the machine is running;
%                All of the above have changed over time, per Run 1,2,3
%
%END:
%    We have succesfully produced a di-higgs event.

\section{Introduction} TODO
    % Theoretically, the dihiggs production is a thing that should exist, and which we would like to observe. 
    % The higgs does not exist in an observable state naturally, so it must be artificially created.
    % For this we need the LHC.
    The high mass and short life-time of the Higgs Boson ensures that it cannot be readily found in nature.
    In order to study the Higgs Boson then, it must first be artificially created through extremely high-energy physical interactions.
    The Large Hadron Collider (LHC), among the largest and most complex machines ever constructed, was designed for exactly this purpose.
    Built by the European Organization for Nuclear Research (CERN, from the French \textit{Conseil Européen pour la Recherche Nucléaire}) with the goal of studying high-energy physics,
    the LHC is able to probe interaction energies well beyond that of any previous particle physics experiments.

    % What's the LHC? A very large proton-proton particle collider.
    % What's a particle collider? A machine which accelerates charged particles to very high velocities/energies using electro-magnetic fields.
    % Why are we smashing particles together? Because (as should have been shown in the last section?) the cross section of a particle interaction increases proportional to the particles' energies
    % Brief discussion of fixed vs dual-beam collider and CoM energy calulation
    % General history, size, specs TODO 
    The LHC is a proton-proton particle accelerator; indeed, it is the largest particle acclerator ever built.
    A particle accelerator is a machine which uses electromagnetic fields to accelerate charged particles to extremely high energies, with the goal of colliding these particle together.
    The reason for doing this is that the cross section of a particle interaction scales with the center-of-mass energy of the incoming particles. %TODO try to back this up with specific details eg a formula from QFT showing how the probability of an interaction depends on energy
    Older accelerators operated in a fixed-target arrangment, in which a single beam of particles was accelerated into a stationary wall of material.
    The energy of such collisions was [FIXME: insert formula for Ecom]
    Newer accelerators, including the LHC, use two independant particle beams which are collided with each.
    Due to the properties of Lorentz Boost transformations, these dual-beam colliders achieve much higher com energies of [FIXME: insert other formula here] \cite{modern_and_future_colliders}
    Construction of the LHC took place between 1995 and 2007, over 40 meters underground beneath the French/Swiss border, near Geneva Switzerland. 




\section{Accelerator Ring} TODO
    % How do we make particles go fast? Injection system + magnets on main ring, also discussion of centripital force/mag-field/radius energy limitations  TODO
    % Why protons? Protons are less susceceptible to synchrotron radiation. Also protons consist of strongly-interacting particles. TODO
    % Why are protons less susceceptible to synchrotron radiation? TODO
    % Why does strong interaction matter? (does it?) TODO
    % Also note that protons have the downside of PDFs TODO
    Gotta go fast! (Beam injection and main ring specs).
    I'm noticing there isn't much discussion on the actual beam injection, so for now I actually think I might just combine it with the main ring section.



\section{Interaction Region} TODO
    % How do you smash particles together? Interaction region TODO
    % How often can it make a higgs? or a dihiggs? This can be found by multiplying the cross-section of the desired interaction by luminosity TODO
    % What is luminosity? A combination of many things: TODO
    % tightness of beam at IR, TODO
    % center of mass energy machine is running at, TODO
    % how many interactions per second (interaction rate and bunch crossing), TODO
    % how long the machine is running; TODO
    % All of the above have changed over time, per Run 1,2,3 TODO
    Particles go smash
    - the beampipe focusing magnets,
    - beam crossing point: how do bunches cross each other and interact
    - luminosity: what is it, what determines it 
    - bunch crossing
    - interaction rate: ~1000 particles every 25 ns w/in |eta| < 2.5.
