\chapter{The LHC}\label{chapter:lhc}

% Purpose, obviously -- DONE
The high mass and short life-time of the Higgs Boson ensures that it cannot be readily found in nature.
In order to study the Higgs Boson then, it must first be artificially created through extremely high-energy physical interactions.
The Large Hadron Collider (LHC), among the largest and most complex machines ever constructed, was designed for just such a purpose.
Built by the European Organization for Nuclear Research (CERN, from the French \textit{Conseil Européen pour la Recherche Nucléaire}) with the goal of studying high-energy physics,
the LHC is able to achieve energies well in excess of any previous particle physics experiments.

% Why is 8 and then 13 TeV so critical for Higgs measurement? -- A: because cross section scales with energy, and we just wanted the highest energy possible TODO
% ok wait, I should probably explain what the LHC is even colliding, and why. i.e. how do you generate a "high energy interaction" TODO
% in fact, I kinda need to explain what a "collider" even is. Why are we smashing things together? Maybe it's worth discussing rutherford scattering and the whole "high energy = small wavelength = tinier things you can probe"?

% General history, size, specs TODO
Construction of the LHC took place between 1995 and 2007, over 40 meters underground beneath the French/Swiss border, near Geneva Switzerland. 


how does a particle accelerator make things go fast?
talk about magnets, centripital force, limits to energy based on mag field and radius.
PDF issues of using protons.
Why do we use protons at the LHC?
EM radiation from changing direction, which affects protons less.
Also they're bigger, which is handy (is it?)

run 1/2/3 history discussion.

What does the LHC do, and why?
It's really worth asking why the mechanism of studying fundamental physics is as crude as smashing particles together.
ok but is it though? cross-section of an interaction scales with CoM energy. That's it. Do I really want to be *that* asshole going on about historical bs?


I'm probably going to scrap this whole stupid paragraph and everything it discusses.
Long ago, small things were discovered by looking at them with a microscope.
This meant bouncing light (photons) off things and looking at that reflected light.
At the dawn of atomic physics (1908), Rutherford worked out the structure of the atom by scattering alpha particles off of gold foil.
Later, in 1968, the proton itself was determined to consist of smaller ``partons" based on higher energy scattering experiments performed at SLAC.
The precident set in these, and many other, experiments was that hitting things at higher energy gives a more detailed view of objects.

    


\section{Accelerator Ring}
    Gotta go fast! (Beam injection and main ring specs).
    I'm noticing there isn't much discussion on the actual beam injection, so for now I actually think I might just combine it with the main ring section.

\section{Interaction Region}
    Particles go smash
    - the beampipe focusing magnets,
    - beam crossing point: how do bunches cross each other and interact
    - luminosity: what is it, what determines it 
    - bunch crossing
    - interaction rate: ~1000 particles every 25 ns w/in |eta| < 2.5.
