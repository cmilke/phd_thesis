%\input{src/introduction/thesis_tutorial.tex}

\chapter{Introduction}\label{chapter:introduction}

%The basic structure of every lab report you've ever done is: 
%    introduction,
%    theory,
%    experimental setup,
%    procedure,
%    results,
%    conclusion.
%
%This thesis is essentially following that same format:
%    Introduction,
%    Standard Model... and Signal modelling? (theory),
%    LHC and ATLAS and Trigger (experimental setup)
%    Reconstruction, Selection, Background Estimation (procedure)
%    results (...)
%    conclusion

The Higgs Boson was proposed as a fundamental particle in 1964
    and has been a driving factor in particle physics research ever since.
Though it was jointly discovered in 2012 by the ATLAS and CMS collaborations,
    there are still a number of properties of the Higgs which have yet to be measured.
Chief among these are two of the Higgs' self-coupling constants
    which determine how strongly the Higgs interacts with both itself and with vector bosons.
The unknown quality of these constants is represented in this analysis by a set of scaling factors, \kl and \kvv,
    which scale the coupling constants from their expected Standard Model value.
This analysis performs a search for the rare,
    as-yet unseen di-Higgs interaction in order to set tighter constraints on these scaling values.
The search utilizes the Vector Boson Fusion (VBF) production mechanism of the di-Higgs process,
    wherein two incoming quarks emit a pair of vector bosons that fuse into some intermediate state
    as the initial quarks continue along deflected trajectories.
The di-Higgs system is identified in its four bottom quark ($b \bar b b \bar b$) decay mode,
    producing a total 6-jet final state.
This is performed using 126 $\textit{fb}^{-1}$ of data from Run 2 of ATLAS.

This thesis will begin by first explaining the importance of the Higgs Boson to the field of particle physics,
    as well as the relevance of the aforementioned \kl and \kvv scaling factors.
I will then describe the physical effects of these constants,
    and how those effects can be exploited in order to make measurements of them.
These topics comprise Chapter \ref{chapter:theory}.
Following from the theoretical discussion,
    I will discus the experimental equipment and setup used to perform these measurements across three chapters.
Chapter \ref{chapter:lhc} discusses the Large Hadron Collider, the machine that generates the conditions necessary to produce the di-Higgs process.
Actual observation of the process once generated is carried out by the ATLAS particle detector array and its Trigger system,
    described in Chapters \ref{chapter:atlas} and \ref{chapter:data}.
The reconstruction process, described in Chapter \ref{chapter:reconstruction},
    entails how data observed by ATLAS are analyzed in order to determine which physics processes occurred during the myriad observed interactions.
Background processes are removed from the data set using a selection procedure explained in Chapter \ref{chapter:selection},
    and the contribution of the remaining background events is estimated using a data-driven technique explained in Chapter \ref{chapter:background}. 
The background estimation process makes use of a neural network reweighting procedure, which I personally configured and trained.
My primary contribution to the analysis however, comes in the form of the signal model that constitutes the hypothesis for the behavior of the Higgs Boson.
The signal model makes use of a Monte-Carlo sample combination technique that I developed and optimized in the context of the \vbfproc process.
This is elaborated upon in Chapter \ref{chapter:signal}.
Finally, in Chapter \ref{chapter:results}, this hypothesis and the background estimate are compared against the data observed from ATLAS,
    in order to make constraints on the di-Higgs self-coupling constants.


