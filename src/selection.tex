\chapter{Event Selection} \label{chapter:selection}
    
    I think I'm just going to do everything else here? It's weird, because I'm going backwards to triggers,
        and then forward to the final analysis stuff. But at least I have all the complicated object algorithms and stuff listed I guess?

    %I need to discuss the kinematics of VBF and 4b in order to justify the ->

    %Truthfully, reconstruction and selection are not distinct stages of the analysis, but rather are interwoven with each other.
    %The first step of selection really occured before any signficant reconstruction had even taken place, in the L1 Trigger system.
    %Yet the final stage of event reconstruction -- the reconstruction of the di-Higgs system --
    %    will be performed \textit{after} almost every other stage of selection.


    \section{Triggers}

        \begin{table}[htbp]
\centering \footnotesize
\begin{tabular}{ccc}
Year                      & Trigger Name                                                                    & \textbf{Trigger Type}  \\ 
\hline
\multirow{2}{*}{2016}                      & HLT\_j100\_2j55\_bmv2c2060\_split                                               & 2b1j                   \\
                      & HLT\_2j35\_bmv2c2060\_split\_2j35\_L14J15.0ETA25                                & 2b2j                   \\

\hline

\multirow{2}{*}{2017}                      & HLT\_j110\_gsc150\_boffperf\_split\_2j35\_gsc55\_bmv2c1070\_split\_L1J85\_3J30  & 2b1j                   \\
                      & HLT\_2j15\_gsc35\_bmv2c1040\_split\_2j15\_gsc35\_boffperf\_split\_L14J15.0ETA25 & 2b2j                   \\

\hline

\multirow{2}{*}{2018}                      & HLT\_j110\_gsc150\_boffperf\_split\_2j45\_gsc55\_bmv2c1070\_split\_L1J85\_3J30  & 2b1j                   \\
                      & HLT\_2j35\_bmv2c1060\_split\_2j35\_L14J15.0ETA25                                & 2b2j                   \\
                
\end{tabular}
\caption{Triggers used for non-resonant searches.\cite{hh4b_2021_int_note}}
\label{tab:nr-triggers-used}
\end{table}


        Triggers used in this analysis and why (do we have any plots showing why we use these triggers?)

        Maybe I should just take that old trigger explaining slide of Mathew's
            and literally just lay out how to interpret what each of these triggers does.

        Trigger bucketing strategy?


    \section{Analysis Cuts} \label{sec:analysis_cuts}

        After the triggers, a number of other things are cut on.

    \subsection{Jet Multiplicity and Categorization}

        Central Jets are defined as those within $|\eta| \leq 2.5$, %TODO
            with pt > 40 GeV and which pass JVT or have pt > 60 or have |eta| > 2.4. %TODO
        
        Forward Jets are those with $ 2.5 < |\eta| \leq 4.5 $,
            with pt >= 30 GeV %TODO

        Must have at least six total central and forward jets.
        At least four of the jets must be b-tagged central jets.
        At least two of the jets (forward or central) must be anti-btagged.

        The Higgs decay jets are chosen as the four highest-pt, b-tagged, Central jets.
        The VBF jets are chosen as the pair of anti-btagged jets (central or forward)
            with the largest vector-sum invariant mass (mjj) between them. %TODO: at most just provide a source that does this, since it's common practice


    \subsection{VBF Topology}
        
        The selected VBF pair must have a \deta between them of at least 3, %TODO
            and an mjj of at least 1 TeV. %TODO
        As well, the combined vector-sum-pt of all six jets
            (the four Higgs products and the two VBF jets)
            must be less than 65 GeV. %TODO


    \subsection{HH \deta Cut}

        MinDR pairing to reconstruct Higgss; %TODO
        Reconstructed Higgs Bosons must have a \deta between them of at most 1.5 %TODO 


    \subsection{Region Definition}
        
        There are three ``regions'' which are used for the analysis.
        The ``Control'' and ``Validation'' regions are used for the Background Estimation (see Chapter \ref{chapter:background}).
        The ``Signal'' Region is the set of data which will be used to search for the di-Higgs signal.

        The region which an event falls into is based on the reconstructed masses of the leading and sub-leading Higgs.
        The Signal Region corresponds to those events for which the reconstructed Higgs masses
            align closely with the measured value of the Higgs Boson (125 GeV), with additional room provided for error in measurment.
        The Control and Validation Regions are adjacent to the Signal Region,
            designating events with kinematics very similar to the Signal Region,
            but with reconstructed Higgs masses incompatible with experimental measurment (see Figure TODO).



        %Do I need to show proof of all our cut steps? Like, surely there's a motivation for each of them?

  %      Two VBF Initial Scatter (IS) quark jets:
  %          light jets (u,d, or maybe charm);
  %          high pt;
  %          wide opening angle;
  %          high mjj;
  %          can be central or forward.

  %      Four b-jets:
  %          decay products of higgss;
  %          all central; %TODO: why are interesting things always central? can i show this mathematically?
  %          also high pt;
  %          b/b-bar products are expected to have very low opening angle between them.

  %      Overall:
  %          Zero pt initially, so expected low pt for vector sum of all jets combined
  %          

  %      Make note of the fact that the anti-$k_t$ algorithm is set with $\Delta R = 0.4$.

  %  kinematic cuts
  %      basic jet multiplicity reqs,
  %      btag jets with DL1r 77\% working point,
  %      4b-tagged, central
  %      also 2b and 3b1f, which are reserved for background estimation
  %      2 non-btagged jets with min-mjj for VBF

  %  Then there's the minDR pairing of b's to reconstruct HH

  %  Then cut on dihiggs and full system kinematics

  %  Then make sure the higgss fall within the "signal region" (their individual masses are approximately 125 GeV)


%...and how we wittle the abundance of events down to a manageable subset.
%If I stick to this format, there's still a bit of event reconstruction done here.
%
%HH4b Resolved Reco and Selection (literally just go through resolved recon...).
%    Final stage is the signal region selection
