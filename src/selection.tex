\chapter{Event Selection}
    
    I think I'm just going to do everything else here? It's weird, because I'm going backwards to triggers,
        and then forward to the final analysis stuff. But at least I have all the complicated object algorithms and stuff listed I guess?

    I need to discuss the kinematics of VBF and 4b in order to justify the ->

    %Truthfully, reconstruction and selection are not distinct stages of the analysis, but rather are interwoven with each other.
    %The first step of selection really occured before any signficant reconstruction had even taken place, in the L1 Trigger system.
    %Yet the final stage of event reconstruction -- the reconstruction of the di-Higgs system --
    %    will be performed \textit{after} almost every other stage of selection.

    \section{Triggers}

        Triggers used in this analysis and why (do we have any plots showing why we use these triggers?)

        Trigger bucketing strategy?

    \section{Analysis Cuts} 

    kinematic cuts
        basic jet multiplicity reqs,
        btag jets with DL1r 77\% working point,
        4b-tagged, central
        2 non-btagged jets with min-mjj for VBF

    Then there's the minDR pairing of b's to reconstruct HH

    Then cut on dihiggs and full system kinematics

    Then make sure the higgss fall within the "signal region" (their individual masses are approximately 125 GeV)

    Pretty sure that's it?



%...and how we wittle the abundance of events down to a manageable subset.
%If I stick to this format, there's still a bit of event reconstruction done here.
%
%HH4b Resolved Reco and Selection (literally just go through resolved recon...).
%    Final stage is the signal region selection
