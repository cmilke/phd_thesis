% This might actually need to become its own chapter...
\section{The Standard Model of Particle Physics}
    \subsection{The Higgs Mechanism}\label{sec:higgs_mechanism}

        Basic higgs mechanism example.

        Have a field interacting with a potential V with quadratic and quartic terms.
        This produces a symmetric but unstable equilibrium.
        Perform passive translation to recenter coordinates around stable equilibrium point/minimum of potential.
        In super-simple example, this gives an originally massless particle mass.

        Ok so let's start with a really basic lagrangian describing only the kinetic energy of a massless scalar particle $\phi$, of the form:

        \begin{equation}
            \Lag = K_{\phi} = \frac{1}{2} (\partial_{\mu} \phi)^2
        \end{equation}

        Let's introduce a quartic potential $V(\phi) = -\frac{1}{2} \mu^2 \phi^2 + \frac{\lambda}{4!} \phi^4$

        \begin{equation}
            \Lag = K_{\phi} - V{\phi} = \frac{1}{2} (\partial_{\mu} \phi)^2 
                +\frac{1}{2} \mu^2 \phi^2 - \frac{\lambda}{4!} \phi^4
        \end{equation}

        Such a potential will result in a Hamiltonian which is symmetric about $\phi=0$, but that point will be a local maximum.% TODO figure for this
        The Hamiltonian will have two minima to either side of $\phi=0$, at points $\pm \nu = \pm \sqrt{\frac{6}{\lambda}} \mu$.

        A system in such a potential would invariably fall into one of these minima.
        The Lagrangian can be rewritten from the perspective of one of these minima (e.g.\ $+\nu$),
            by substituting in a shifted field $h$, where $\phi(x)=\nu+h(x)$.
        The Langrangian now takes the form (after simplifying)

        \begin{equation} \begin{split} \label{eq:basic_higgs}
            \Lag & = \frac{1}{2} (\partial_{\mu} h)^2
                - \mu^2 h^2
                -\sqrt{\frac{\lambda}{6}} \mu h^3
                - \frac{\lambda}{4!} h^4 \\
             & = \frac{1}{2} (\partial_{\mu} h)^2
                - m^2_{h} h^2
                - k_{h} h^3
                - \frac{k_{2h}}{4!} h^4
        \end{split} \end{equation} %FIXME: you screwed up the terms somewhere

        With the latter equation taking the form of a now massive field $h$ with both a three and four point vertex,
            governed by two different coupling values $k_{h}$, and $k_{2h}$.

        \cite{Halzen_book}

        

    \subsection{Electro-Weak Symmetry Breaking}

        Let's now give the scalar field a complex phase and spinor components:
        \begin{equation}
            \phi(x) = \frac{1}{\sqrt{2}} e^{i \beta} \tinymatrix{\phi_1 \\ \phi_2}
        \end{equation}
        $\phi$ is still a scalar in spacetime, but now also has vector components in the $SU(2)$ subspace.

        As with the dirac fields, we will then impose $U(1) \times SU(2)$ gauge symmetry on the field, so it transforms as:
        \begin{equation}
            \phi(x) \rightarrow e^{i \alpha^a \tau^a} e^{i \beta/2 } \phi(x)
        \end{equation}.

        The added symmetries require that the derivative be changed to a covarient derivative 
            $D^{\mu} = \partial^{\mu} - \frac{ig}{2} \wField^a_{\mu} \sigma^a - \frac{ig'}{2} B_{\mu}$,
            producing a Lagrangian:
        \begin{equation} \begin{split}
            \Lag & = \frac{1}{2} (D_{\mu} h)^2
                - \mu^2 h^2
                -\sqrt{\frac{\lambda}{6}} \mu h^3
                - \frac{\lambda}{4!} h^4 \\
        \end{split} \end{equation}

        Once again, we allow the scalar field to fall into its offset vev $v$, as $\phi(x) \rightarrow h(x) + v$.
        Now however, $v$ is a spinor value $v = \tinymatrix{v_1 \\ v_2}$.
        With gauge freedom, this can be rotated entirely along one axis as $\vec{v} = \frac{1}{\sqrt{2}}\tinymatrix{0 \\ v}$,
            with $v = \sqrt{\mu^2/\lambda}$.
        Substituting into the Lagrangian now produces a more complex expression:
        \begin{equation} \begin{split}
            \label{eq:fullHiggs}
            \Lag & = \frac{1}{2} (D_{\mu}^{ij} (h+v)_j)^2
                + \mu^2 (h+v)^2
                - \frac{\lambda}{4} (h+v)^4 \\
             & = \Lag_h + \Lag_v
        \end{split} \end{equation}
        Where $\Lag_h$ takes a form similar to Equation \ref{eq:basic_higgs}, incorporating both the $h$ and $h$/$v$ cross terms.
        Meanwhile, $\Lag_v$ refers only to the terms arising from $D_{\mu}$ acting on the vev:
        \begin{equation}
            \label{eq:lagV}
            \Lag_v = \frac{1}{2} (D_{\mu}^{ij} v_j)^2
        \end{equation}

        The expansion of this term is crucial to the structure of the Standard Model,
            as it alone will lead to the breakdown of Electro-Weak Symmetry,
            and the $W$ and $Z$ bosons acquiring mass.

        %Get covariant derivative, evaluate at vev, pull out W,Z, and photon fields, and their masses (pg 722);
        Expanding only $D_{\mu}^{ij} v_j$ to start, the $\partial_{\mu}$ immediately vanishes ($v$ is a constant), yielding 
        \begin{equation} \begin{split}
            D^{\mu} v  = \big( \partial^{\mu} & - \frac{ig}{2} \wField^a_{\mu} \sigma^a - \frac{ig'}{2} B_{\mu} \big) \frac{1}{\sqrt{2}}\minimatrix{0\\v} \\
            = \big( & - \frac{ig}{2} \wField^a_{\mu} \sigma^a - \frac{ig'}{2} B_{\mu} \big) \frac{1}{\sqrt{2}}\minimatrix{0\\v} \\
            = - \frac{i}{2} \big( & g \wField^a_{\mu} \sigma^a + g' B_{\mu} \big) \frac{1}{\sqrt{2}}\minimatrix{0\\1} v
        \end{split} \end{equation}

        It is useful here to fully expand the $U(1) \times SU(2)$ fields into their matrix components and add them explicitly,
            as doing so reveals the origin of the photon and W and Z bosons.
        \begin{equation} \begin{split}
            g \wField^a_{\mu} \sigma^a + g' B_{\mu} & =
                g \wField^1_{\mu} \sigma^1
                + g \wField^2_{\mu} \sigma^2
                + g \wField^3_{\mu} \sigma^3
                + g' B_{\mu} I \\
            & = \begin{pmatrix}
                0 & g\wField^1_{\mu} \\ g\wField^1_{\mu} & 0 \end{pmatrix}
                + \begin{pmatrix} 0 & -ig\wField^2_{\mu} \\ ig\wField^2_{\mu} & 0 \end{pmatrix}
                + \begin{pmatrix} g\wField^3_{\mu} & 0 \\ 0 & -g\wField^3_{\mu} \end{pmatrix}
                + \begin{pmatrix} g'B_{\mu} & 0 \\ 0 & g'B_{\mu}
            \end{pmatrix} \\
            & = \begin{pmatrix} 
                g\wField^3_{\mu} + g'B_{\mu} & g\wField^1_{\mu} - ig\wField^2_{\mu} \\
                g\wField^1_{\mu} + ig\wField^2_{\mu} & -g\wField^3_{\mu} + g'B_{\mu}
            \end{pmatrix}
        \end{split} \end{equation}

        The four components of this matrix are the gauge boson fields of the electromagnetic ($A$) and weak ($W$ \& $Z$) interactions
        \begin{equation} \begin{split}
            \begin{pmatrix} 
                g\wField^3_{\mu} + g'B_{\mu} & g\wField^1_{\mu} - ig\wField^2_{\mu} \\
                g\wField^1_{\mu} + ig\wField^2_{\mu} & -g\wField^3_{\mu} + g'B_{\mu}
            \end{pmatrix} =
            \begin{pmatrix} 
                \sqrt{g^2 + g^{\prime 2}}\ A_{\mu} & g \sqrt{2}\ W^+_{\mu} \\
                g \sqrt{2}\ W^-_{\mu} & - \sqrt{g^2 + g^{\prime 2}}\ Z^0_{\mu}
            \end{pmatrix}
        \end{split} \end{equation}

        With $A$, $W^+$, $W^-$, and $Z^0$ related to the unbroken $\wField^a$ and $B$ fields by
        \begin{equation} \begin{split}
            A_{\mu} & = \frac{1}{\sqrt{g^2 + g^{\prime 2}}} ( g\wField^3_{\mu} + g'B_{\mu} ) \\
            Z^0_{\mu} & = \frac{1}{\sqrt{g^2 + g^{\prime 2}}} ( g\wField^3_{\mu} - g'B_{\mu} ) \\
            W^{\pm}_{\mu} & = \frac{1}{\sqrt{2}} (\wField^1_{\mu} \mp i\wField^2_{\mu})
        \end{split} \end{equation}

        The $\sqrt{g^2 + g^{\prime 2}}$ factor is the result of converting between
            $\tinymatrix{Z^0 \\ A}$ and $\tinymatrix{\wField^3 \\ B}$ by way of a rotation matrix
        \begin{equation} \begin{split}
            \begin{pmatrix} Z^0 \\ A \end{pmatrix} =
            \begin{pmatrix}
                \frac{g}{\sqrt{g^2 + g^{\prime 2}}} & \frac{-g'}{\sqrt{g^2 + g^{\prime 2}}} \\
                \frac{g'}{\sqrt{g^2 + g^{\prime 2}}} & \frac{g}{\sqrt{g^2 + g^{\prime 2}}}
            \end{pmatrix} \begin{pmatrix} \wField^3 \\ B \end{pmatrix} = 
            \begin{pmatrix}
                \cos\theta_w & -\sin\theta_w \\
                \sin\theta_w & \cos\theta_w
            \end{pmatrix} \begin{pmatrix} \wField^3 \\ B \end{pmatrix}
        \end{split} \end{equation}

        Where $\theta_w \equiv \cot(\frac{g'}{g})$ is known as the \textit{weak mixing angle}.

        Returning now to Equation \ref{eq:lagV}, we now have
        \begin{equation} \begin{split}
            \Lag_v & = \frac{1}{2} (D_{\mu}^{ij} v_j)^2 \\
            & = \frac{1}{2}
                \frac{1}{\sqrt{2}} \begin{pmatrix} 0 & v \end{pmatrix}
                \left| -\frac{i}{2}
                    \begin{pmatrix} 
                        \sqrt{g^2 + g^{\prime 2}}\ A_{\mu} & g \sqrt{2}\ W^+_{\mu} \\
                        g \sqrt{2}\ W^-_{\mu} & - \sqrt{g^2 + g^{\prime 2}}\ Z^0_{\mu}
                    \end{pmatrix}
                \right|^2
                \frac{1}{\sqrt{2}} \begin{pmatrix} 0 \\ v \end{pmatrix} \\
            & = \frac{1}{2} \frac{v^2}{2} \frac{1}{4}
                \begin{pmatrix} 0 & 1 \end{pmatrix}
                \begin{pmatrix} 
                    \sqrt{g^2 + g^{\prime 2}}\ A_{\mu} & g \sqrt{2}\ W^+_{\mu} \\
                    g \sqrt{2}\ W^-_{\mu} & - \sqrt{g^2 + g^{\prime 2}}\ Z^0_{\mu}
                \end{pmatrix}
                \begin{pmatrix} 
                    \sqrt{g^2 + g^{\prime 2}}\ A_{\mu} & g \sqrt{2}\ W^+_{\mu} \\
                    g \sqrt{2}\ W^-_{\mu} & - \sqrt{g^2 + g^{\prime 2}}\ Z^0_{\mu}
                \end{pmatrix}
                \begin{pmatrix} 0 \\ 1 \end{pmatrix} \\
            & = \frac{1}{2} \frac{v^2}{2} \frac{1}{4}
                \begin{pmatrix} 
                    g \sqrt{2}\ W^-_{\mu} & - \sqrt{g^2 + g^{\prime 2}}\ Z^0_{\mu}
                \end{pmatrix}
                \begin{pmatrix} 
                     g \sqrt{2}\ W^+_{\mu} \\
                     - \sqrt{g^2 + g^{\prime 2}}\ Z^0_{\mu}
                \end{pmatrix} \\
            & = \frac{1}{2} \frac{v^2}{2} \frac{1}{4} 
                \left[ 2 g^2  W^-_{\mu} W^+_{\mu}
                + \left(\sqrt{g^2 + g^{\prime 2}}\right)^2 (Z^0_{\mu})^2 \right] \\
            & = \frac{1}{2} \left[ \left(\frac{vg}{2}\right)^2\  W^-_{\mu} W^+_{\mu}
                + \frac{1}{2} \left(\frac{v}{2}\sqrt{g^2 + g^{\prime 2}}\right)\ (Z^0_{\mu})^2 \right]
        \end{split} \end{equation}

        As in Section \ref{sec:higgs_mechanism}, the $W$ and $Z$ fields now have additional mass terms associated with their kinetic energy terms,
            with $M_W = \frac{vg}{2}$ and $M_Z = \left(\frac{v}{2}\sqrt{g^2 + g^{\prime 2}}\right) $.
        The photon field $A_{\mu}$ is notably absent in the final vev product, and thus is left massless.
        A similar procedure can then be followed to allow the Higgs Field to interact with fermions
            (albeit with complications arising from mass-mixing and chirality), which will grant mass to the fermion particles.


        % Fermion time!!
    %\subsection{Giving Mass to Fermions}
    % Ok this is seriously just the same thing but now we have to split the fermion fields into left and right and up-type and down-type
    % and split the higgs into the higgs and conjugate higgs and also insert the complex mass matrix except don't insert it for charged leptons
    % because reasons or something. And then poof your charged leptons have mass and your quarks have mass and flavour changing
        

        %Split fermions between right and left fields, assign LH to SU(2) doublets (T=+-1?), RH to SU(2) singlet (T=0), assign Y too (pg724-725);

        %Sandwitch covariant derivative terms between fermion fields, expand to get field currents (pg 725-726);

        %*Try* to expand masses and fail because of representation incompatibilities (pg 725);

        %Anamoly cancellation thing I'll probably ignore (pg 726);

        %Add higgs interaction psibar phi psi to fermion lagrangian, expand into higgs interaction, convert to fermion mass (pg 734);



        
    \subsection{The Higgs Boson and di-Higgs Interactions}

        With the critical role of the Higgs Field established, it is now time to return to Equation \ref{eq:fullHiggs},
            and investigate $\Lag_h$, the Lagrangian of the Higgs Boson itself.
        The terms involved therein provide information not only about the Higgs Field,
            but also provide insight into how the Higgs may be further studied.
        \begin{equation} \begin{split}
            \Lag & = \frac{1}{2} (D_{\mu}^{ij} (h+v)_j)^2
                + \mu^2 (h+v)^2
                - \frac{\lambda}{4} (h+v)^4 \\
            = \big( \partial^{\mu} & - \frac{ig}{2} \wField^a_{\mu} \sigma^a - \frac{ig'}{2} B_{\mu} \big) \frac{1}{\sqrt{2}}h \\
                - \mu^2 h^2
                -\sqrt{\frac{\lambda}{6}} \mu h^3
                - \frac{\lambda}{4!} h^4 \\
        \end{split} \end{equation}


        
        
        %Further work must then be done to couple the higgs boson to fermions and itself
        %You need to tie k2v, kl, and kv in to the shape of the higgs potential here
