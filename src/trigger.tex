\chapter{Trigger System} %TODO
\section{Introduction}
%What
ATLAS trigger system

%Why
To reduce overall ATLAS data output during runtime by removing low-interest events.
Refered to as ``online" analysis, as selection of data is occuring live, while the machine is running.
The data which survives this selection is read out to disk and distributed to individual research teams for more sophisticated ``offline" analysis.

%When
The trigger system process all ATLAS events immediately after occurance, reducing the 40 MHz bunch crossing rate to an output rate of 1 kHz.

%Where
To reduce latency from data transfer times, all the trigger systems are located very close to the ATLAS detector system.

%How
Triggering is achieved by running bunch crossing ``events" through two sequential trigger systems.
First is the hardware-based Level 1 Trigger (L1).
Then is the software-based High Level Trigger (HLT).
Both of these triggers involve a plethora of different measurements on various aspects of the events, such as total transverse energy, transverse momentum, jet multiplicity, and opening angles between jets.
A ``trigger chain" is a combination of several different such conditions set to various thresholds.
A bunch crossing satisfies a trigger chain if it is able to pass all conditions in the chain simultaneously.
There are a huge number of different trigger chains, all comprising the ``trigger menu".
An event is ultimately accepted and read out to disk for further analysis offline if it passes any one of the trigger chains, and each accepted event is tagged with which trigger chains it passed.
The chains included in the trigger menu were decided upon before the beginning of the Run 2 data taking period, based on input from various analysis teams.



\section{Level 1 Trigger}

%What
Hardware level trigger

%Why
Reduce event rate acceptance to acceptable levels for more sophisticated (but slower) software level trigger

%When
Input rate of 40 MHz (every 25 ns, the LHC bunch crossing rate), output rate of 100 kHz (utilizes detector buffer memory to keep up). \cite{trigger_run2}

%Where
As close to ATLAS as possible in order to reduce latency, specifically in the USA15 underground chamber \cite{trigger_tdr}.


%How
Uses only information from calorimeters and Muon Trigger Chamber


Calorimeter Algorithms: \cite{L1_calo_run1}
    Based on Calorimeter information.

    Reduces granularity of cals by clustering sensors together into ``trigger towers" each with a resolution of 0.1x0.1 in $\Delta \eta x \Delta \phi$.

    A ``Region of Interest" (RoI) is created as a group of towers which collectively satisfy various conditions (varying between the trigger in question).

    % TODO condense these three down... or maybe don't? I'll just write it all out and let steve tell me to trim it later
    Cluster Processor Module (CPM)
        Based on Barrel Cals
        For electron/photon and tau/hadron identification.
        Electrons and photons are triggered by clusters which are limited to the ECal and do not penetrate into the HCal.
        Taus and hadrons are identified by the fact that they do penetrate into the HCal.
        Both groups work by checking all possible 4x4 windows of trigger towers, and identifying windows containing an isolated ``Region of Interest" (RoI).
        An RoI is defined as a 2x2 cluster of towers with an Et sum that is a relative maximum compared to surrounding towers.
        This 2x2 RoI is the center of the 4x4 window (see figure [TODO INSERT FIGURE 13 OF L1_calo_run1]).
        Windows are considered as passing the CPM trigger if the RoI satisfies an isolation requirement, meaning that the 12 towers surrounding that core fall \textit{below} a predifined Et ``isolation threshold" value.

    Jet/Energy Module (JEM)
        Uses Barrel and Endcap, and does not distinguish between ECal and HCal
        Basic units of data collection are 2x2 collections of trigger towers called ``jet elements", resulting in minimum resolution of 0.2x0.2 in $\Delta \eta x \Delta \phi$.
        Windows can vary in size, but must be based around a core of 2x2 jet element RoI which are (as with the CPM) a local maximum in Et.
        Counts multiplicity of hits in each region and begins Et sums for the MET calculation.
        
    Common Merger Module (CMM) -> replaced by Cluster Merger Modules (CMX) in Run 2 \cite{trigger_run2}
        Uses all Cals including FCal.
        Carries out final jet multiplicity counting and Et/missing Et sums

L1 Muon Trigger \cite{trigger_run1}
    Uses the Muon Trigger Chamber %\ref{sec:muon-trigger_chamber} FIXME uncomment 
    to trigger based on particle pt and multiplicity


%\section{Level 2 Trigger} Software level trigger on L1-based ROIs. Merged with HLT in Run 2

\section{High Level Trigger}

%What
Software trigger using data from all detector elements (including the Inner Detector).

%Why
To further reduce event levels to levels permitting full offline analysis.

%When
Input rate of 100 kHz (output of L1), output rate of ~1 kHz.

%Where
Also very close to ATLAS, at the surface, in building SCX1. %see trigger_tdr figure 2

%How
The HLT performs rapid online reconstruction of event data [FIXME forward reference reconstruction here] into various physics ``signatures", with each signature corresponding to specific particles or physics processes.
The main signatures used in ATLAS are: minimum bias signatures, electron/photons (Egamma), muons, jets, taus, missing transverse energy (MET), b-jets (as in jets from bottom quarks), and B-physics (as in B-hadrons).
The process of reconstruction and the algorithms applied to these reconstructed signatures are based on offline software algorithms.
These offline algorithms are adjusted for use in the online environment, which is discussed further in the next chapter.
    


% TODO: I think I'm going to move this into the "event reconstruction" chapter, since discussion of the retuning process is entirely based on discussion of the individual low-level and high-level algorithms used for the bjet trigger
%\section{HLT Retuning Framework}
%Finally something that I did!
%Except it didn't end up mattering...
%Do I need to mention specifically my work on the FTK? NOPE. Just talk about the retuning at a more general level.
