\chapter{Trigger System} %TODO
\section{Introduction}
You've worked on this damned thing you better be able to talk about it
%What
ATLAS trigger system

%Why
To reduce overall ATLAS data output during runtime by removing low-interest events.

%How
Triggering is achieved by running events through two sequential trigger systems.
First is the hardware-based Level 1 Trigger (L1).
Then is the software-based High Level Trigger (HLT).

%When
The trigger system process all ATLAS events immediately after occurance, reducing the 40 MHz bunch crossing rate to an output rate of 1 kHz.

%Where
To reduce latency from data transfer times, all the trigger systems are located very close to the ATLAS detector system.



\section{Level 1 Trigger}

%What
Hardware level trigger

%Why
Reduce event rate acceptance to acceptable levels for more sophisticated (but slower) software level trigger

%How
uhhh? What algorithms does this use?

%When
Input rate of 40 MHz (every 25 ns, the LHC bunch crossing rate), output rate of 100 kHz (utilizes detector buffer memory to keep up). \ref{trigger_run2}

%Where
As close to ATLAS as possible in order to reduce latency, specifically in [XXX].


%\section{Level 2 Trigger} Software level trigger on L1-based ROIs. Merged with HLT in Run 2

\section{High Level Trigger}

%What
Software trigger on full event reconstruction.

%Why
To further reduce event levels to levels permitting full offline analysis.

%How
List of HLT triggers...
Also probably worth discussing the idea of ROI-based algorithms.

%When
Output rate of 1 kHz.

%Where
Also very close to ATLAS, in [XXX]



\section{HLT Retuning Framework}
Finally something that I did!
Except it didn't end up mattering...
Do I need to mention specifically my work on the FTK? NOPE. Just talk about the retuning at a more general level.
