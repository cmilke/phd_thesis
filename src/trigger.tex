\chapter{Trigger System} %TODO
\section{Introduction}
%What
ATLAS trigger system

%Why
To reduce overall ATLAS data output during runtime by removing low-interest events.

%How
Triggering is achieved by running events through two sequential trigger systems.
First is the hardware-based Level 1 Trigger (L1).
Then is the software-based High Level Trigger (HLT).
A bunch crossing passes the trigger stage if at least one of a plethora of ``trigger conditions" is satisfied, where these conditions range from detecting some combination of particles to missing Et thresholds.

%When
The trigger system process all ATLAS events immediately after occurance, reducing the 40 MHz bunch crossing rate to an output rate of 1 kHz.

%Where
To reduce latency from data transfer times, all the trigger systems are located very close to the ATLAS detector system.



\section{Level 1 Trigger}

%What
Hardware level trigger

%Why
Reduce event rate acceptance to acceptable levels for more sophisticated (but slower) software level trigger

%How
Uses only information from calorimeters and Muon Trigger Chamber


Calorimeter Algorithms: \cite{L1_calo_run1}
    Based on Calorimeter information.
    Reduces granularity of cals by clustering sensors together into ``towers".
    A ``Region of Interest" (RoI) is created as a group of clusters which collectively satisfy various conditions (varying between the trigger in question).

    Cluster Processor Module (CPM)
        Based on Barrel Cals
        For electron/photon and tau/hadron identification.
        Both groups require clusters with ET isolation (i.e.\ a region which passes the Et threshold surrounded by cluster which do not pass the threshold).
        Electrons and photons are triggered by clusters which are limited to the ECal and do not penetrate into the HCal.
        Taus and hadrons are identified by the fact that they do penetrate into the HCal.

    Jet/Energy Module (JEM)
        Uses Barrel and Endcap, and does not distinguish between ECal and HCal
        Counts multiplicity of hits in each region and begins Et sums for the MET calculation.
        
    Common Merger Module (CMM) -> replaced by Cluster Merger Modules (CMX) in Run 2
        Uses all Cals including FCal.
        Carries out final jet multiplicity counting and Et/missing Et sums

L1 Muon Trigger
    Uses the Muon Trigger Chamber %\ref{sec:muon-trigger_chamber} FIXME uncomment 
    to trigger based on particle pt and multiplicity




%When
Input rate of 40 MHz (every 25 ns, the LHC bunch crossing rate), output rate of 100 kHz (utilizes detector buffer memory to keep up). \cite{trigger_run2}

%Where
As close to ATLAS as possible in order to reduce latency, specifically in [XXX].


%\section{Level 2 Trigger} Software level trigger on L1-based ROIs. Merged with HLT in Run 2

\section{High Level Trigger}

%What
Software trigger on full event reconstruction.

%Why
To further reduce event levels to levels permitting full offline analysis.

%How
List of HLT triggers...
Also probably worth discussing the idea of ROI-based algorithms.

%When
Output rate of 1 kHz.

%Where
Also very close to ATLAS, in [XXX]



\section{HLT Retuning Framework}
Finally something that I did!
Except it didn't end up mattering...
Do I need to mention specifically my work on the FTK? NOPE. Just talk about the retuning at a more general level.
