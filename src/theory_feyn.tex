%How do we test any of this?
%From Lagrangian to Cross-Section:
%    I need to study up on exactly how you go from the lagrangian to the Feynman rules, and from there to a calcualable cross section
\section{From Theory to Experiment: The Feynman Rules and Cross-Sections} \label{sec:feyn_rules}
    
    %Justify why we care about cross sections
    After so much discussion of theory, the obvious question to ask is: how can this be tested?
    The most direct physically observable effects of the equations of the Standard Model are those of \textit{differential cross-sections}.
    Cross-sections will be discussed in more detail in Section \ref{sec:lhc-interaction_region},
        but for now it is sufficient to state that the probability of some physical interaction taking place
        is directly proportional to its differential cross-section.
    Here I will provide a general outline for how the Lagrangian of Eqn. \ref{eq:higgskappas} can be related to a measurable cross-section.

    In quantum mechanics, probabilities are defined as the absolute square of amplitudes of wave functions, $\left|\braket{\psi}\right|^2$.
    The probability of a transition between different states of a wavefunction are similarly represented
        as the absolute square of the original state, $\psi_i$, \textit{in the basis of} the final state $\psi_f$,
        written $ |\Tbraket{\psi_f}{\psi_i}|^2$.
    In QFT, states correspond to which particles are in existence at a given moment.
    Thus, a state of one electron and one anti-electron could be written as $\ket{e \bar{e}}$,
        and the transition of an electron-positron pair into a muon/anti-muon pair could be written
        as $\Tbraket{\mu \bar{\mu}}{e \bar{e}}$.

    The process used in this thesis to probe the Higgs's $\kappa$ values is that of
        vector boson fusion (VBF) to a di-Higgs pair decaying to 4b (\vbfhhproc).
    That is, two incoming quarks, $\ket{q_1 q_2}$, form the initial state of this process.
    These quarks each emit a vector boson (either $W^{\pm}$ or $Z^0$), 
        which will in turn fuse into an intermediate state of two Higgs bosons.
    After being deflected by the weak boson ejection, the quarks then continue on,
        possibly having been flavor-changed if they emitted a charged $W$.
    Meanwhile, the Higgs bosons have a very short lifetime and will decay almost immediately into one of a number of possible particles.
    The likelihood that the Higgs will decay into a given state is given by that state's \textit{branching ratio}.
    As seen from Table \ref{tab:higgsbranching}, a bottom/anti-bottom quark pair is the most likely decay product of the Higgs,
        hence its use as the final state in this analysis.
    The final state of this process consists of two Higgs and the $b \bar{b}$ pairs.
    However, since the Higgs' coupling to the bottom quark does not probe the couplings of interest in this thesis,
        for this chapter I want to focus only on the intermediate state
        consisting of two Higgs and two deflected quarks,
            $\bra{h_1 h_2 q_3 q_4}$.
    I can then write the transition of this process as $\Tbraket{ h_1 h_2 q_3 q_4}{q_1 q_2}$.
    %It should be noted that while there are other intermediate processes besides VBF
    %    that could produce these same initial and final states, 
    %    but these will not be considered in this analysis.

    \begin{table}[tbh]
\begin{center}
\caption{
    Branching ratio of the Higgs to its most common final states.
    The $b \bar{b}$ final state is dominant with over twice the branching ratio of the subleading decay mode,
        and nearly an order of magnitude higher than the sub-sub-leading decay mode\cite{particlephysicsreview2021}.
}
\label{tab:higgsbranching}
%\footnotesize
\begin{tabular}{|l|l|l|}
    \toprule
    Decay channel & Branching ratio & Rel. uncertainty  \\
    \midrule
    $ H \to b \bar{b}        $    & $5.82 \times 10^{-1} $    & $ +1.2\% \atop -1.3\% $ \\
    $ H \to W^+ W^-          $    & $2.14 \times 10^{-1} $    & $\pm 1.5\%        $   \\
    $ H \to \tau^+ \tau^-    $    & $6.27 \times 10^{-2} $    & $\pm 1.6\%        $   \\
    $ H \to c \bar{c}        $    & $2.89 \times 10^{-2} $    & $ +5.5\% \atop -2.0\% $ \\
    $ H \to ZZ               $    & $2.62 \times 10^{-2} $    & $\pm 1.5\%        $   \\
    $ H \to \gamma \gamma    $    & $2.27 \times 10^{-3} $    & $    2.1\%        $   \\
    $ H \to Z \gamma         $    & $1.53 \times 10^{-3} $    & $\pm 5.8\%        $  \\
    $ H \to \mu^+ \mu^-      $    & $2.18 \times 10^{-4} $    & $\pm 1.7\%        $  \\
    \bottomrule
\end{tabular}
\end{center}
\end{table}


    In principle, this transition process can take a finite period of time.
    In the realm of high energy physics experiments though,
        the interacting particles are moving so fast that the interaction period can be thought of as occurring at a single instant in time.
    Given this context, the initial state occurs in the (comparatively) distant past, $t_i$, and the final state in the equally distant future, $t_f$.
    Since the transition occurs at an instantaneous moment,
        I need to perform a time-translation transformation on both states to place them at the moment of the transition, $t_0$.
    Using the Hamiltonian $H \equiv i\partial_0$ as the time translation operator,
        I can relate the initial state at $t_0$ to its time $\Delta t$ units in the future, $t_i$, by the transformation
    \begin{equation}
        \ket{q_{1} q_{2} (t_i)} = e^{i\Delta tH}\ket{q_{1} q_{2} (t_0)}
        \,.
    \end{equation}
    The same can be done to transform the final state backwards in time
    \begin{equation}
        \bra{h_1 h_2 q_{3} q_{4} (t_f)}
        = \bra{h_1 h_2 q_{3} q_{4} (t_0)} (e^{i(-\Delta t)H})^\dag
        = \bra{h_1 h_2 q_{3} q_{4} (t_0)} e^{i\Delta tH}
        \,.
    \end{equation}
    Putting both of these together yields
    \begin{equation} \begin{split} \label{eq:transition_amplitude}
        \Tbraket{ h_1 h_2 q_{3} q_{4} (t_f)}{q_{1} q_{2} (t_i)}
        &= \TbraketA{ h_1 h_2 q_{3} q_{4} (t_0)}{e^{i\Delta tH} e^{i\Delta tH}}{q_{1} q_{2} (t_0)}
        \\&= \TbraketA{ h_1 h_2 q_{3} q_{4} (t_0)}{e^{2\Delta tH}}{q_{1} q_{2} (t_0)}
        \\&= \TbraketA{ h_1 h_2 q_{3} q_{4} (t_0)}{1 + iT}{q_{1} q_{2} (t_0)}
        \,.
    \end{split} \end{equation}

    In the last step, the exponential operator is expanded as an infinite series of terms,
        in a reversal of the procedure from Section \ref{sec:group_theory}.
    The first of these terms will just be 1, corresponding to the static situation in which no interaction occurs at all.
    The sum of the remaining terms, represented as $iT$, is the part relevant for calculating the interaction probability,
        with the entire transition amplitude referred to as the \textit{invariant amplitude}, $\invAmp$:
    \begin{equation}
        i\invAmp \equiv \TbraketA{ h_1 h_2 q_{3} q_{4} (t_0)}{iT}{q_{1} q_{2} (t_0)}
        \,.
    \end{equation}

    As stated above, the differential cross-section, $\dXsec$,
        of a process is proportional to its probability $\mathcal{P}$,
        which in turn can be related to the invariant amplitude
    \begin{equation} \begin{split}
        \dXsec &\propto \mathcal{P}( q_{1} q_{2} \rightarrow q_{3} q_{4} h_1 h_2 ) 
            = \left| \TbraketA{ h_1 h_2 q_{3} q_{4} (t_0)}{iT}{q_{1} q_{2} (t_0)} \right|^2 
            = |i \invAmp|^2 
        \\\dXsec &= \Gamma(p_1, p_2, ...) |\invAmp|^2
        \,.
    \end{split} \end{equation}

    Where $\Gamma(p_1, p_2, ...)$ is a function of the scattering kinematics,
        e.g.\ the particles' crossing angle, energies, and momenta, etc.
    These kinematics are mostly related to the properties of the setup of the scattering experiment under consideration.
    All dependence on the Standard Model Lagrangian is contained within the $\invAmp$ term,
        and the remainder of this chapter will be devoted to its calculation.
    To do this, I turn to Feynman diagrams.

    Calculation of the transition expectation value of the $iT$ term has historically been a very technical challenge.
    Feynman diagrams are an elegant tool for performing this task more easily and intuitively.
    The general process to using them begins with the following steps:
    \begin{itemize}
        \item Draw the initial and final states of the process in question as dots
        \item Fully connect the initial and final state particles using any \textit{valid} intermediate lines and vertices
        \item Valid connections can be identified through the terms present in the Lagrangian:
        \begin{itemize}
            \item Kinetic energy terms correspond to lines connecting a particle to itself
            \item Interaction terms correspond to interaction vertices, connecting three or more particles at a time
            \item All vertices must ensure conservation of any relevant quantum number
        \end{itemize}
        \item Repeat the above steps in order to draw all possible diagrams
    \end{itemize}

    This final step may seem impossible, given that an infinite number of intermediate particles can be inserted between any two states.
    Recall from the structure of $\invAmp$, that $iT$ is not one term, but in fact an infinite series expansion of the Hamiltonian operator.
    For most situations, each higher order of the terms will contribute less to the overall calculation.
    Eventually, the higher-order terms will contribute so little that they can be safely ignored.
    This property is directly reflected in the Feynman diagrams, in the form of \textit{loops}.
    Any loop in a diagram indicates that the diagram is a higher order term of the expansion.
    A diagram with no loops -- referred as ``tree-level'' or ``leading order'' (LO) -- is part of the first term of $iT$.
    Diagrams with one loop are part of the second term (next-to-leading order, or NLO),
        two loops the third (next-to-next-to-leading order, or NNLO), and so forth.

    One need only draw as many diagrams as is needed for the level of desired precision.
    As a side note, just as successive loops correspond to higher order terms in $iT$,
        diagrams which are not fully connected correspond to the ``1'' term from the original $1+iT$ in Eqn. \ref{eq:transition_amplitude},
        hence why disconnected diagrams are ignored entirely.

    \begin{figure}
    \centering
    \begin{subfigure}{0.32\textwidth} 
        \resizebox{0.9\textwidth}{!}{
\begin{tikzpicture} \begin{feynman}
    \vertex (kv1) {$\kv$};
    \vertex [below=of kv1] (kv2) {$\kv$};
    \vertex [right=of kv1] (h1) {$h_1$};
    \vertex [right=of kv2] (h2) {$h_2$};
    \vertex [above left=of kv1] (vb1);
    \vertex [below left=of kv2] (vb2);
    \vertex [left=of vb1] (q1) {$q_{1}$};
    \vertex [left=of vb2] (q2) {$q_{2}$};

    \vertex [above=of h1] (q3) {$q_{3}$};
    \vertex [below=of h2] (q4) {$q_{4}$};

    \diagram* {
        (q1) -- (vb1) -- (q3),
        (q2) -- (vb2) -- (q4), 
        (vb1) -- [boson] (kv1) -- [boson] (kv2)-- [boson] (vb2),
        (h1) -- [scalar] (kv1),
        (h2) -- [scalar] (kv2),
    };
\end{feynman} \end{tikzpicture}
}
 
        \caption{$M_t$}
        \label{fig:tree_level_vbfhh:kv}
    \end{subfigure}
    \begin{subfigure}{0.32\textwidth}
        \begin{tikzpicture} \begin{feynman}
    \vertex (kv) {$\kv$};
    \vertex [right=of kv] (kl) {$\kl$};
    \vertex [above right=of kl] (h1) {$h_1$};
    \vertex [below right=of kl] (h2) {$h_2$};
    \vertex [above left=of kv] (vb1);
    \vertex [below left=of kv] (vb2);
    \vertex [left=of vb1] (q1i) {$q_{i1}$};
    \vertex [left=of vb2] (q2i) {$q_{i2}$};

    \vertex [above=of h1] (q1f) {$q_{f1}$};
    \vertex [below=of h2] (q2f) {$q_{f2}$};

    \diagram* {
        (q1i) -- (vb1) -- (q1f),
        (q2i) -- (vb2) -- (q2f), 
        (vb1) -- [boson] (kv) -- [boson] (vb2),
        (kv) -- [scalar] (kl),
        (h1) -- [scalar] (kl) -- [scalar] (h2),
    };
\end{feynman} \end{tikzpicture}
 
        \caption{$M_s$}
        \label{fig:tree_level_vbfhh:kl}
    \end{subfigure}
    \begin{subfigure}{0.32\textwidth}
        \resizebox{0.8\textwidth}{!}{
\begin{tikzpicture} \begin{feynman}
    \vertex (k2v) {$\kvv$};
    \vertex [above right=of k2v] (h1) {$h_1$};
    \vertex [below right=of k2v] (h2) {$h_2$};
    \vertex [above left=of k2v] (vb1);
    \vertex [below left=of k2v] (vb2);
    \vertex [left=of vb1] (q1i) {$q_{i1}$};
    \vertex [left=of vb2] (q2i) {$q_{i2}$};

    \vertex [above=of h1] (q1f) {$q_{f1}$};
    \vertex [below=of h2] (q2f) {$q_{f2}$};

    \diagram* {
        (q1i) -- (vb1) -- (q1f),
        (q2i) -- (vb2) -- (q2f), 
        (vb1) -- [boson] (k2v) -- [boson] (vb2),
        (h1) -- [scalar] (k2v) -- [scalar] (h2),
    };
\end{feynman} \end{tikzpicture}
}
 
        \caption{$M_x$}
        \label{fig:tree_level_vbfhh:k2v}
    \end{subfigure}
    \caption{Tree-level diagrams of the \hhproc process.}
    \end{figure}

    Once all possible diagrams have been drawn to the order desired, each diagram has a value assigned to it.
    ``Feynman Rules'' are the rules governing how these values are assigned.
    The rules are derived based on the structure of the Lagrangian and are rather extensive;
        hence, they will not be listed here.
    After the values have been determined, the invariant amplitude is trivially calculated as the sum of the values of all drawn diagrams.
    The \hhproc process studied here is primarily done at tree-level
        (Fig. \ref{fig:tree_level_vbfhh:kv}-\ref{fig:tree_level_vbfhh:k2v}),
        but N\textsuperscript{3}LO calculations are also used to some degree.

    Finally, the differential cross-section can be calculated as the absolute square of the sum of the Feynman diagrams.
    There is one critical (to this analysis) detail of the Feynman rules I will mention,
        which is that each diagram's value $M$ is proportional to the \textit{product of the coefficients}
        associated with each interaction vertex in the diagram.
    Take for example Fig. \ref{fig:tree_level_vbfhh:kv}, which contains four total vertices;
        two corresponding to the Higgs-vector boson interaction (whose coefficient based on Eqn. \ref{eq:higgskappas} is $\kv q_V$),
        and two corresponding to the quark-vector boson interaction (whose coefficient is $c_{qV}$).
    As per the Feynman rules, this means that $M_t$ is proportional to $\kv q_V \kv q_V c_{qV} c_{qV}$.
    The quantities I am interested in are of course the $\kappa$ values,
        so I can pull these values out in front of the invariant amplitude as $M_t \rightarrow \kv^2 M_t$.
    At tree-level, I can then write out the absolute square of the invariant amplitude as
    \begin{equation} \begin{split} \label{eq:tree_level_invamp}
        |\invAmp|^2 &= |  \kv^2 M_t + \kv \kl M_s + \kvv M_x |^2
        \\&= \kv^2 \kl^2 M_s^2 + \kv^4 M_t^2 + \kvv^2 M_x^2 
            \\&\qquad + \kv^3 \kl (M_s^* M_t + M_t^* M_s) 
            \\&\qquad + \kv \kl \kvv (M_s^* M_x + M_x^* M_s ) 
            \\&\qquad + \kv^2 \kvv (M_t^* M_x + M_x^* M_t )
        \\\dXsec \propto |\invAmp|^2 &= \kv^2 \kl^2 a_1 + \kv^4 a_2 + \kvv^2 a_3 + \kv^3 \kl a_4 + \kv \kl \kvv a_5 + \kv^2 \kvv a_6
        \,.
    \end{split} \end{equation}

    As noted earlier, the precise value of this cross-section will vary depending on the kinematics of the incoming particles,
        most notably their center of mass (CoM) energy.
    But for this process to occur at all, the CoM energy must be at least the combined mass of the two Higgs bosons (250 GeV).
    Only a handful of facilities in the world are capable of producing energies at this scale,
        and to produce di-Higgs processes with any appreciable abundance,
        even higher energies are required.
    For measuring the Higgs self-couplings, there is only one machine on the planet that will suffice.
