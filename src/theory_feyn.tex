%How do we test any of this?
%From Lagrangian to Cross-Section:
%    I need to study up on exactly how you go from the lagrangian to the Feynman rules, and from there to a calcualable cross section
\section{From Theory to Experiment: The Feynman Rules and Cross-Sections}
    
    %Justify why we care about cross sections
    After all this discussion of theory, the obvious question to ask is: how can this be tested.
    The most direct physically observable effects of the equations of the Standard Model are those of \textit{cross-sections}.
    Cross-sections will be discussed in more detail in Section \ref{sec:lhc-interaction_region},
        but for now it is sufficient to state that the probability of some physical interaction taking place is directly proportional to its cross-section.
    Here I will provide a general outline for how to produce a measurable cross-section from the Lagrangian of Equation \ref{eq:higgskappas}.

    In quantum mechanics, probabilities are measured as the absolute square of amplitudes of wave functions, $\left|\braket{\psi}\right|^2$.
    The probability of a transition between different states of a wavefunction are simililarly represented
        as the absolute square of the original state, $\psi_i$, \textit{in the basis of} the final state $\psi_f$,
        written $ |\Tbraket{\psi_f}{\psi_i}|^2$.
    In Quantum Field Theory, states correspond to which particles are in existence at a given moment.
    Thus, a state of one electron and one anti-electron could be written as $\ket{e \bar{e}}$,
        and the transition of an electron-positron pair into a muon/anti-muon pair could be written
        as $\Tbraket{\mu \bar{\mu}}{e \bar{e}}$.

    The core process used in this paper to probe the Higgs' $\kappa$ values is that of Vector Boson Fusion to two Higgs Bosons.
    The initial state of this process is two incoming quarks, $\ket{q_{i1} q_{i2}}$.
    These quarks each emit a vector boson (either $W^{\pm}$ or $Z^0$), 
    which should in turn fuse into two Higgs Bosons.
    The initial quarks then continue on, significantly deflected by their emissions, and possibly flavor-changed if they emitted a charged $W$.
    The final state of this process thus consists of two Higgs and two deflected quarks, $\bra{h_1 h_2 q_{f1} q_{f2}}$.
    The transition of this process would then be written as $\Tbraket{ h_1 h_2 q_{f1} q_{f2}}{q_{i1} q_{i2}}$.
    It should be noted that there are other intermediate processes (besides VBF)
        that could produce these same initial and final states, but these will not be considered in this analysis.

    In principle, this transition process can take an indeterminate period of time.
    In the realm of high energy physics experiments though,
        the interacting particles are moving so fast that the interaction period can be thought of as occuring at a single instant in time.
    Given this context, the initial state occurs in the (comparitively) distant past ($t_i$), and the final state in the equally distant future ($t_f$).
    Since the transistion occurs at an instantaneous moment,
        I need to perform a time-translation transformation both states to place them at the moment of the transition ($t_0$).
    Using the Hamiltonian $H$ as the time translation operator,
        I can relate the initial state at $t_0$ to its time $\Delta t$ units in the future, $t_i$, by the tranformation
    \begin{equation}
        \ket{q_{i1} q_{i2} (t_i)} = e^{i\Delta tH}\ket{q_{i1} q_{i2} (t_0)}
    \end{equation}
    The same can be done to transorm the final state backwards in time
    \begin{equation}
        \bra{h_1 h_2 q_{f1} q_{f2} (t_f)}
        = \bra{h_1 h_2 q_{f1} q_{f2} (t_0)} (e^{i(-\Delta t)H})^\dag
        = \bra{h_1 h_2 q_{f1} q_{f2} (t_0)} e^{i\Delta tH}
    \end{equation}
    Putting both of these together yields
    \begin{equation} \begin{split}
        \Tbraket{ h_1 h_2 q_{f1} q_{f2} (t_f)}{q_{i1} q_{i2} (t_i)}
        &= \TbraketA{ h_1 h_2 q_{f1} q_{f2} (t_0)}{e^{i\Delta tH} e^{i\Delta tH}}{q_{i1} q_{i2} (t_0)}
        \\&= \TbraketA{ h_1 h_2 q_{f1} q_{f2} (t_0)}{e^{i2\Delta tH}}{q_{i1} q_{i2} (t_0)}
        \\&= \TbraketA{ h_1 h_2 q_{f1} q_{f2} (t_0)}{1 + iT}{q_{i1} q_{i2} (t_0)}
    \end{split} \end{equation}

    In the last step, the exponential operator is expanded as an infinite series of terms.
    The first of these terms will just be 1, corresponding to the static situation in which no interaction occurs at all.
    The sum of the remaining terms, represented as $iT$, is the part relevant for calculating the interaction probability.
    Calculation of the value $\TbraketA{ h_1 h_2 q_{f1} q_{f2} (t_0)}{iT}{q_{i1} q_{i2} (t_0)}$ involves both
        the kinematics of the incoming and outgoing particles as well as the physics involved in the interaction itself.
    As such, it is useful to fuck all this I'm just going to skip the stupid transition amplitude stuff because
        there's no clear transition from transition amplitude to cross section.




    




    %basic formalism of cross-sections calculation as vacuum to particles (is that right?) to produce matrix elements

    %individual matrix elements as different internal processes

    %feynman diagrams as representation of matrix elements.
    %explain how feynman diagrams work.
    %Explain how I'm not going to deal with loop corrections...
    %    except maybe I should briefly mention this?
    %We do use N3LO for SM...

    %Explain how feynman rules produce matrix element values for each diagram.

    %Cover the main three diagrams for VBF->HH

    %Explain how these cross-section values go up with energy. Thus LHC
    %Mention that cross-section will be explained more later.
    %Important thing now is just to note that higher cross-section = more likely that process will occur

