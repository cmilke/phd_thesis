\chapter{The LHC and ATLAS}\label{chapter:experiment}

Here I should justify why this chapter needs to exist. As said in intro, this is basically the "experimental setup" bit. This is where I explain the tools and setup I am using to do my experiment, so people have context and in theory could reproduce these results in the future.


\section{LHC}

% where is it, why was it even built, who built it.
The Large Hadron Collider (LHC) is a massive particle accelerator located near Geneva, Switzerland.
It was constructed by CERN (European Organization for Nuclear Research, from the French \textit{Conseil Européen pour la Recherche Nucléaire}) for the purpose of studying high-energy physics well above energies observed in any previous experiment.

Construction took place between 1995 and 2007, 




% probably should explain the Run 1/2/3/etc stuff, since things change for that. I can incorporate this into the LHC's overall history: when did construction start, finish, when was it first turned on
% also luminosity (I think I might include a discussion of wtf luminosity/cross-section means in the appendices)
% as well as integrated luminosity per run

% ok, so why tf am i writing this bit? This is for someone who either doesn't know wtf the lhc/atlas are, or has just joined the group and needs to taught the basics of how this experiment even works. 
% what does it do
% technical specs (size, energy* (discuss this with the "Run" system, since it changes), number of detectors, what does it collide)
% more specs (collision rate, what is a 'bunch crossing', particles in a bunch, strength of magnets?)

\section{ATLAS}

ATLAS is like a giant camera on that ring.

% The general structure I've seen everybody else using is to essentially describe ATLAS one subdetector at a time, I think from the innermost tracker to the outermost calorimeter
% Also need to discuss the trigger system
