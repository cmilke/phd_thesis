\section{Introduction and Basics: The Composition of the Universe} 

    What is the Universe made of (If you stick to this format at all, keep this section brief):

    \subsection{Matter and Forces}
       I need to discuss the elementary particles and their organization before I can really discuss the gauge fields.
       Might as well do it here first and foremost

    \subsection{Position and Momentum: Quantum Mechanics}
       The position and momentum of that matter:
           Described via the Canonical Commutation Relation and the formalism of Quantum Mechanics
           (explain 'h' here?)
       Describe basic function of schrodinger equation and why it fails

    \subsection{Space and Time: Special Relativity}
       The space-time in which that matter resides: described by the Minkowski Metric Tensor.
        Explain what the metric tensor means and maybe covariant notation,
        as well as mass-energy-momentum equivalence,
        and also the speed of light
        Reformat schrodinger equation as klein gordon and show how this also fails


    \subsection{Particles and Fields: Quantum Field Theory}
       A description of how the position of matter can change: Described by the Dirac Equation.
        A unification of Quantum Mechanics and Special Relativity
        (maybe go through the derivation, starting from schrodinger -> klein-gordon -> dirac and why each fails)
        Describe Weyl Spinors and Chiral representation (peskin pg 64)


\section{The Formalism of Quantum Field Theory}
    things I'm thinking of talking about but I'm not sure I actually need to discuss and also I'm not sure which section to put them in:

        The poincare group transformations and their conserved quantities
            
            Translation: $\psi(x) \rightarrow e^{-\Delta x^{\mu} \partial_{\mu}} \psi(x)$

            $U(1)$ Global: $\psi(x) \rightarrow e^{i \alpha} \psi(x)$

            $U(1)$ Local: $\psi(x) \rightarrow e^{i \alpha(x)} \psi(x)$

            $SU(2)$ Global:  $\psi_i(x) \rightarrow  U(\epsilon)_{ij} \psi_j$; $U_{ij} \equiv e^{i \epsilon_a (h_{a,ij} / 2)}$; $a=1,2,3$
                $h_a$ are the pauli matrices
                %- (cheng djvu pg 97 for details)

            $SU(2)$ Local:  $\psi_i(x) \rightarrow  e^{i \epsilon_a(x) (h_{a,ij} / 2)} \psi_j$
                %- (cheng djvu pg 238 for details)

            $SU(3)$ Global:  $\psi_i(x) \rightarrow  U(\epsilon)_{ij} \psi_j$; $U_{ij} \equiv e^{i \epsilon_a (\lambda{a,ij} / 2)}$; $a=1,..., 8$
                $\lambda_a$ are 3x3 traceless hermitian matrices
            

        what does a basic transformation look like. Show how you go from (1+da/dq) to e\^(dq*da/dq).



    \subsection{Generating Motion: Group Theory and Transformations}

        The Dirac field is entirely static. We need a way to make it dynamic. "Group" operations provide this
        
        Ok I guess I need a section on this because I -forgot- never actually understood how the representation crap works...

        A ``group'' is a set of elements which can be ``multiplied'' according to some rule,
            and which satisfies the four conditions of:
                \begin{itemize}
                    \item Closure - the product of any two elements of the group are still in that group;
                    \item Associativity - $(a \times b)\times c = a\times(b \times c)$;
                    \item Identity - there is some element in the group $I$ for which $I \times a=a$;
                    \item and Inversion - every element $a$ has an inverse $a^{-1}$ such that if $b \times a = c$ then $c \times a^{-1} = b$.
                \end{itemize}

        If the group operation is commutative ($ab=ba$) then the group is ``Abelian''; if not, it is ``non-Abelian''.

        Ok now for the representation stuff. [Sredneki draft pg411]

        \cite{Cheng_book}

    \subsection{Restricting Motion: The Lagrangian and Symmetry}

        Explain how motion is described by a lagrangian of fields. 
        Minization of Action.
        Equations of motion derived from Euler-Lagrange Equations.
        The lagrangian takes the form it does in order to satisfy poincare symmetry.

        Discuss Noether's Theorem; refer to Halzen pg 314 (djvu=331) or, maybe better, Sredneki pg 144).
            ... Wait, do I even need to mention noether's theorem?
            I don't think it's actually a factor in producing the higgs... which makes me slightly sad :-(

        I also should maybe end this with a discussion of how forces aren't really a thing
            and particles in field theory exchange momentum by merely bumping into other particles.
        Maybe, *maybe**** (maybe not) give a toy example lagrangian showing a particle which can interact with itself via a three-point vertex or something.
        Note the basic symmetries that the basic lagrangian must satisfy (hence group theory going first)
        \cite{Halzen_book}


    \subsection{Transferring Motion: Gauge Symmetry}

        Gauge symmetries; U1, SU(N).
        The effects of imposing gauge symmetries on the Lagrangian, and the advent of the gauge bosons and their forces.
        how do gauge bosons come out from symmetries.

        U(1) is phase transforms

        SU(2) is based on 2x2 pauli matrices and thus requires pairing generations of particles together;
            so up and down-type quarks are paired together and charged leptons with neutrinos.
        It treats left handed fields as these doublet pairs,
            but works in "singlet representation" (which means it basically is just gone) for right-handed fields

        SU(3) is based on a 3x3 structure constant, and thus acts on all three "generations" of quarks as one 3x1 vector.
        It is a singlet (read, it literally doesn't matter) for leptons.
        
        \cite{Osborn_notes}
        \cite{Peskin_book}
        \cite{Halzen_book}
