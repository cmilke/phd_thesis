\section{Introduction}

    The Higgs Boson sits as the crown jewel of a grand overarching theory of the behaviour of the universe,
        known as the Standard Model of Particle Physics.
    While its recent discovery has shed light on many of its key properties,
        there are still many details of its nature that are as yet uncomfirmed.
    In this chapter, I want to explain what these properties are and how they can be further studied.
    Moreover, I want to justify why the Higgs is so important as to be worth studying in the first place.
    And to understand the importance of the Higgs Boson, one must understand the structure of the Standard Model itself.

    The structure of the following sections will begin with a discussion of the purpose and fundamental structure of the Standard Model.
    I will follow this with an introduction to the mathematical formalism the Standard Model is based around,
        involving Group Theory, the calculus of variations, and symmetry.
    Here I will introduce the reason for the original postulation of the Higgs, what it is, how it works, and how it fits into the Standard Model.
    Finally, I will motivate further study into the Higgs and provide a technique to perform this study.


%Matter and Forces
%I need to discuss the elementary particles and their organization before I can really discuss the gauge fields.
%Might as well do it here first and foremost
\section{The Standard Model of Particle Physics}
    
    At its core, the Standard Model of Particle Physics is a description of the behaviour and interaction of matter.
    First and foremost then, I want to discuss what this matter actually is.
    All matter can be described as a specific type of elementary particle called a fermion,
        defined by the fact that it contains no discernable substructure and possesses an intrinsic spin of 1/2.
    These particles all have different masses, and are able to interact with each other through three ``fundamental interactions''.
    The three fundamental interactions (more commonly called ``Fundamental Forces'') are known as
        the Electromagnetic, Weak, and Strong interactions (gravity is entirely absent in the Standard Model).
    All the interactions have an associated ``charge'' which can be ascribed to different particles,
        and which govern how strongly that particle can interact with similarly charged particles.
    The various fermions are distinct largely because of the charges they carry.

    \begin{figure}[h!]
        \includegraphics[width=\linewidth,height=\textheight,keepaspectratio]{theory/Standard_Model_of_Elementary_Particles}
        \caption{I'm probably going to need to find something else since this came from wikipedia. I just wanted a placeholder}
        \label{fig:sm_particles}
    \end{figure}
        

    There are twelve distinct elementary fermions (see Figure \ref{fig:sm_particles}),
        which are split evenly into two subgroups, called quarks and leptons.
    Quarks have a charge of 1 with the Strong interaction,
        while leptons have a charge of 0 (and thus cannot interact via the Strong Interaction at all).
    Both classes of particles, quarks and leptons, are divided into three ``generations'' of progressively heavier particles.
    Each generation thus consists of two quarks and two leptons.
    These pairs, called ``doublets'', behave the same across all generations.
    Among the quarks, every generation contains a doublet of an up-type quark (Up, Charm, Top) with electromagnetic charge of 2/3,
        and a down-type quark (Down, Strange, Bottom) with EM charge of -1/3.
    For leptons, each doublet consists of a particle with EM charge of -1, and a neutrino with EM charge of 0.

    In addition to the fermions, there is also an entirely seperate class of particles, called gauge bosons,
        which play a fundamental role in the aforementioned interactions.
    However, the nature of these particles will be discussed later.
    Now, with the enumeration of the various particles complete, it is time to begin the discussion of the Standard Model itself,
        which will serve to explain how these fermions interact with each other.



\section{Composition of Matter: Fields and Dirac Spinors}
    %What is the Universe made of (If you stick to this format at all, keep this section brief):

    %%Position and Momentum: Quantum Mechanics
    %   The position and momentum of that matter:
    %       Described via the Canonical Commutation Relation and the formalism of Quantum Mechanics
    %       (explain 'h' here?)
    %   Describe basic function of schrodinger equation and why it fails

    %%Space and Time: Special Relativity
    %   The space-time in which that matter resides: described by the Minkowski Metric Tensor.
    %    Explain what the metric tensor means and maybe covariant notation,
    %    as well as mass-energy-momentum equivalence,
    %    and also the speed of light
    %    Reformat schrodinger equation as klein gordon and show how this also fails


    %%Particles and Fields: Quantum Field Theory
    %   A description of how the position of matter can change: Described by the Dirac Equation.
    %    A unification of Quantum Mechanics and Special Relativity
    %    (maybe go through the derivation, starting from schrodinger -> klein-gordon -> dirac and why each fails)
    %    Describe Weyl Spinors and Chiral representation (peskin pg 64)




\section{Generating Motion: Group Theory and Transformations}

    The Dirac field is entirely static. We need a way to make it dynamic. "Group" operations provide this
    
    Ok I guess I need a section on this because I -forgot- never actually understood how the representation crap works...

    A ``group'' is a set of elements which can be ``multiplied'' according to some rule,
        and which satisfies the four conditions of:
            \begin{itemize}
                \item Closure - the product of any two elements of the group are still in that group;
                \item Associativity - $(a \times b)\times c = a\times(b \times c)$;
                \item Identity - there is some element in the group $I$ for which $I \times a=a$;
                \item and Inversion - every element $a$ has an inverse $a^{-1}$ such that if $b \times a = c$ then $c \times a^{-1} = b$.
            \end{itemize}

    If the group operation is commutative ($ab=ba$) then the group is ``Abelian''; if not, it is ``non-Abelian''.

    Ok now for the representation stuff. [Sredneki draft pg411]

    \cite{Cheng_book}

\section{Restricting Motion: The Lagrangian and Symmetry}

    Explain how motion is described by a lagrangian of fields. 
    Minization of Action.
    Equations of motion derived from Euler-Lagrange Equations.
    The lagrangian takes the form it does in order to satisfy poincare symmetry.

    Discuss Noether's Theorem; refer to Halzen pg 314 (djvu=331) or, maybe better, Sredneki pg 144).
        ... Wait, do I even need to mention noether's theorem?
        I don't think it's actually a factor in producing the higgs... which makes me slightly sad :-(

    I also should maybe end this with a discussion of how forces aren't really a thing
        and particles in field theory exchange momentum by merely bumping into other particles.
    Maybe, *maybe**** (maybe not) give a toy example lagrangian showing a particle which can interact with itself via a three-point vertex or something.
    Note the basic symmetries that the basic lagrangian must satisfy (hence group theory going first)
    
    Should I also discuss renormalizability? (Peskin pg 80/101djvu)
    Basically, all lagrangians must be renormalizable.
    Renormalizability just means that the lagrangian doesn't explode from the unconstrained nature of virtual particles.
    So infinite-mass virtual particles should not break a renormalizeable lagrangian.
    \cite{Halzen_book}


\section{Transferring Motion: Gauge Symmetry}
    Gauge Transformations are ones where the transformation is imposed differently at each point in spacetime.
    Trying to impose a constraint on the Lagrangian that it be symmetric under gauge transformations would surely cause all manner of complications.
    Obviously, this is exactly what nature seems to have chosen to do.

    Gauge symmetries; U1, SU(N).
    The effects of imposing gauge symmetries on the Lagrangian, and the advent of the gauge bosons and their forces.
    how do gauge bosons come out from symmetries.

    U(1) is phase transforms

    SU(2) is based on 2x2 pauli matrices and thus requires pairing generations of particles together;
        so up and down-type quarks are paired together and charged leptons with neutrinos.
    It treats left handed fields as these doublet pairs,
        but works in "singlet representation" (which means it basically is just gone) for right-handed fields

    SU(3) is based on a 3x3 structure constant, and thus acts on all three "generations" of quarks as one 3x1 vector.
    It is a singlet (read, it literally doesn't matter) for leptons.
    
    \cite{Osborn_notes}
    \cite{Peskin_book}
    \cite{Halzen_book}

