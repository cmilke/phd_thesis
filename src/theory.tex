\chapter{Theory}\label{chapter:experiment}

Describe the SM using a layered approach: 

\section{Introduction and Basics}

    layer 1 covers everything, but without math and at a lay-person level.
    Layer 2 delves into the deep math.

    Start with a basic explanation of forces as simple exchanges of momenta between particles;
    different forces simply correspond to which particles can interact with each other


    For the deep dive, let's break things into two parts:
        The fundamental stuff the universe is made of; Matter, existing in Spacetime, with a specific Position and Momentum
        The rules governing how the position and momentum of matter can change


\section{The Composition of the Universe}

    What is the Universe made of:

    \subsection{Matter: Elementary Particles}

    1) Matter: the array of leptons and quarks and all of their associated masses. A general discussion of the concept of flavour and generation.
        Also the mass matrix, but we can't really use that yet.

    \subsection{Position and Momentum: The Formalism of Quantum Mechanics}

    2) The position and momentum of that matter: Described via the Canonical Commutation Relation and the formalism of Quantum Mechanics (obviously explain 'h' here)

    \subsection{Space and Time: Special Relativity}

    3) The space-time in which that matter resides: described by the Minkowski Metric Tensor.
        Explain what the metric tensor means and maybe covariant notation,
        as well as mass-energy-momentum equivalence,
        and also the speed of light



\section{A Dynamic Universe: The Rules Governing Change}

    How do momentum and position change, and how do particles interact.

    \subsection{Changing Position: The Dirac Equation}

    1) A description of how the position of matter can change: Described by the Dirac Equation.
        Maybe leave out how this is derived, but at least explain the spin and anti-matter implications of it

    \subsection{Changing Momentum: The Lagrangian Formalism and Symmetry}

    2) A description of how the momentum of matter can change: Provided by the Lagrangian formalism,
        with the methods by which momentum may change inscribed in a set of symmetries imposed on the Lagrangian;
        Poincare and CPT invariance (discuss Noether's Theorem in the appendix)

    \subsection{Exchanging Momentum: Gauge Symmetries}

    3) The more specific symmetries:
        Gauge symmetries; U1, SU2, and SU3 (or, if I want to be bold, U1xSU2 and SU3 only).
        The effects of imposing gauge symmetries on the Lagrangian, and the advent of the gauge bosons and their forces.
        ( in appendix, how do gauge bosons come out from symmetries )


        ... there is beauty in that which is conflicting, that is under tension, that is \textit{asymmetric}.

    \subsection{Breaking Symmetry: The Higgs Mechanism}

    4) The Broken Symmetry and the origin of mass:
        The splitting of the electroweak symmetry into the weak and EM forces. You should see if you can tie c2v, kl, and kv in to the shape of the higgs potential here

    \subsection{The Universe Unified: The Standard Model}

    5) Putting everything together: The Standard Model Lagrangian, the mass matrix, and the particle zoo
        Either here or in appendix: how do forces arise from particle interactions (Lagrangian -> Euler-Lagrange equations -> equations of motion)
        Also definetly need to explain and summarize the SM table of particles: fermions, bosons, quarks, leptons, etc.
        (Do I need a discussion of fermi-dirac statistics? Bose-Einstien? If so, where should that go...)


\section{Using and Testing the Standard Model}

    From theory to experiment: How do we test any of this?
        
    From Lagrangian to Cross-Section:
        This is multiple sections, but I need to study up on exactly how you go from the lagrangian to the Feynman rules, and from there to a calcualable cross section



