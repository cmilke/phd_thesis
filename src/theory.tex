\chapter{Theory}\label{chapter:experiment}

So what even do I need to talk about here?

Final goal = Higgs (final final goal should actually probably be di-higgs and a discussion of the relevence of this...)
    What: What is the Higgs?
    Why: Why do we care about it? It gives things mass and messes with the electroweak interaction
    When: In what contexts does the Higgs become relevant? The mass of (most) elementary particles and the weak force being weak
    Where: Where does the higgs fit into the SM? The "Higgs Mechanism"
    How: How does the higgs mechanism work?



%\section{Introduction and Basics: The Composition of the Universe} ...It looks like Matthew doesn't mention any of this. Maybe I can skip this part entirely?
%
%    What is the Universe made of (If you stick to this format at all, keep this section brief):
%
%    \subsection{Space and Time: Special Relativity}
%       The space-time in which that matter resides: described by the Minkowski Metric Tensor.
%        Explain what the metric tensor means and maybe covariant notation,
%        as well as mass-energy-momentum equivalence,
%        and also the speed of light
%
%    \subsection{Position and Momentum: Quantum Mechanics}
%       The position and momentum of that matter:
%           Described via the Canonical Commutation Relation and the formalism of Quantum Mechanics
%           (explain 'h' here?)
%
%
%    \subsection{Particles and Fields: Quantum Field Theory}
%       A description of how the position of matter can change: Described by the Dirac Equation.
%        A unification of Quantum Mechanics and Special Relativity
%        (maybe go through the derivation, starting from schrodinger -> klein-gordon -> dirac and why each fails)
%        Describe Weyl Spinors and Chiral representation (peskin pg 64)
%

\section{The Formalism of Quantum Field Theory}
    things I'm thinking of talking about but I'm not sure I actually need to discuss and also I'm not sure which section to put them in:

        The poincare group transformations and their conserved quantities
            
            Translation: $\psi(x) \rightarrow e^{-\Delta x^{\mu} \partial_{\mu}} \psi(x)$

            $U(1)$ Global: $\psi(x) \rightarrow e^{i \alpha} \psi(x)$

            $U(1)$ Local: $\psi(x) \rightarrow e^{i \alpha(x)} \psi(x)$

            $SU(2)$ Global:  $\psi_i(x) \rightarrow  U(\epsilon)_{ij} \psi_j$; $U_{ij} \equiv e^{i \epsilon_a (\sigma_{a,ij} / 2)}$; $a=1,2,3$
                $\sigma_a$ are the pauli matrices
                %- (cheng djvu pg 97 for details)

            $SU(2)$ Local:  $\psi_i(x) \rightarrow  e^{i \epsilon_a(x) (\sigma_{a,ij} / 2)} \psi_j$
                %- (cheng djvu pg 238 for details)

            $SU(3)$ Global:  $\psi_i(x) \rightarrow  U(\epsilon)_{ij} \psi_j$; $U_{ij} \equiv e^{i \epsilon_a (\lambda{a,ij} / 2)}$; $a=1,..., 8$
                $\lambda_a$ are 3x3 traceless hermitian matrices
            

        what does a basic transformation look like. Show how you go from (1+da/dq) to e^(dq*da/dq).



    \subsection{Generating Motion: Group Theory and Transformations}

        The Dirac field is entirely static. We need a way to make it dynamic. "Group" operations provide this
        
        Ok I guess I need a section on this because I -forgot- never actually understood how the representation crap works...

        A ``group'' is a set of elements which can be ``multiplied'' according to some rule,
            and which satisfies the four conditions of:
                \begin{itemize}
                    \item Closure - the product of any two elements of the group are still in that group;
                    \item Associativity - $(a \times b)\times c = a\times(b \times c)$;
                    \item Identity - there is some element in the group $I$ for which $I \times a=a$;
                    \item and Inversion - every element $a$ has an inverse $a^{-1}$ such that if $b \times a = c$ then $c \times a^{-1} = b$.
                \end{itemize}

        If the group operation is commutative ($ab=ba$) then the group is ``Abelian''; if not, it is ``non-Abelian''.

        Ok now for the representation stuff. [Sredneki draft pg411]

        \cite{Cheng_book}

    \subsection{Restricting Motion: The Lagrangian, Symmetry, and Noether's Theorem}

       Explain how motion is described by a lagrangian of fields. 
       Minization of Action.
       Equations of motion derived from Euler-Lagrange Equations.
       The lagrangian takes the form it does in order to satisfy poincare symmetry.
       Discuss Noether's Theorem; refer to Halzen pg 314 (djvu=331) or, maybe better, Sredneki pg 144).

       I also should maybe end this with a discussion of how forces aren't really a thing
            and particles in field theory exchange momentum by merely bumping into other particles.
       Maybe, *maybe**** (maybe not) give a toy example lagrangian showing a particle which can interact with itself via a three-point vertex or something.
       Note the basic symmetries that the basic lagrangian must satisfy (hence group theory going first)


    \subsection{Transferring Motion: Gauge Symmetry}

        Gauge symmetries; U1, SU(N).
        The effects of imposing gauge symmetries on the Lagrangian, and the advent of the gauge bosons and their forces.
        how do gauge bosons come out from symmetries.

        \cite{Halzen_book}


        

% This might actually need to become its own chapter...
\section{The Standard Model of Particle Physics}
    \subsection{The Higgs Mechanism}

        Basic higgs mechanism example.

        Have a field interacting with a potential V with quadratic and quartic terms.
        This produces a symmetric but unstable equilibrium.
        Perform passive translation to recenter coordinates around stable equilibrium point/minimum of potential.
        In super-simple example, this gives an originally massless particle mass.

        Ok so let's start with a really basic lagrangian describing only the kinetic energy of a massless scalar particle $\phi$, of the form:

        \begin{equation}
            \Lag = K_{\phi} = \frac{1}{2} (\partial_{\mu} \phi)^2
        \end{equation}

        Let's introduce a quartic potential $V(\phi) = -\frac{1}{2} \mu^2 \phi^2 + \frac{\lambda}{4!} \phi^4$

        \begin{equation}
            \Lag = K_{\phi} - V{\phi} = \frac{1}{2} (\partial_{\mu} \phi)^2 
                +\frac{1}{2} \mu^2 \phi^2 - \frac{\lambda}{4!} \phi^4
        \end{equation}

        Such a potential will result in a Hamiltonian which is symmetric about $\phi=0$, but that point will be a local maximum.% TODO figure for this
        The Hamiltonian will have two minima to either side of $\phi=0$, at points $\pm \nu = \pm \sqrt{\frac{6}{\lambda}} \mu$.

        A system in such a potential would invariably fall into one of these minima.
        The Lagrangian can be rewritten from the perspective of one of these minima (e.g.\ $+\nu$),
            by substituting in a shifted field $\sigma$, where $\phi(x)=\nu+\sigma(x)$.
        The Langrangian now takes the form (after simplifying)

        \begin{equation}
            \Lag = \frac{1}{2} (\partial_{\mu} \sigma)^2
                - \mu^2 \sigma^2
                -\sqrt{\frac{\lambda}{6}} \mu \sigma^3
                - \frac{\lambda}{4!} \sigma^4 \\
        \end{equation}

        \begin{equation}
            \Lag = \frac{1}{2} (\partial_{\mu} \sigma)^2
                - m_{\sigma} \sigma^2
                - k_{\sigma} \sigma^3
                - \frac{k_{2\sigma}}{4!} \sigma^4
        \end{equation}

        With the latter equation taking the form of a now massive field $\sigma$ with both a three and four point vertex,
            governed by two different coupling values $k_{\sigma}$, and $k_{2\sigma}$.

        \cite{Halzen_book}

        

    \subsection{Electro-Weak Symmetry Breaking}
        In full form, with bosons obeying U1XSU2 gauge symmetry, this produces 3 massive bosons W+, W-, Z0, and a massless photon.
            (you might have to work this out first for only the kinetic term, not for the full lagrangian)

        With the toy model done, let's now use the Higg's Mechanism to produce the massless photon and massive W and Z bosons.
        How the Higg's Mechanism gives rise to fermion masses and the Higgs Boson will be left to the next section.
        To do this, we start with the universe in a high-energy state,
            with a scalar field that is simultaneously symmetric with regards to both the $U(1)$ and $SU(2)$ gauge groups.
        This field then transforms as
        \begin{equation}
            \phi \rightarrow e^{i \alpha^a \tau^a} e^{i \beta/2 } \phi
        \end{equation}.

        Assign vev, perform gauge transform, get three massive and one massless boson (pg 722);

        Get covariant derivative, evaluate at vev, pull out W,Z, and photon fields, and their masses (pg 722);

        Rewrite covariant derivative in terms of mass eigenstates (pg 723);

        Simplify by collecting terms into the EM and Weak couplings (pg 723-724);

        Split fermions between right and left fields, assign LH to SU(2) doublets (T=+-1?), RH to SU(2) singlet (T=0), assign Y too (pg724-725);

        Sandwitch covariant derivative terms between fermion fields, expand to get field currents (pg 725-726);

        *Try* to expand masses and fail because of representation incompatibilities (pg 725);

        Anamoly cancellation thing I'll probably ignore (pg 726);

        



%        Working out the full lagrangian gets...complicated (GWS theory).
%        Weak interactions only interact with certain helicity states??
%        Coupling all this to fermions requires weird mass treatment? Resolving this requires use of the CKM matrix (pg 714).
%        Some kind of loop anomaly problem that can only be solved by a weird balance of leptons and quarks???
%
%
%    \subsection{The Higgs Boson and di-Higgs Interactions}
%        Further work must then be done to couple the higgs boson to fermions and itself
%        You need to tie k2v, kl, and kv in to the shape of the higgs potential here
%
%    \subsection{The Particle Zoo}
%        Summarize all the particle types (minus Higgs?), and their relationship to the gauge symmetries.


%
%       Either here or in appendix: how do forces arise from particle interactions (Lagrangian -> Euler-Lagrange equations -> equations of motion)
%       Also definetly need to explain and summarize the SM table of particles: fermions, bosons, quarks, leptons, etc.
%       (Do I need a discussion of fermi-dirac statistics? Bose-Einstien? If so, where should that go...)
%
%\section{From Theory to Experiment: The Feynman Rules and Cross-Sections}
%
%     How do we test any of this?
%        
%    From Lagrangian to Cross-Section:
%        I need to study up on exactly how you go from the lagrangian to the Feynman rules, and from there to a calcualable cross section
%
%
%
