\chapter{Event Reconstruction}



    \section{Introduction}
        Once a bunch-crossing event has cleared the Level 1 Trigger system, the process of event reconstruction begins.
        An event in ATLAS is, initially, nothing but a collection of electrical signals emitted from the various detectors.
        Event reconstruction is the process wherein detector readings are aggregated together into meaningful patterns,
            which are interpreted as physical objects and processes.
        The key objects of interest in this analysis are \textit{particle tracks} and \textit{jets}.
        As well, the process by which b-quark-originating jets (``b-jets'') are distinguished from other jet types,
            called \textit{flavour tagging}, plays a particularly critical role in this analysis.
        Finally, special attention will be given to the fact that reconstruction occurs not once, but twice;
            first in the High Level Trigger online running environment,
            and again in the offline environment after events have been read out.
        The ways that reconstruction differs between these two environments plays a subtle but important part
            in later stages of this analysis, and so is worth discussing further.


    %    I should have an image for this in some way %TODO
    \section{Tracks}
            
            %Intro about tracks being the first things we see because they're assembled from the inner detector or something.
            %Also I should probably explain what a track is...
        \subsection{Track Definition}
            
            The first point of contact for anything leaving the interaction region of ATLAS is the tracking detectors.
            Consequently, the objects reconstructed from the tracking detectors will be the first point of discussion.
            Prior to encountering the ATLAS calorimeters, particles exiting the IR travel along a relatively unimpeded trajectory.
            Thanks to the inclusion of the Solenoid Magnet field encompassing the Inner Detectors,
                these trajectories reveal crucial information about the particles in an event.
            This is due to the fact that any charged particle, with a momentum component orthogonal to a magnetic field,
                will trace out a \textit{helical} trajectory.
            If the shape of the helix is known, then basic electrodynamics principles can be used to determine the
                momentum and charge of the particle that formed it.
            Specifically, for a particle with four-momentum given as 
            \begin{equation}
            p = \minimatrix{E \\ p_T \\ \theta \\ \phi}
            \end{equation}
            the track's helical shape can reveal all components except for the energy $E$ (this will be handled by the calorimeters).
            To understand how this is achieved, it is worth discussing how a track helix is described mathematically.
            A helix can be described using five parameters structured in a parametric equation.
            There are several conventions for the form these parameters should take,
                but in ATLAS a helix is described by the equations\cite{thesis_giacinto}:
            \begin{equation} \begin{split}
            x(\alpha) &= x_{PV} + d_0 \cos(\phi) + \rho \left[ \cos(\alpha) - \cos(\phi) \right] \\
            y(\alpha) &= y_{PV} + d_0 \sin(\phi) + \rho \left[ \sin(\alpha) - \sin(\phi) \right] \\
            z(\alpha) &= z_{PV} + z_0 - \rho \cot(\theta) (\alpha - \phi)
            \end{split} \end{equation}

            The helix is described relative to the Primary Vertex, whose position is given by the coordinates $(x_{PV},y_{PV},z_{PV})$.
            The five parameters governing the shape of this helix, called \textit{perigee paramters}, 
                are described largely in terms of the ``Point of Closest Approach'' (PoCA),
                the point on the helix closest to the PV in the $x,y$ plane.
            \begin{itemize}
                \item $d_0$: $|(x,y)|$-Distance from the PV to the closest point on the helix, in the $x,y$ plane
                \item $z_0$: $z$-Distance between the PoCA and the PV
                \item $\phi$: Angle in the $x,y$ plane of the PoCA
                \item $\theta$: The angle of the helix's trajectory in the $\rho,z$ plane
                \item $\rho$: The radius of curvature of the helix
            \end{itemize}

            \begin{figure}
                \subfloat[Perigee Parameters]{
                    \includegraphics[width=0.5\linewidth,height=\textheight,keepaspectratio]{reconstruction/perigee_base}
                }
                \subfloat[Front View, $(x,y)$-Plane]{
                    \includegraphics[width=0.5\linewidth,height=\textheight,keepaspectratio]{reconstruction/perigee_front}
                }\\
                \subfloat[Top View, $(x,z)$-Plane]{
                    \includegraphics[width=0.5\linewidth,height=\textheight,keepaspectratio]{reconstruction/perigee_top}
                }
                \caption{
                    Perigee parameters are hard!
                }
                \label{fig:perigee_params}
            \end{figure}
            

            Both $d_0$ and $z_0$ will be used in later sections for the purpose of flavour tagging.
            Returning to the four-momentum, $\phi$ and $\theta$ are themselves components of the particle's 4-vector.
            The final parameter, $\rho$ can then be related to the particle's tranverse momentum and electric charge as
            \begin{equation}
            r = \frac{p_T}{qB} = \frac{\vec{p} \sin(\theta)}{qB}
            \end{equation}
            where $B$ is the magnetic field of the Inner Detector Solenoid Magnet.
            \cite{thesis_track_sim_and_reco}

        \subsection{Track Reconstruction}

            As particles pass through the inner detector subsystems, they trace out a path of ionized detector elements.
            The trajectory can be reconstructed by playing connect the dots.

            %Clusterization
            The first step to producing a track is the process of clusterization.
            Ionizing particles often deposit energy across several adjacent pixels on a given layer.
            A \textit{connected component analysis} algorithm is used to group pixels together. 
            Based on the pattern of energy distribution in these groups,
                a \textit{space-point} is created indicating the estimated position at which a particle crossed the detector.
            Several space-points can be assigned to the same pixel cluster,
                if the energy deposition pattern suggests multiple particles traversed the same location.

            %Combinatorial track finding
            Initial guesses at tracks, called track seeds, are formed by assembling all realistic combinations of three space-points.
            The track seeds are assigned helix parameters by assuming they travel through a uniform magnetic field,
                allowing immediate estimates of the tracks' momentum.
            These seeds are expanded into \textit{track candidates},
                by including more space-points across additional detectors in the ID using a \textit{Kalman filter}.

            %Ambiguity solving; NN clustering; Track fit
            A number of criteria are then used to reject poor-quality tracks, as well as to assign scores to all remaining track candidates.
            The scores are then used to resolve ambiguities where multiple tracks are assigned to the same space-points,
                with preference given to higher-scoring candidates.
            Neural networks are used to assist in some ambiguity solving situations,
                as well as to help identify clusters with multiple valid tracks.
            Once ambiguities have been resolved and all malformed track candidates removed,
                the remaining tracks are refit using all available information at high-resolution.
            \cite{atlas_track_reco_performance}

    \section{Jets}
        Jets are an algorithm.
        Specifically, the anti-$k_t$ algorithm, with $\Delta R = 0.4$.
        Something about topological-clusters (topo-clusters).
        \cite{anti_kt}
        Something about assuming hard-scatter jets come from the primary vertex.
        Also explain what a primary vertex is.
        FYI, a primary vertex is a "reconstructed vertex with at least two associated tracks, and the largest sum of squared track momentum".

        Jets created using only calorimeter-based topological-clusters (at the electromagnetic scale [what does this mean??]) %TODO?
            are reffered to as \textit{EMtopo Jets}.
        The jets for this analysis use a more advanced jet algorithm, called \textit{Particle Flow},
            which are constructed by matching tracks from the inner detector to topo-clusters from the calorimeter,
            based on both location and energy projections.
        The resulting \textit{PFlow Jets} are then used for the rest of the reconstruction process.
        \cite{pflow}
        \cite{jet_energy_scale13TeV}

    \section{Flavour Tagging}
        Of crucial importance:
            b-jets used for higgs,
                (I should justify this with the high branching ratio of H->b,bbar;
                you put this in your thesis commitee talk, use it);
            VBF initial scatter jets anti-b-tagged

        Flavour Tagging is performed in two-stages:
            first with basic low-level taggers,
            which are then used as inputs for the high level taggers.

        \subsection{Low Level Taggers}

            IPxD - IP2D and IP3D - Impact Parameter 2/3 Dimensional; 

            Vertex Algos - SV1 (Secondary Vertex 1) and JetFitter
            \cite{thesis_giacinto}

        \subsection{High Level Taggers}

            MV2, a Boosted Decision Tree (BDT);
            DL1, a Deep Learning Neural Network (DL1r is what we specifically use)
            \cite{bjet_id_and_performance}
            \cite{btagging_optimisation}

    \section{Online VS Offline Reconstruction}
        I should explain how online and offline reco differs;
            it seems to just come down to the fact that the HLT doesn't neccesarily use the most high resolution calorimeter information,
            and often generates tracks using a faster, less acurate algorithm.

        Reconstruction is performed for an event twice.
        It is first done rapidly, using coarse measurments and calculations,
            for the purpose of allowing the aforementioned HLT to trigger events for readout.
        These events which pass the trigger are passed off to a massive global computer cluster reffered to as the GRID.
        No longer bound by the stringent time contrainsts of the ``online'' running environment,
            the GRID is able to devote vast amounts of time and computing power to a second, 
            more comprehensive, reconstruction of each event.
        For both the ``online'' (HLT) and ``offline'' (GRID) environments, event reconstruction is carried out by the same suite of software,
            called \textit{Athena}.
            
        I should also describe the mv2 reweighting procedure here
            (We don't use mv2 for the late-stage b-tag requirement on the higgs decay candidates,
            but we DO use it in the triggers):
            Step1.1 - Run HLT reco on MC files;
            Step1.2 - Return low level taggers on MC HLT reco;
            Step2   - Rerun HLT reco on MC, now with retuned low level tagging info;
            Step3.1 - Perform BDT training for MV2 on HLT MC reco w/ low level tags;
            Step3.2 - Convert training output to useable histograms;
            Step4   - Rerun HLT reco on MC once more, to get full retuned tagging ouput. Use output performance to determine working points;


    %\section{Scale Factors} %TODO
        %Ugghhh... I wonder if this should go up into the trigger chapter? Might make more sense
