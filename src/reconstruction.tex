\chapter{Event Reconstruction and Selection}

    Oh god how do I even start with this crap. These two are so tightly linked and there's no obvious way to speparate them AHHHHHH.
    
%    Once a bunch crossing event has cleared the Level 1 Trigger system, the process of event reconstruction begins.
%    An event in ATLAS is, initially, nothing but a collection of electrical signals emitted from the various detectors.
%    Event reconstruction is the process wherein detector readings are aggregated together into meaningful patterns,
%        which are interpreted as physical objects and processes.
%


\section{Object Classification}

    Ok let's just get the basic objects contructed first, since nothing else (triggers, later reconstruction, final selection)
        make sense until I have them.

    This entails ... what? For VBF 4b I think we have:

    Tracks: Helix fitting, I don't think I really know how this works but that's fine I'll figure it out...

    Primary Vertex

    Jets:
        what is a jet.

    Jet flavour tagging.
        b-jets used for higgs,
        light jets used for VBF IS quarks.
    
    ... is that it? I'll look around to see if there's anything else.

    I should explain how online and offline reco differs.
    \cite{cell_clustering}
    \cite{bjet_id_and_performance}
    


    %Reconstruction is performed for an event twice.
    %It is first done rapidly, using coarse measurments and calculations,
    %    for the purpose of allowing the aforementioned HLT to trigger events for readout.
    %These events which pass the trigger are passed off to a massive global computer cluster reffered to as the GRID.
    %No longer bound by the stringent time contrainsts of the ``online'' running environment,
    %    the GRID is able to devote vast amounts of time and computing power to a second, 
    %    more comprehensive, reconstruction of each event.
    %For both the ``online'' (HLT) and ``offline'' (GRID) environments, event reconstruction is carried out by the same suite of software,
    %    called \textit{Athena}.
    





\section{HH4b VBF Analysis}
    
    I think I'm just going to do everything else here? It's weird, because I'm going backwards to triggers,
        and then forward to the final analysis stuff. But at least I have all the complicated object algorithms and stuff listed I guess?

    %Truthfully, reconstruction and selection are not distinct stages of the analysis, but rather are interwoven with each other.
    %The first step of selection really occured before any signficant reconstruction had even taken place, in the L1 Trigger system.
    %Yet the final stage of event reconstruction -- the reconstruction of the di-Higgs system --
    %    will be performed \textit{after} almost every other stage of selection.

    Triggers used in this analysis and why (do we have any plots showing why we use these triggers?)

    Trigger bucketing strategy?

    I need to discuss the kinematics of VBF and 4b in order to justify the ->

    kinematic cuts
        basic jet multiplicity reqs,
        4b-tagged, central
        2 non-btagged jets with min-mjj for VBF

    Then there's the minDR pairing of b's to reconstruct HH

    Then cut on dihiggs and full system kinematics

    Then make sure the higgss fall within the "signal region" (their individual masses are approximately 125 GeV)

    Pretty sure that's it?



%...and how we wittle the abundance of events down to a manageable subset.
%If I stick to this format, there's still a bit of event reconstruction done here.
%
%HH4b Resolved Reco and Selection (literally just go through resolved recon...).
%    Final stage is the signal region selection
