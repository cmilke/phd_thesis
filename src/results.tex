\chapter{Results} \label{chapter:results}

%Need to discuss how we set put data, the background estimate, and the signal model together
%    and make a claim as to the compatibility of the hypothesis with the data.
%Largely, this means I need to finally understand how pyhf actually works and what the hell the limit framework is doing.
%
%
%
%    I'm not sure I actually have to explain Baye's theorem here...
%    it doesn't appear to be obviously used, or if it is, we're implicitly using the "Uniform" Prior
%    Ask Steve
%
%    L gives probability of seeing the data we have, based on the given model.
%    We need the probability that the given model is the one responsible for the data we see.
%    i.e. we have P(data|model), but we need P(model|data)
%
%    To do this, we need Baye's rule, which comes from the basic 'anding' of probabilities:
%    P(a \& b) = P(a)*P(b|a) , or P(a \& b) = P(b)*P(a|b) . Thus
%    P(a)*P(b|a) = P(b)*P(a|b) 
%    P(b|a) = P(a|b) P(b) / P(a)
%    So we need P(model|data) = P(data|model) * P(model) / P(data)
%
%    For our data we have to use the extended L, which accounts for poisson stats.
%    Basic test is to test mu*S+B for what value of mu is compatible with data
%
%        
%
%

%\section{Statistical Mathematics}
%
%    Statistics is a powerful tool in science,
%        but one which can be very misleading if not used carefully.
%    The oft-used quote comparing lies and statistics exists for a reason;
%        both can lead to incorrect scientific conclusions.
%    But whereas a lie can be caught by a simple slip of the tongue,
%        it can take scientists years to discover a slight (intentional or not) mishandling of statistics.
%    Unfortunately, the use of statistical methods is not optional.
%    In this section, I want to explain why statistics are required for this analysis,
%        as well as describe the basic mathematics and techniques utilized to obtain results.
%
%    To begin, let me propose a far simpler experiment than the one described in this analysis.
%    I have a coin, which I suspect may be weighted to one side.
%    How can I test this?
%    basic binomial distribution (coin flip) allows obvious p-test. 
%    Show of whether or not theory is compatible with data.
%    ?? Can show how to set basic limits in absence of enough data


What is a test statistic and why do we need it?
%    I have a single 6-sided die.
%    I suspect that this die is not a fair die,
%        and that it is actually weighted to land with the ``4'' side facing up more often than other sides.
%    How can I test this?

Description of how "test statistic" q~ is constructed:

    Likelihood function

        L = product[ for each category:
            product[for each bin: poissons]
            * product[ nuissance params ] 
        ]

        
    lambda = L(mu, theta vary-opt) / L(mu-opt, theta-opt);
    t = -2*ln(lambda);
    q~ = t with edge cases;

%\section{VBF \to HH \to 4b Limit-Setting Framework}
%
%    Discuss sources of error, assumptions, categorization, concept of mu values;
%    basically all the complicated stuff the limit framework is doing
%
Discussion of how the error is calculated for the shape is a particular point here...

Discuss how this formula for L specifically functions in this analysis
    (emphasis on explaining the bits in parentheses):

        L = product[ for each category (2: eta hi and lo) 
            product[for each bin: poissons]
            * product[ nuissance params (4 for bgd shape error, 1 for norm error) ] 
        ]

    
%\section{Final Interpretation}
%
%    Finally, I need to reveal what our final results actually are, and what they mean.
%    This will mostly just be plots of mu values, limit values, and my 2D exclusion plots.
%    I can discuss the shapes and stuff here, as well as talk about any mismatches between expected and observed results.
%    Be ready to just put in filler results, since we probably won't have unblinded by the time I get here.
%

Single mu vs q~ PDF constructed from monte-carlo distros

C-PDF showing where mu-model resides for Psb, Pb, and finally Ps=Psb/(1-Pb)
CLs = CLs+b/CLb explained in \cite{Barlow:2019svl} (pg. 192)

In practice, the q~ CPDF distros are estimated using an asymptotic approximation method\cite{asymptotic_formulae_for_likelihood}.
(because the MC method is way too slow)

SM mu scan plot of signal p-value.
If you do this using one of the other stats (t, or t~ or something),
    I think you can use it to show that pure discovery of the HH process is impossible right now,
    and therefore justify putting limits on the couplings instead.
Should also probably dig up the "sensitivity" metric and show how bad that is here as well.

multi k2v-scan mu plots

1D SM k2v plot

Multi-kl 1D k2v plots

2D k2v/kl plot

Multi-dimensional slice plots
