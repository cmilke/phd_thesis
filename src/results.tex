\chapter{Results} \label{ch:results}

%Need to discuss how we set put data, the background estimate, and the signal model together
%    and make a claim as to the compatibility of the hypothesis with the data.
%Largely, this means I need to finally understand how pyhf actually works and what the hell the limit framework is doing.
\section{Statistical Mathematics}

    Statistics is a powerful tool in science,
        but one which can be very misleading if not used carefully.
    The oft-used quote comparing lies and statistics exists for a reason;
        both can lead to incorrect scientific conclusions.
    But where a lier can be caught by a simple slip of the tongue,
        it can take a group of scientists years to discover a slight (intentional or not) mishandling of statistics.
    Unfortunately, the use of statistical methods is not optional.
    In this section, I want to explain why statistics are required for this analysis,
        as well as describe the basic mathematics and techniques utilized to obtain results.

    To begin, let me propose a far simpler experiment than the one described in this analysis.
    I have a single 6-sided die.
    I suspect that this die is not a fair die,
        and that it is actually weighted to land with the ``4'' side facing up more often than other sides.
    How can I test this?
        


\section{VBF \to HH \to 4b Limit-Setting Framework}

    Discuss sources of error, assumptions, categorization, concept of mu values;
    basically all the complicated stuff the limit framework is doing

\section{Final Interpretation}

    Finally, I need to reveal what our final results actually are, and what they mean.
    This will mostly just be plots of mu values, limit values, and my 2D exclusion plots.
    I can discuss the shapes and stuff here, as well as talk about any mismatches between expected and observed results.
    Be ready to just put in filler results, since we probably won't have unblinded by the time I get here.


