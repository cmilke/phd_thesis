% The general structure I've seen everybody else using is to essentially describe ATLAS one subdetector at a time, I think from the innermost tracker to the outermost calorimeter
% Also need to discuss the trigger system

\chapter{ATLAS}
    ATLAS is like a giant camera...

    One of the two general purpose detectors on the LHC

    Appendix on Helical Coordinates and Helix Parameters?

    Discussion of radiation hardness?

    Then I'll give an overview of the rest of this:
        Enter ATLAS beampipe,
        focussed by Toroid Magnets,
        Collision in Interaction Region,
        Then propogation through the detector elements listed below



\section{Interaction Region}
    I might split this up slightly... For now I want to discuss:
        the beampipe focusing magnets,
        the beampipe?,

    ~1000 particles every 25 ns w/in |eta| < 2.5.



\section{Inner Detector}
    Inner Detector:
        Uses 2T Magnetic field from Solenoid Magnet to bend tracks (for identifying mass and charge)
        5.3 m long x 2.5 m diameter
        trackers cover region of |eta| < 2.5


    \subsection{Pixel Detector}
        Pixel Detector:


    \subsection{Semiconductor Tracker (SCT)}
        How does this differ from the pixel detector? (pretty sure it's litterally meant to do the same thing, but it's bigger/further from IR so everything is made from cheaper parts)


    \subsection{Transition Radiation Tracker (TRT)}
        I might need to read more on this... it seems to do a lot and I'm unfamiliar with this kind of detector.
        It's soooort of like a another (cheaper) tracker,
        but it's also used to distinguish some types of particles based on their "transition radiation",
        i.e. how they emit photon radiation as the material the pass through changes.



\section{Calorimetry}
    LAr Electromagnetic Calorimeters,
    LAr Hadronic Calorimeters,
    Tile Calorimeters,


    \subsection{LAr Electromagnetic Calorimeter}
        Higher resolution
        For electrons and photons


    \subsection{LAr Hadronic Calorimeter}
        It makes hadrons sad


    \subsection{Tile Calorimeter}
        It's made of tiles I guess


% TODO: Let's try filling out the detectors one by one, starting at the end and working my way inward
% That will hopefully make it more clear to me what makes the inner/outer detectors special
\section{Muon Chamber} 
    The thing at the end that checks for muons...
    


\section{Readout Electronics}
    Not sure if this needs its own section or should be combined with the next...

\section{Trigger System}
    You've worked on this damned thing you better be able to talk about it
