% Structure
%    intro
%    purpose
%    (maybe) helix coords (appendix?)
%    general barrel/endcap cylindrical structure
%    walkthrough of the systems, from inner to outer, discussing why they are there and their basic purpose
%    Then split off into sections for the individual subsystems


    %Discussion of radiation hardness?
    %Things in the endcap suffer from more radiation exposure than things in the barrel
    %    (you should be able to show this from the basic kinematics of the particle beams. most energy is deposited in parallel to the beams, not orthogonal to them)
    %things closer to the IR suffer more than things further away (literally just the inverse-square law)

    %The method by which these detectors work presents a logistical challenge as to their placement.
    %Specifically, the calorimeters measure particle energy through a purely destructive process.
    %Current technology to measure the energy of fast-moving particles is done through a process that stops a particle in its path, and often forces it to decay into a vast number of lower-energy particles, through a process known as "showering".
    %Once this energy measurement is complete, the measured particle has been either absorbed or completely destroyed, making further measurement of its properties impossible.
    %As such, measurement of a particle's momentum and charge must be done prior to the energy measurment. This requirement determines the location of the different sub-detectors.
    %The tracking detectors are placed the closest to the interaction region. The tracking detecor barrels are closest in $r$,
    %and the tracking endcaps closest in $z$.

        
    %TODO things I don't really know...
    %Why is the muon chamber last? Why does it get its own special chamber? does it detect other things?
    %    Because punch through all the other detectors without stopping, due to [look up the reason in that energy source]. Thus, it goes last so that muons can be id'd after everything else HAS been stopped.
    %Is their a preference in detector placement? angular resolution gets better as you move further out; does this factor into anything? is this why further out things can be designed using lower resolution tech?
    %is there any reason to prefer things closer to the IR
    %    yes, this allows detection of short lived particles like b-quarks and taus \cite{CERN-LHCC-97-016}
    %Why is the only way to measure energy destructive? is there a way to measure energy non-destructively? What does the muon chamber do? does it destroy things? does it not measure energy? If so, how?
    %Can we measure neutral particles in trackers? how
    %    we can, but only photons because they interact through EM. Things like neutrons and pi0 go through it largely undetected until the hcals
    %Do we measure particle mass with trackers (via mass/charge ratio), or just charge? Do we just get mass from energy momentum calculation?


\chapter{ATLAS}
    % Intro
    Production of new physics and particles is of little use without the ability to observe said physics.
    Herein lies the purpose of ATLAS.
    One of the two general purpose detectors at the LHC, construction of the ATLAS detector was completed on October 4, 2008.
    ATLAS is among the largest particle detectors ever built, measuring 46m long with a 25m diameter, and weighing in at 7,000 tonnes \cite{atlas_website}.

    % Purpose
    Core purpose is to accurately record the physical properties of the particle interactions which take place in the interaction region of the detector.
    Most of the particles of interest are extremely short-lived, and so their properties cannot be measured directly.
    Instead, we must detect the decay products of these particles, and reconstruct the original particles of interest after the fact.
    Accurate reconstruction of these original particles is critically dependant on measuring, as precisely as possible, the physical properties of the decay products.
    More specifically, ATLAS is designed to record the paths and decay showers of the particles which pass through the detector, in order to determine their mass, energy, momentum, and electric charge.

    % General barrel/endcap cylindrical structure
    In order to measure all the required properties, ATLAS is divided into many different subsystems.
    Each of these subsystems has a very different design and objective, but they are all constructed with roughly the same overall cylindrical geomtry.
    The reason for this design is simple kinematics.
    The LHC particle beams cross with no initial transverse momentum, which means particles are ejected without preference in the radial angle $\phi$.
    Furthermore, the extremely high longitudinal momentum of the beams results in many particles continuing along a highly "forward" (parallel to the beampipe) trajectory.
    These two properties lead naturally to a radially symmetric detector which is elongated in the forward direction; a cylinder centered on the beam axis.
    To accomodate this geometry, the various sub-detectors of ATLAS are generally split into two distinct parts, called "barrels" and "endcaps".
    The barrels are a series of radially symmetric cylindrical shells, concentric about the beampipes, meant to detect particles moving primarily in the transverse direction.
    Conversely, the endcaps are a series of flat, circular plates, stacked one behind the next along the beampipes, intended to detect more forward particles.

    % Walkthrough of the systems, from inner to outer, discussing why they are there and their basic purpose
    The various detector subsystems can be broken up into three primary groups, based on the subsystems' purpose.
    Moving out from the interaction region, the subsystems can be classified as belonging to the Inner Detector, the Calorimeters, or the Muon Spectrometer.
    The first of these, located as close in $r$ and $z$ as possible, is the Inner Detector system.
    The Inner Detector is desigined to provide momentum measurement, vertexing, and electron identification.
    It must be located so close to the interaction region in order to permit detection of short lived particles like b-quarks and taus \cite{CERN-LHCC-97-016}.
    Following immediately behind (endcap) and around (barrel) the Inner Detector is the collection of sensors comprising the Calorimetry system.
    These sensors are purpose-built to measure the energy of incoming particles and additionally provide supplementary tracking information for particle trajectories
    Their location between the IR and Muon System is based on their tertiary purpose, which is to shield the Muon system from escaping hadrons.
    Finally, at the outer edge of ATLAS, is the Muon Spectrometer, which has been built to measure the momentum of muons leaving the detector volume.
    %TODO insert overall picture of ATLAS here for reference
    


% Then split off into sections for the individual subsystems


\section{Inner Detector} %TODO
    % Purpose of subsystem
    % Basic specs
    % What mechanism is used to achieve this purpose
    % What are the individual detectors and how do they contribute to this goal
    
    The Inner Detector system is intended to provide measurement of particles' momentum, provide vertex information, and help in identifying electrons.
    The way it achieves these goals is primarily through a series of very high resolution tracking sensors, which are used to trace out the paths that particles traverse as they leave the IR. 
    Momentum and charge measurement is facilitated by using a solenoid magnet to encompass the entire Inner Detector with a 2T axial magnetic field.
    This field bends the high-momentum charged particles into a helical trajectory, allowing their momentum, mass, and charge to be determined from the shape of the particle's path.
    The entire ID system, consisting of three independent detectors, measures 5.3 m in length and 2.5 m in diameter, and is able to provide accurate tracking within $|\eta| < 2.5$ \cite{id_tdr}.
    From the innermost to outermost, the sub-detectors are the Pixel Detector, the Semiconductor Tracker (SCT), and the Transition Radiation Tracker (TRT).
    The Pixel Detector and SCT together are responsible for high resolution tracking of particle trajectories.
    Further from the IR, the TRT provides more particle tracking capability, as well the ability to help distinguish electrons. 


    \subsection{Pixel Detector and Semiconductor Tracker}
        % purpose
        Purpose is to provide high resolution position and momentum information about particles as they leave the IP,
        They ensure that any particle exiting the IR crosses at least seven detector layers (3 pixel, 4 strip), while having minimal effect on the particle trajectory and energy.


        %what kinds of particles will they detect,
        They detect any kind of ionizing radiation, which is either photons or any particle with electric charge.

        % what kind of detector tech are they,
        Both the pixel detector and semiconductor tracker are semiconductor diode-based detectors.

        % how does this tech work? what happens as particles pass through them, and how do we convert that response into something useful
        Semiconductor diode detectors function by exploiting the properties of semiconductor p-n junctions.
        These particular detectors are made using silicon.
        Silicon has four valence electrons, so a pure silicon crystal lattice will have its valence band perfectly filled, leading to a very stable structure.
        A pure semiconductor crystal lattice (in this case, silicon) can have impurities intentionally introduced to it through the process of doping.
        Doping the lattice with an element possesing only three valence electrons (e.g. Boron) will result in a number of gaps in the valence band (called "holes").
        In such a situation, known as p-type doping, the lattice will accept additional electrons to fill these holes, which will lead to an excess of negatively charged ions.
        Conversly, an element with five valence electrons can be introduced for doping, leading to an excess of electrons in the valence band.
        Known as n-type doping, such an excess results in a lattice with a propensity for shedding these excess valence electrons, which in turn leads and an excess of positive ions.
        A p-n junction can be produced by taking a single silicon wafer and n-type doping one half, while p-type doping the other.
        The junction where the two dopings meet will then see a transfer of excess valence electrons moving from the n-type side to fill the holes of the p-type side, as illustrated in figure %TODO include an illustration of this.
        As the excess "donor" electrons migrate to fill the "acceptor" holes, the area around the junction has its valence band perfectly filled, creating an area called the "depletion zone".
        Though the depletion zone has a filled valence band, it has done so at the cost of ionization; an excess of electrons now populates the p-type side, with an equal number of positive ions remaining on the n-type side.
        The depletion zone grows larger until the migration of holes and electrons is balanced by the electric potential created through this ionization.
        When equilibrium is achieved, the lattice is left with an electric potential which monotonically decreases from the n-type to the p-type side, and which spans the full width of the depletion zone. %TODO illustration of this too?
        If a voltage is the applide across the semiconductor, the width and potential difference of the depletion region can be altered.
        If the voltage is applied with the positive end of the difference at the p-type side, then the semiconductor is said to be "forward biased", and the depletion region will become smaller (and with a high enough voltage can be eliminated entirely). \cite{wiley_radiation_detection}
        If the positive end of the voltage difference is applied to the n-type side though, the semiconductor becomes "reverse biased", and the depletion region and potential difference across the junction will grow larger.
        In this reverse-biased state, the electric potential of the p-n junction becomes very effective at rapidly sweeping excess ions from the depletion region off to the edges of the semiconductor wafer, and it is this mechanism which the ATLAS semiconductor detectors exploit in order to detect particles.

        When ionizing radiation passes through an element of the Pixel Detector or SCT, it will momentarily separate electrons from their nuclei in the silicon lattice.
        Normally, such separated ions would just recombine in a matter of moments.
        Because of the electric potential in the depletion region though, these ions are further seperated, and swiftly arrive at opposite ends of the semiconductor wafer, moving at speeds of about something m/s.%TODO look this speed up in the TDR
        The very leads responsible for biasing the semiconductor are then responsible for collecting these seperated ions, which will cause a sudden jump in the circuit's current.
        The current through these semiconductor detectors is closely monitored, and these spikes are used to identify the passage of a particle through the semiconductors.


        % if there are multiple kinds of similar detector (e.g. pixel vs strip), how do these differ and (ideally) why are both in use?
        % why is the detector where it is
        % have a table describing position of layers, size of components, resolution achieved and what we actually see based on this


    \subsection{Transition Radiation Tracker (TRT)} %TODO
        % purpose
            To provide continuous tracking and aid in the identification of electrons via transition radiation \cite{ID_DTR}.
        % what kinds of particles will they detect,
            Any ionizing radiation
        % what kind of detector tech are they,
            Polyimide drift tube straws using tungsten anode wires, filled with Xe and CO2, interleaved with polypropylene fibres as the transition radiation material \cite{atlas_tdr}.

            The TRT consists of a large number of proportional drift tubes, often referred to as ``straws".
            Proportional drift tubes function in a similar way to semiconductor diode detectors, but using a gas instead of a doped semiconductor.
            The primary component of a drift tube is a cylinder filled with a gas mixture.
            In the TRT, these cylinders are 4 mm in diameter, filled with a mixture of 70\% Xenon, 27\%CO\textsuperscript{2}, and 3\% O\textsuperscript{2}.
            Ions are produced in this mixture when ionizing radiation passes through it.
            As in the semiconductor diode detectors, these ions are collected by applying an electric field through the ionization medium in order to sweep ions away into external circutry for readout.
            In a drift tube this is accomplished by maintaining an electric potential between the tube wall and a conductive wire running through the cylinder axis \cite{drift_chambers}.  %NOTE: do I need to mention avalanche multiplication?
            The wall of the tube acts as the positively charged anode to collect electrons produced by the ionization, while the axial wire is kept at high electrical potential to act as the cathode.
            In the TRT, the cathode wall is made of aluminium and the 30 $\mu$m diameter anode wire of gold-plated tungsten. \cite{trt_design}

            For the TRT barrel, there are 52,544 straws 144 cm in length, while the endcaps each contain 122,880 straws 37 cm in length.
            These straws are arranged parallel to the beampipe in the barrel, and orthogonal to the beampipe in the endcap. %TODO: maybe use atlas_tdr fig 4.2 and 4.3 as illustrations
            This arrangment is to maximize the number of straws traversed by an outgoing particle, typically 35-40 straws for $0 < |\eta| < 2$.
            Though each individual hit has a relatively low spatial resolution compared to the semiconductor detectors, the large number of hits compensates for this with a reduced statistical uncertainty.

            In addition to providing additional tracking information, the TRT provides a secondary purpose of aiding in the identification of electrons by means of detecting transition radiation. 
            Transition radiation is a phenomenon in which radiation is emitted when a charged particle crosses a boundary between two materials \cite{transition_radiation}.
            The TRT uses polypropylene as its transition radiation material, which is interleaved between the layers of straws.
            In the barrel, there are 73 layers of straws interwoven with polypropylene fibres, and in the endcaps there are 160 layers of straws with polypropylene foil between them.
            The large number of layers provide ample and repeated oppurtunity for charged particles crossing the TRT to encounter transition layers and emit identifying radiation.

        % why is the detector where it is
        % have a table describing position of layers, size of components, resolution achieved and what we actually see based on this


\section{Calorimetry} %TODO also include table 1.3 of atlas_tdr
    You need to discuss how a sampling calorimeter works here.
    you should also explain why a cal is important. Either here or above in the intro; not sure which I want yet

    ecal and hcal differ by the far larger amount of material needed to stop hadronically interacting matter.
    Neutrons for instance can only be stopped by nuclear interactions, requiring significant amounts of high-Z material to be contained. \cite{energy_measurement}
    Photons and electrons meanwhile can be readily stopped in a much smaller amount of material oweing to the electromagnetic interaction.
    atlas_tdr figure 5.2 beautifully illustrates just how much more dense/thick the hcals are than the ecals



    \subsection{Electromagnetic Calorimeters}
        LAr Cal with accordian geometry
        Ecal uses lead for inactive, Lar for active.
        Collects charge via capacitive coupling to electrodes placed outside the active medium.
        Also uses a "presampler", due to an excess of material in front of the ecal. Presampler needed to account for energy lost in this material


    \subsection{Hadronic Calorimeter}
        Tile Cal for barrel, Lar for Endcap.
        Tile cal uses steel plates as absorber and scintillating tiles as active material.
        Tile cal used because this region (r>2.2 m) has lower radiation levels, so cheaper materials can be used \cite{Lar_cal_tdr}.
        Now I need to explain what a scintillator does...
        Lar uses copper for inactive, Lar for active.

        Needs to be able to fully contain hadrons, so they do not penetrate into the muon system and contaminate measurements there.


    \subsection{Forward Calorimeter}
        Divided into three parts: 1 ECal, and 2 HCals
        All LAr-based, but ECal uses copper while Hcals use Tungsten
        ?? something requires it be pushed back and compressed, necessitating such high-Z materials
        Intended to extend calorimetry coverage as far as $|\eta| < 4.9$.



\section{Muon Spectrometer}  %TODO
    The Muon Spectrometer has the purpose of providing track position and momentum measurements for particles (mosty muons) exiting the ATLAS detector.The Muon barrels start at a radii of 5 m from the beam axis, extending out to 10 m. The endcaps start at a $|z|$ of roughly 7.4 m, and proceed to an extent of 21.5 m. The immense size of the muon system poses a challenge, as it must provide tracking across the entire volume. It's distance from the interaction point ameliorates this issue though, as it permits the Muon detectors to operate at much lower spatial resultions than the inner detectors, while still retaining similar anngular resolution.

    \subsection{Toroid magnets}
        Three large air-core toroids meant to deflect muons.
        Barrel provides 1.5-5.5 Tm (Tesla-meters???) of bending power in $0<|\eta|<1.4$,
        and endcaps provide 1-7.5 Tm in $1.6<|\eta|<2.7$.
        The power is lower in the region where the fields overlap ($1.4<|\eta|<1.6$)

    \subsection{Precision Track Chambers}
        The Precision Track Chambers are designed to provide high resolution measurements of track position and momentum for particles escaping the ATLAS detector. Split across two technologies, the Monitored Drift Tube Chambers (MDT's) and Cathode-Strip Chambers (CSC's).
        The MDT's are drift chambers filled with Ar/CO2 gas using a tungsten-rhenium wire for charge collection. These are used in both the barrel and endcap regions, covering the range $|\eta| < 2.7$. The MDT consists of three endcap and three barrel layers, with a notable exception in the endcap region $2 < |\eta| < 2.7$. Within this $\eta$ range, for the first layer, the muon track density exceeds the resolution capabilities of the MDT's. As such, the first layer of the endcap in this range is replaced with CSC's. The CSC's are muliwire chambers using orthogonal cathode plane strips. These chambers can read both coordinates of a track simultaneosly, preventing the ghosting issues that the MDT would suffer in this higher-flux region.

    \subsection{Trigger Chambers}
        Meant to provide rapid information on muon track multiplicity and energy range.
        Also provides additional track information for higher level triggers.
        Provides acceptance in range $|\eta| < 2.4$.
        The barrel and end-cap use two different technologies in order to address the different issues present in their respective regions.
        The barrel uses Resistive Plate Chambers (RPC's), which have higher temporal resolution. The endcap consists of Thin Gap Chambers (TGC's), designed to withstand higher radiation levels while still delivering high enough time resolution to tag beam-crossings.




\section{Readout Electronics} %TODO
    Not sure if this needs its own section or should be combined with something else...
