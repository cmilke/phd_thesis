% Draft stages:
% BRAINSTORM - vague, unordered ideas of what and where things should go
% OUTLINE - Basic layout and section design
% DRAFT ALPHA - Information for all sections in a vaguely structured format
% DRAFT BETA - All sections revised into actual sentences
% DRAFT 0 - Chapter has all information with proper sentence structure and section transitions, along with needed figures
% DRAFT # - Anything after revising the #th series of comments from Steve
% DRAFT OMEGA 0 - The draft at which Steve decides things are ready for the committee
% DRAFT OMEGA # - Anything after revising the #th series of comments from the committee/higher bureaucracy

%NOTE: according to Steve, you can (and SHOULD) use "I" in the theory chapter. Indeed, use "I" as well literally anywhere I specifically was responsible for something. Do NOT use royal "we". Only use "we" in the context of what "we" the HDBS 4b or dihiggs group have done as a group, and only if you have explicitly defined who "we" are.

\documentclass[12pt]{report}

\makeatletter
\setlength{\@fptop}{0pt}
\setlength{\@fpbot}{0pt plus 1fil}
\makeatother

\usepackage{Dedman-Thesis-Latex-Template/sty/DL_thesis}  % Global Style for particular SMU College, e.g. DL_thesis.sty
\input{Dedman-Thesis-Latex-Template/latex/packages.tex}
% Load additional packages here.
% Be careful of the loading order with respect to packages.tex as conflicts can easily arise.
% This is not a part of the base template and so will never be overwritten during updates.
\usepackage{tikz}
\usepackage{graphicx}
\usepackage{tikz-feynman}
\usepackage{slashed}
\usepackage{lipsum}
\usepackage{fontspec}
\usepackage{placeins}
\usepackage[font=footnotesize]{caption}
\usepackage[bottom]{footmisc}
%\usepackage{mathtools}
\newfontface\mathcursivefont{Daytonia.otf}
\DeclareTextFontCommand{\mathcurse}{\mathcursivefont}
 % Uncomment to load additional user required packages
\input{Dedman-Thesis-Latex-Template/latex/preamble.tex}
\input{Dedman-Thesis-Latex-Template/latex/custom_commands.tex}
% Place your own personal commands here. 
% This is not a part of the base template and so will never be overwritten during updates.
\newcommand{\Xsec}{\mathcurse{S}}
\newcommand{\xsec}{{\Large \mathcurse{s}}}
\newcommand{\wField}{\mathcal{A}}
\newcommand{\amp}{\mathcurse{A}}
\newcommand{\matel}{\,\mathcurse{M}\,} % Matrix Element

\newcommand{\kvv}{\kappa_{2V}}
\newcommand{\kl}{\kappa_{\lambda}}
\newcommand{\kv}{\kappa_{V}}

\newcommand{\fkvv}[1]{\kappa_{2V,#1}}
\newcommand{\fkl} [1]{\kappa_{\lambda,#1}}
\newcommand{\fkv} [1]{\kappa_{V,#1}}

\newcommand{\Lag}{\mathscr{L}}
\newcommand{\bra}[1]{\big< #1 \big|}
\newcommand{\ket}[1]{\big| #1 \big>}
\newcommand{\braket}[1]{\big< #1 \big| #1 \big>}
\newcommand{\Tbraket}[2]{\big< #1 \big| #2 \big>}
\newcommand{\braketA}[2]{\big< #1 \big| #2 \big| #1 \big>}
\newcommand{\TbraketA}[3]{\big< #1 \big| #2 \big| #3 \big>}
 % Uncomment to use your own personal commands
\AuthorFirstName{Christopher}
\AuthorLastName{Milke}
% \AuthorNameSuffix{Jr.}

% Title can be up to three lines of no more than 48 characters per line
% Write the title with the capitalization you want to be used in the abstract
\ThesisTitle{Smashy Physics:}{How I Pretended to Do More Smashing}{Using Smashy Math}
\GraduateDepartment{Physics}

\FirstDegree%
\FirstDegreeType{B.S.}
\FirstDegreeMajor{Physics}
\FirstDegreeUniversity{University of California, Santa Cruz}

\SecondDegree%
\SecondDegreeType{M.S.}
\SecondDegreeMajor{Physics}
\SecondDegreeUniversity{Southern Methodist University}

\ThesisDefenseDateYear{2022}
\ThesisDefenseDateMonth{April}
\ThesisDefenseDateDay{1}

\GraduationDateYear{2022}
\GraduationDateMonth{July}
\GraduationDateDay{1}

\SMUCollege{Dedman College}

% Committee
\AdvisorFullName{Dr.~Stephen Sekula}
\AdvisorTitle{Associate Professor}

\CommitteeMemberA{Dr.~Roberto Vega}
\CommitteeMemberTitleA{Professor}

\CommitteeMemberB{Dr.~Jingbo Ye}
\CommitteeMemberTitleB{Associate Professor}

\CommitteeMemberC{Dr.~External Member}
\CommitteeMemberTitleC{Assistant Professor}

\input{latex/metadata.tex}

\makeglossaries
\input{src/glossary.tex}

% \thesisdraft % uncomment if want draft printing

\begin{document}

% DRAFT OMEGA 0
\input{Dedman-Thesis-Latex-Template/latex/front_pages.tex}

\begin{thesis}

%\input{src/preface.tex}

% Ch1: Introduction - DRAFT 2
\chapter{Introduction}\label{chapter:introduction}

\input{src/introduction/thesis_tutorial.tex}




% Ch2: Standard Model Discussion - DRAFT 2
\newcommand{\tinymatrix}[1]{ \big(\begin{smallmatrix} #1 \end{smallmatrix}\big) }
\newcommand{\minimatrix}[1]{{\scriptsize\arraycolsep=0.3\arraycolsep\ensuremath{\begin{pmatrix}#1\end{pmatrix}}}}

%NOTE: according to Steve, you can (and SHOULD) use "I" in this chapter. Indeed, use "I" as well literally anywhere I specifically was responsible for something. Do NOT use royal "we". Only use "we" in the context of what "we" the HDBS 4b or dihiggs group have done as a group, and only if you have explicitly defined who "we" are.

\chapter{Theory}\label{chapter:theory}

%So what even do I need to talk about here?
%
%Final goal = Higgs (final final goal should actually probably be di-higgs and a discussion of the relevence of this...)
%    What: What is the Higgs?
%    Why: Why do we care about it? It gives things mass and messes with the electroweak interaction
%    When: In what contexts does the Higgs become relevant? The mass of (most) elementary particles and the weak force being weak
%    Where: Where does the higgs fit into the SM? The "Higgs Mechanism"
%    How: How does the higgs mechanism work?

\section{Introduction}

    The Higgs Boson sits as the crown jewel of a grand overarching theory of the behaviour of the universe,
        known as the Standard Model of Particle Physics.
    While its recent discovery has shed light on many of its key properties,
        there are still many details of its nature that are as yet uncomfirmed.
    In this chapter, I want to explain what these properties are and how they can be further studied.
    Moreover, I want to justify why the Higgs is so important as to be worth studying in the first place.
    And to understand the importance of the Higgs Boson, one must understand the structure of the Standard Model itself.

    The structure of the following sections will begin with a discussion of the purpose and fundamental structure of the Standard Model.
    I will follow this with an introduction to the mathematical formalism the Standard Model is based around,
        involving Group Theory, the calculus of variations, and symmetry.
    Here I will introduce the reason for the original postulation of the Higgs, what it is, how it works, and how it fits into the Standard Model.
    Finally, I will motivate further study into the Higgs and provide a technique to perform this study.


%Matter and Forces
%I need to discuss the elementary particles and their organization before I can really discuss the gauge fields.
%Might as well do it here first and foremost
\section{The Standard Model of Particle Physics}
    
    At its core, the Standard Model of Particle Physics is a description of the behaviour and interaction of matter.
    First and foremost then, I want to discuss what this matter actually is.
    All matter can be described as a specific type of elementary particle called a fermion,
        defined by the fact that it contains no discernable substructure and possesses an intrinsic spin of 1/2.
    These particles all have different masses, and are able to interact with each other through three ``fundamental interactions''.
    The three fundamental interactions (more commonly called ``Fundamental Forces'') are known as
        the Electromagnetic, Weak, and Strong interactions (gravity is entirely absent in the Standard Model).
    All the interactions have an associated ``charge'' which can be ascribed to different particles,
        and which govern how strongly that particle can interact with similarly charged particles.
    The various fermions are distinct largely because of the charges they carry.

    \begin{figure}[h!]
        \includegraphics[width=\linewidth,height=\textheight,keepaspectratio]{theory/Standard_Model_of_Elementary_Particles}
        \caption{I'm probably going to need to find something else since this came from wikipedia. I just wanted a placeholder}
        \label{fig:sm_particles}
    \end{figure}
        

    There are twelve distinct elementary fermions (see Figure \ref{fig:sm_particles}),
        which are split evenly into two subgroups, called quarks and leptons.
    Quarks have a charge of 1 with the Strong interaction,
        while leptons have a charge of 0 (and thus cannot interact via the Strong Interaction at all).
    Both classes of particles, quarks and leptons, are divided into three ``generations'' of progressively heavier particles.
    Each generation thus consists of two quarks and two leptons.
    These pairs, called ``doublets'', behave the same across all generations.
    Among the quarks, every generation contains a doublet of an up-type quark (Up, Charm, Top) with electromagnetic charge of 2/3,
        and a down-type quark (Down, Strange, Bottom) with EM charge of -1/3.
    For leptons, each doublet consists of a particle with EM charge of -1, and a neutrino with EM charge of 0.

    In addition to the fermions, there is also an entirely seperate class of particles, called gauge bosons,
        which play a fundamental role in the aforementioned interactions.
    However, the nature of these particles will be discussed later.
    Now, with the enumeration of the various particles complete, it is time to begin the discussion of the Standard Model itself,
        which will serve to explain how these fermions interact with each other.



\section{Composition of Matter: Fields and Dirac Spinors}
    I'm actually starting to wonder if I should scrap this entire section and just merge the basic description of fields,
        as well as a brief mention of the Dirac equation, into the prior section.
    I don't think I'm going to go into detail about the weak-fermion coupling,
        so I don't really need to describe left/right fields, so I don't really need to break down the structure of
        the dirac and klein gordon equations.
    I might even just move the klein gordon equation introduction over to the Higgs section.

    %Fields are a thing.
    The description of matter is governed by the formalism of Quantum Field Theory, which describes particles as pertubations of an overall particle \textit{field}.
    %They're kinda like quantum waves, but not.
    A particle field $\varphi$, describes every particle of the same type at once,
        and particle states $\ket{\varphi}$ correspond to the number of that kind of particle that exists at a given moment.
    The equation of motion used to describe a field varies depending on the properties of the field.
    For a scalar (spinless) field, \phi, the equation of motion just comes from the relativistic mass-four-momentum relation $p^2=m^2$,
        but repurposed for a quantum field.
    This is the Klein-Gordon Equation
    \begin{equation} \begin{split}
        p^2 \phi = m^2 \phi
        \\(p^2 - m^2) \phi = 0
    \end{split} \end{equation}
    Recalling the covariant, natural units, quantum mechanical definition of momentum as $p^u = i\partial^u$,
        this can also be written as
    \begin{equation} \begin{split}
        (p^2 - m^2) \phi = 0 \rightarrow (\partial^\mu \partial_\mu + m^2) \phi = 0
    \end{split} \end{equation}

    The Klein-Gordan Equation will be used later when describing the Higgs Field,
        but first I will use it here as the foundation for the equation of motion required for a spinor field.
    For spinor fields, the equation of motion is the Dirac Equation,
        formulated to be a first-order equation (i.e.\ linear in $p$).
    \begin{equation}
        (\gamma^\mu p_\mu - m I) \psi = 0
        (i\gamma^\mu \partial_\mu  - m I) \psi = 0
    \end{equation}

    The issue with creating a first-order equation in $p$ is that $p_\mu$ is a vector, which means it cannot be merely added to the scalar $m$.
    The key feature of the Dirac Equation which addresses this is the inclusion of the \textit{gamma matrices}, $\gamma^\mu$.
    $\gamma^\mu$, is a vector of four matrices, each of size 4x4 (note that $m$ is in turn multiplied by a 4x4 identity matrix $I$).
    There are many different conventions for the exact structure of the matrices,
        but the main requirement is that they all multiply together with very specific commutation rules.
    The technical description of this property is that the gamma matrices must
        anti-commute as $\{\gamma^u, \gamma^\nu\} = 2g^{\mu \nu}\times I$,
        where $g^{\mu\nu}$ is the minkowski metric.
    In practical terms, the goal is just to ensure that when the Dirac Equation is multiplied by its conjugate,
        the product returns back to the Klein-Gordan Equation.
    \begin{equation} \begin{split}
        (\gamma^\nu p_\nu + m I) (\gamma^\mu p_\mu - m I) \psi = 0
        \\ (\gamma^\nu \gamma^\mu p_\nu p_\mu - m \gamma^\nu p_\nu + m \gamma^\mu p_\mu - m^2 I) \psi = 0
        \\ (\gamma^\nu \gamma^\mu p_\nu p_\mu - m^2 I) \psi = 0
        \\ (g^{\mu \nu} I p_\nu p_\mu - m^2 I) \psi = 0
        \\ (p^\mu p_\mu - m^2 ) \psi = 0
        \\ (p^2 - m^2) \psi = 0
    \end{split} \end{equation}
    Where the anticommutation relation has been used in the fourth line to eliminate all the momentum cross terms.




    %Fields don't describe a single particle, but all particles of the same type at once.
    %Field states are just a count of how many of that particles exist at a given time.
    %Scalar particles are described by Klein Gordon Equation.
    %Spin 1/2 particles are described by dirac equation.
    %It's first order in p, but this requires the gamma matrices.
    %Matrices take many forms, but one of the most revealing is the Weyl, or Chiral, form.
    %Do some math to show fields in chiral form, to show Chirality.


    %It takes from quantum mechanics the uncertainty principle, and from relativity the mass/energy equivalence 
    %It takes from quantum mechanics the wave-like description of position and momentum, along with the uncertainty principle.
    %From special relativity it incorporates the equivalence of space and time, the constance of the speed of light,
    %    and the exchange between mass and energy.
    %Combining the energy/time uncertainty principle with mass/energy equivalence in particular
    %    has the outcome that the number of particles for a given system need not remain constant.
    %To account for this, particles are not described as waves as with the Schrodinger Equation, but rather as ``fields''.
    %A fermion field, $\psi$, does not describe a single particle, but rather all possible fermion for the same kind of particle.
    %For instance, there is a seperate fermion field to describe electrons, up-quarks, and so forth.

    
    describe gamma matrices

    describe anti particle psi bar

    describe chiral representation
    

    %What is the Universe made of (If you stick to this format at all, keep this section brief):

    %%Position and Momentum: Quantum Mechanics
    %   The position and momentum of that matter:
    %       Described via the Canonical Commutation Relation and the formalism of Quantum Mechanics
    %       (explain 'h' here?)
    %   Describe basic function of schrodinger equation and why it fails

    %%Space and Time: Special Relativity
    %   The space-time in which that matter resides: described by the Minkowski Metric Tensor.
    %    Explain what the metric tensor means and maybe covariant notation,
    %    as well as mass-energy-momentum equivalence,
    %    and also the speed of light
    %    Reformat schrodinger equation as klein gordon and show how this also fails


    %%Particles and Fields: Quantum Field Theory
    %   A description of how the position of matter can change: Described by the Dirac Equation.
    %    A unification of Quantum Mechanics and Special Relativity
    %    (maybe go through the derivation, starting from schrodinger -> klein-gordon -> dirac and why each fails)
    %    Describe Weyl Spinors and Chiral representation (peskin pg 64)



\section{Generating Motion: Group Theory and Transformations}

    The Dirac Equation provides a description of matter.
    The next key piece of the Standard Model is a description of motion itself.
    For this, I will need to introduce Group Theory.

    A function can be altered using a \textit{transformation operator}.
    A simple example of this would be a function $x(t)$, 
        which (assuming constant velocity) can be transformed into a time $\Delta t$ in the future as
    \begin{equation}
    x(t) \rightarrow x'(t) = x(t) + v \Delta t
    \end{equation}
    Noting that $v$ is just $\frac{d}{dt} x(t)$, this can be rewritten as
    \begin{equation}
    x(t) \rightarrow x'(t) = x(t) + \Delta t \frac{d}{dt} x(t) = \left(1+\Delta t \frac{d}{dt}\right) x(t)
    \end{equation}

    This term $\left(1+\Delta t \frac{d}{dt}\right)$ is the classical time-translation operator.
    Notice the assumption of \textit{constant velocity} though.
    If velocity were not constant, this operator would be invalid, except for in the specific case in which $\Delta t$ is infinitesimal.
    \begin{equation}
    x(t) \rightarrow x'(t) = \lim_{\delta t \to 0} \left(1+\delta t \frac{d}{dt}\right) x(t)
    \end{equation}

    To produce a more general finite operator I can apply the infinitesimal operator an infinite number of times
    \begin{equation} \begin{split}
    x(t) \rightarrow x'(t) &= \lim_{\delta t \to 0} \left(1+\delta t \frac{d}{dt}\right)\left(1+\delta t \frac{d}{dt}\right)\left(1+\delta t \frac{d}{dt}\right)...\ x(t)
    \\x(t) \rightarrow x'(t) &= \lim_{N \to \infty} \lim_{\delta t \to 0} \left(1+\delta t \frac{d}{dt}\right)^N x(t)
    \\x(t) \rightarrow x'(t) &= e^{\Delta t \frac{d}{dt}} x(t)
    \end{split} \end{equation}

    Where $\Delta t$ is again a finite time transformation,
        and I have compressed the infinite product of terms using the power series expansion of the exponential function.
    In order to use this classical operator in quantum field theory, it must have a complex factor `$i$' associated with it,
    \begin{equation} \begin{split}
    x(t) \rightarrow x'(t) = e^{i\Delta t \frac{d}{dt}} x(t)
    \end{split} \end{equation}

    Returning to Dirac Fields, which are functions of four-position $x_\mu$, this same transformation can be used
         with the minor adjustment of changing the total derivative to a partial derivative in time, $\frac{partial}{\partial t} = \partial_0$
    \begin{equation} \begin{split}
    \psi(x) \rightarrow \psi'(x) = e^{i\Delta x^0 \partial_0} \psi(x)
    \end{split} \end{equation}

    Meanwhile, the Dirac anti-particle field transforms with a negative sign as
    \begin{equation} \begin{split}
        \psibar(x) \rightarrow \psibar'(x) = e^{-i\Delta x^0 \partial_0} \psibar(x)
    \end{split} \end{equation}
    
    The time translation operator is but one of a myriad of different tranformation operators.
    However, all of the transformation operators that will be used here will all take a form very similar to this one,
        appearing something like

    \begin{equation} \begin{split}
        \psi(x) \rightarrow \psi'(x) = e^{ q \cdot \mathcal{F}_q } \psi(x)
    \end{split} \end{equation}

    where $\mathcal{F}_q$ is the transformation \textit{generator} (e.g.\ $i\partial_0$),
        and $q$ the amount to transform by (e.g.\ $\Delta t$).

    These transforms all fall under a larger realm of mathematics known as Group Theory,
        and are in turn called ``group transformation''.
    The basic definition of a ``group'' is a set of elements which can be ``multiplied'' according to some rule,
        and which satisfies the four conditions of\cite{Cheng_book}:
    \begin{itemize}
        \item Closure - the product of any two elements of the group are still in that group;
        \item Associativity - $(a \times b)\times c = a\times(b \times c)$;
        \item Identity - there is some element in the group $I$ for which $I \times a=a$;
        \item and Inversion - every element $a$ has an inverse $a^{-1}$ such that if $b \times a = c$ then $c \times a^{-1} = b$.
    \end{itemize}

    It should be relatively simple to see that time translation operations satisfy each of these.
    Additional classifications for groups are whether a group is discrete or continuous,
        and whether a group is ``Abelian'' (commutative) or ``non-Abelian'' (non-commutatitve).
    Time translations are continuous (one can tranlate by an infinitesimally small amount of time),
        and are Abelian (the order that the translation are applied does not matter).
    An example of a non-Abelian group would be that of three-dimensional rotations (formally called the $SO(3)$ group),
        as applying rotations in a different order can lead to a different final product.

    As a final point regarding Group Theory, is the absolutely crucial idea of \textit{symmetry}.
    If a system can be transformed under a group transformation and remain overall unchanged,
        then that system is said to be invariant, or symmetric, under that group.
    In a very deep sense, Group Theory and group symmetry are the driving forces behind the entire Standard Model,
    As will be seen in the coming sections,
        the motion of the Standard Model is ultimately described by Group Theory,
        its interactions defined by Group Theory,
        and its entire organizational structure ultimately rooted in Group Theory.

\section{Restricting Motion: The Lagrangian and Symmetry}
    
    The Dirac Equation describes the equation of motion of a single field,
        but what is required is a way to describe the interactions of \textit{many} fields.
    For this, the Standard Model makes use of the Principle of Least Action,
        also known as the Lagrangian Formulation\cite{Halzen_book}.
    The idea of the Principle of Least Action is to construct an overall description of all aspects of a system,
        called the Lagrangian, $\Lag$.
    One can then find the ``path'' of this Lagrangian which minimizes the ``Action'', $\mathscr{A}$
    \begin{equation} %TODO: double check this!!
        \mathscr{A} = \int \Lag d^4 x
    \end{equation}

    This minimization is performed using the four-dimension Euler-Lagrange equations
    \begin{equation}
        \partial_\mu \frac{\partial \Lag}{\partial_\mu \varphi } - \frac{\partial \Lag}{\partial \varphi} = 0
    \end{equation}
    which yields the equations of motion of the interacting particles.

    All of the physics of the Standard Model is inscribed within the Standard Model Lagrangian.
    The construction of this Lagrangian will be the focus of most of the rest of the chapter.
    To begin then, what physics \textit{does} the Lagrangian need to encode?
    Paraphrasing Murray Gell-Mann, ``that which is not forbidden, is \textit{mandatory}.''
    That is to say, that any physics which is not expressly prohibited, must automatically be allowed.
    Formulation of the Standard Model Lagrangian must therefore start with an enumeration of restrictions,
        given in the form of a series of groups that the Lagrangian must remain symmetric under.
    Specifically, the Lagrangian must remain symmetric under ten transformations,
        collectively reffered to as the Poincare Group.
    Four of these transformations correspond to the space-time translations,
        three to spatial rotations, and the remaining three to space-\textit{time} rotations,
        also known as Lorentz Transformations (also known as just changing velocity).

    The Lagrangian of a Dirac field satisfying these symmetries takes the form
    \begin{equation}
        \Lag = \psibar (\slashed{p} - m) \psi = i \psibar \slashed{\partial} \psi - m \psibar \psi
    \end{equation}
    As an example of the symmetry now described, I will reuse the time translation operator $e^{i\Delta x^0 \partial_0}$

    \begin{equation} \begin{split}
        \Lag(t) \to \Lag' = \Lag(t+\Delta t) =
            \left( e^{-i\Delta x^0 \partial_0} \psibar \right) (\slashed{p} - m) \left( e^{i\Delta x^0 \partial_0}\psi \right)
        \\  = e^{-i\Delta x^0 \partial_0 + i \Delta x^0 \partial_0} \psibar (\slashed{p} - m) \psi
        \\  = \psibar (\slashed{p} - m) \psi
    \end{split} \end{equation}

    Applying the Euler-Lagrange equations to this equation will seperately return the Dirac Equation
        for the fermion and anti-fermion fields, as it should.

    As there are many different kinds of fermions, the Lagrangian can be expanded to include all of them
    \begin{equation}
        \Lag = \bar{e} (\slashed{p} - m) e
        + \bar{\mu} (\slashed{p} - m) \mu
        + \bar{\tau} (\slashed{p} - m) \tau
        + \bar{u} (\slashed{p} - m) u
        + ...
    \end{equation}

    Although this equation qualifies as a valid description of the behaviour of fermions,
        there is a glaring flaw with it: the fields it describes are all completely independent of one another.
    None of the fields are able to interact with each other.
    In its current form, the Lagrangian proposes that particles, given some initial momentum,
        will persist along the same trajectory forever, unable to affect or be affected by any other particle.
    The Lagrangian has succesfully constrained the range of allowed physics,
        but it has done so to such a degree that the described physics are now practically static.
    Clearly this is not a satisfactory description of the universe, full of dynamic interactions as it is.
    The resolution to this issue is, counter-intuitively, to enforce yet \textit{more} symmetry requirements.
    In doing so, the various interactions of the Standard Model will be laid bare,
        and the true beauty of the theory made apparent.

    %Should I also discuss renormalizability? (Peskin pg 80/101djvu)
    %Basically, all lagrangians must be renormalizable.
    %Renormalizability just means that the lagrangian doesn't explode from the unconstrained nature of virtual particles.
    %So infinite-mass virtual particles should not break a renormalizeable lagrangian.


\section{Transferring Motion: Gauge Symmetry}

    %Describe Global U1
    I want to introduce another group, called $U(1)$ (Unitary Rank 1 Group).
    This group corresponds to the \textit{phase} of a field, and the transformation takes the form
    \begin{equation}
        \psi(x) \to \psi'(x) = e^{i\theta} \psi(x)
    \end{equation}

    %describe global vs local
    As in prior examples, the exponential changes signs for the anti-fermion.
    The generator here is just $i$, the phase shift amount given by the angle $\theta$.
    It is trivial to see that the Lagrangian is invariant under such a group.
    This group, along with all the others already described under the Poincare group,
        all fall under the category of Global transformations.
    A global transformation is one which is applied uniformly across all space-time.
    There is a different kind of transformation, called a Local or ``Gauge'' transformation.
    A gauge transform is one in which the transformation amount varies \textit{at each point in space-time}.
    A local $U(1)$ transform would thus take the form 
    %show local U1
    \begin{equation}
        \psi(x) \to \psi'(x) = e^{i\theta(x)} \psi(x)
    \end{equation}
    Noting that the phase-angle $\theta(x)$ is now a function of $x$.

    Demanding that the Lagrangian remain invariant under such a transformation would clearly be an absurd mandate.
    Yet, this is exactly what nature seems to have done.

    %Show how this breaks lagrangian
    I want to show how the Lagrangian in its current form breaks when attempting to perform a $U(1)$ gauge transform,
        and how the resolution to this issue leads directly to fully dynamic, interacting Lagrangian.
    First, apply the gauge transformation
    \begin{equation} \begin{split}
        \Lag(\theta_0) \to \Lag' = \Lag(\theta_0+\theta(x)) &=
            \left( e^{-i\theta(x)} \psibar \right) (\slashed{p} - m) \left( e^{i\theta(x)}\psi \right)
            \\ &= \left( e^{-i\theta(x)} \psibar \right) i\slashed{\partial} \left( e^{i\theta(x)}\psi \right)
                - \left( e^{-i\theta(x)} \psibar \right) m \left( e^{i\theta(x)}\psi \right)
    \end{split} \end{equation}
    In the mass term, the transform will cancel in the usual fashion.
    The momentum term however will \textit{not} cancel, as the derivative will also affect the position-dependant $\theta(x)$.
    \begin{equation} \begin{split}
        \Lag' &= \left( e^{-i\theta(x)} \psibar \right) i\slashed{\partial} \left( e^{i\theta(x)}\psi \right) - m \psibar(x) \psi(x)
        \\ &= i \left( e^{-i\theta(x)} \psibar \right) \left[
                \left( \slashed{\partial} e^{i\theta(x)} \right) \psi 
                + e^{i\theta(x)} \left( \slashed{\partial} \psi \right)
            \right] - m \psibar(x) \psi(x)
        \\ &= i \left( e^{-i\theta(x)} \psibar \right)
                \left( \slashed{\partial} e^{i\theta(x)} \right) \psi 
            + i \psibar \slashed{\partial} \psi
            - m \psibar(x) \psi(x)
        \\ &= - \psibar \left( \slashed{\partial}\theta(x) \right) \psi 
            + \psibar ( \slashed{p} - m ) \psi
    \end{split} \end{equation}

    %Show how to fix lagrangian
    Under a local $U(1)$ transform, the Lagrangian is clearly not invariant,
        as a result of the additional $\slashed{\partial}\theta(x)$ term.
    In order to enforce gauge symmetry, a subtle but profound adjustment must be made to the Lagrangian.
    This adjustment will take the form of a modification to the derivative,
        replacing the standard derivative with a \textit{covarient} derivative, $\partial_\mu \to D_\mu \equiv \partial_\mu + i A_\mu(x)$.


    %introduce photon field and show newly produced interaction

    %at least briefly mention how equation of motion of fermion is now Lorentz force law,
    %    eom of A field is Maxwell's Equation

    %introduce SU(2) and SU(3)

    %Tease at U1, Su2, and su3 corresponding to EM, weak and strong.
    %Break this by explaining that w and z bosons have mass, which collapses the entire idea of basing the SM on symmetry.

    %Transition with the idea of the higgs as the savior of the entire theory
    


    %Gauge Transformations are ones where the transformation is imposed differently at each point in spacetime.
    %Trying to impose a constraint on the Lagrangian that it be symmetric under gauge transformations would surely cause all manner of complications.
    %Obviously, this is exactly what nature seems to have chosen to do.

    %Gauge symmetries; U1, SU(N).
    %The effects of imposing gauge symmetries on the Lagrangian, and the advent of the gauge bosons and their forces.
    %how do gauge bosons come out from symmetries.

    %U(1) is phase transforms

    %SU(2) is based on 2x2 pauli matrices and thus requires pairing generations of particles together;
    %    so up and down-type quarks are paired together and charged leptons with neutrinos.
    %It treats left handed fields as these doublet pairs,
    %    but works in "singlet representation" (which means it basically is just gone) for right-handed fields

    %SU(3) is based on a 3x3 structure constant, and thus acts on all three "generations" of quarks as one 3x1 vector.
    %It is a singlet (read, it literally doesn't matter) for leptons.
    
    \cite{Osborn_notes}
    \cite{Peskin_book}
    \cite{Halzen_book}



% This might actually need to become its own chapter...
\section{The Standard Model of Particle Physics}
    \subsection{The Higgs Mechanism}\label{sec:higgs_mechanism}

        Basic higgs mechanism example.

        Have a field interacting with a potential V with quadratic and quartic terms.
        This produces a symmetric but unstable equilibrium.
        Perform passive translation to recenter coordinates around stable equilibrium point/minimum of potential.
        In super-simple example, this gives an originally massless particle mass.

        Ok so let's start with a really basic lagrangian describing only the kinetic energy of a massless scalar particle $\phi$, of the form:

        \begin{equation}
            \Lag = K_{\phi} = \frac{1}{2} (\partial_{\mu} \phi)^2
        \end{equation}

        Let's introduce a quartic potential $V(\phi) = -\frac{1}{2} \mu^2 \phi^2 + \frac{\lambda}{4!} \phi^4$

        \begin{equation}
            \Lag = K_{\phi} - V{\phi} = \frac{1}{2} (\partial_{\mu} \phi)^2 
                +\frac{1}{2} \mu^2 \phi^2 - \frac{\lambda}{4!} \phi^4
        \end{equation}

        Such a potential will result in a Hamiltonian which is symmetric about $\phi=0$, but that point will be a local maximum.% TODO figure for this
        The Hamiltonian will have two minima to either side of $\phi=0$, at points $\pm \nu = \pm \sqrt{\frac{6}{\lambda}} \mu$.

        A system in such a potential would invariably fall into one of these minima.
        The Lagrangian can be rewritten from the perspective of one of these minima (e.g.\ $+\nu$),
            by substituting in a shifted field $h$, where $\phi(x)=\nu+h(x)$.
        The Langrangian now takes the form (after simplifying)

        \begin{equation} \begin{split} \label{eq:basic_higgs}
            \Lag & = \frac{1}{2} (\partial_{\mu} h)^2
                - \mu^2 h^2
                -\sqrt{\frac{\lambda}{6}} \mu h^3
                - \frac{\lambda}{4!} h^4 \\
             & = \frac{1}{2} (\partial_{\mu} h)^2
                - m^2_{h} h^2
                - k_{h} h^3
                - \frac{k_{2h}}{4!} h^4
        \end{split} \end{equation} %FIXME: you screwed up the terms somewhere

        With the latter equation taking the form of a now massive field $h$ with both a three and four point vertex,
            governed by two different coupling values $k_{h}$, and $k_{2h}$.

        \cite{Halzen_book}

        

    \subsection{Electro-Weak Symmetry Breaking}

        Let's now give the scalar field a complex phase and spinor components:
        \begin{equation}
            \phi(x) = \frac{1}{\sqrt{2}} e^{i \beta} \tinymatrix{\phi_1 \\ \phi_2}
        \end{equation}
        $\phi$ is still a scalar in spacetime, but now also has vector components in the $SU(2)$ subspace.

        As with the dirac fields, we will then impose $U(1) \times SU(2)$ gauge symmetry on the field, so it transforms as:
        \begin{equation}
            \phi(x) \rightarrow e^{i \alpha^a \tau^a} e^{i \beta/2 } \phi(x)
        \end{equation}.

        The added symmetries require that the derivative be changed to a covarient derivative 
            $D^{\mu} = \partial^{\mu} - \frac{ig}{2} \wField^a_{\mu} \sigma^a - \frac{ig'}{2} B_{\mu}$,
            producing a Lagrangian:
        \begin{equation} \begin{split}
            \Lag & = \frac{1}{2} (D_{\mu} h)^2
                - \mu^2 h^2
                -\sqrt{\frac{\lambda}{6}} \mu h^3
                - \frac{\lambda}{4!} h^4 \\
        \end{split} \end{equation}

        Once again, we allow the scalar field to fall into its offset vev $v$, as $\phi(x) \rightarrow h(x) + v$.
        Now however, $v$ is a spinor value $v = \tinymatrix{v_1 \\ v_2}$.
        With gauge freedom, this can be rotated entirely along one axis as $\vec{v} = \frac{1}{\sqrt{2}}\tinymatrix{0 \\ v}$,
            with $v = \sqrt{\mu^2/\lambda}$.
        Substituting into the Lagrangian now produces a more complex expression:
        \begin{equation} \begin{split}
            \label{eq:fullHiggs}
            \Lag & = \frac{1}{2} (D_{\mu}^{ij} (h+v)_j)^2
                + \mu^2 (h+v)^2
                - \frac{\lambda}{4} (h+v)^4 \\
             & = \Lag_h + \Lag_v
        \end{split} \end{equation}
        Where $\Lag_h$ takes a form similar to Equation \ref{eq:basic_higgs}, incorporating both the $h$ and $h$/$v$ cross terms.
        Meanwhile, $\Lag_v$ refers only to the terms arising from $D_{\mu}$ acting on the vev:
        \begin{equation}
            \label{eq:lagV}
            \Lag_v = \frac{1}{2} (D_{\mu}^{ij} v_j)^2
        \end{equation}

        The expansion of this term is crucial to the structure of the Standard Model,
            as it alone will lead to the breakdown of Electro-Weak Symmetry,
            and the $W$ and $Z$ bosons acquiring mass.

        %Get covariant derivative, evaluate at vev, pull out W,Z, and photon fields, and their masses (pg 722);
        Expanding only $D_{\mu}^{ij} v_j$ to start, the $\partial_{\mu}$ immediately vanishes ($v$ is a constant), yielding 
        \begin{equation} \begin{split}
            D^{\mu} v  = \big( \partial^{\mu} & - \frac{ig}{2} \wField^a_{\mu} \sigma^a - \frac{ig'}{2} B_{\mu} \big) \frac{1}{\sqrt{2}}\minimatrix{0\\v} \\
            = \big( & - \frac{ig}{2} \wField^a_{\mu} \sigma^a - \frac{ig'}{2} B_{\mu} \big) \frac{1}{\sqrt{2}}\minimatrix{0\\v} \\
            = - \frac{i}{2} \big( & g \wField^a_{\mu} \sigma^a + g' B_{\mu} \big) \frac{1}{\sqrt{2}}\minimatrix{0\\1} v
        \end{split} \end{equation}

        It is useful here to fully expand the $U(1) \times SU(2)$ fields into their matrix components and add them explicitly,
            as doing so reveals the origin of the photon and W and Z bosons.
        \begin{equation} \begin{split}
            g \wField^a_{\mu} \sigma^a + g' B_{\mu} & =
                g \wField^1_{\mu} \sigma^1
                + g \wField^2_{\mu} \sigma^2
                + g \wField^3_{\mu} \sigma^3
                + g' B_{\mu} I \\
            & = \begin{pmatrix}
                0 & g\wField^1_{\mu} \\ g\wField^1_{\mu} & 0 \end{pmatrix}
                + \begin{pmatrix} 0 & -ig\wField^2_{\mu} \\ ig\wField^2_{\mu} & 0 \end{pmatrix}
                + \begin{pmatrix} g\wField^3_{\mu} & 0 \\ 0 & -g\wField^3_{\mu} \end{pmatrix}
                + \begin{pmatrix} g'B_{\mu} & 0 \\ 0 & g'B_{\mu}
            \end{pmatrix} \\
            & = \begin{pmatrix} 
                g\wField^3_{\mu} + g'B_{\mu} & g\wField^1_{\mu} - ig\wField^2_{\mu} \\
                g\wField^1_{\mu} + ig\wField^2_{\mu} & -g\wField^3_{\mu} + g'B_{\mu}
            \end{pmatrix}
        \end{split} \end{equation}

        The four components of this matrix are the gauge boson fields of the electromagnetic ($A$) and weak ($W$ \& $Z$) interactions
        \begin{equation} \begin{split}
            \begin{pmatrix} 
                g\wField^3_{\mu} + g'B_{\mu} & g\wField^1_{\mu} - ig\wField^2_{\mu} \\
                g\wField^1_{\mu} + ig\wField^2_{\mu} & -g\wField^3_{\mu} + g'B_{\mu}
            \end{pmatrix} =
            \begin{pmatrix} 
                \sqrt{g^2 + g^{\prime 2}}\ A_{\mu} & g \sqrt{2}\ W^+_{\mu} \\
                g \sqrt{2}\ W^-_{\mu} & - \sqrt{g^2 + g^{\prime 2}}\ Z^0_{\mu}
            \end{pmatrix}
        \end{split} \end{equation}

        With $A$, $W^+$, $W^-$, and $Z^0$ related to the unbroken $\wField^a$ and $B$ fields by
        \begin{equation} \begin{split}
            A_{\mu} & = \frac{1}{\sqrt{g^2 + g^{\prime 2}}} ( g\wField^3_{\mu} + g'B_{\mu} ) \\
            Z^0_{\mu} & = \frac{1}{\sqrt{g^2 + g^{\prime 2}}} ( g\wField^3_{\mu} - g'B_{\mu} ) \\
            W^{\pm}_{\mu} & = \frac{1}{\sqrt{2}} (\wField^1_{\mu} \mp i\wField^2_{\mu})
        \end{split} \end{equation}

        The $\sqrt{g^2 + g^{\prime 2}}$ factor is the result of converting between
            $\tinymatrix{Z^0 \\ A}$ and $\tinymatrix{\wField^3 \\ B}$ by way of a rotation matrix
        \begin{equation} \begin{split}
            \begin{pmatrix} Z^0 \\ A \end{pmatrix} =
            \begin{pmatrix}
                \frac{g}{\sqrt{g^2 + g^{\prime 2}}} & \frac{-g'}{\sqrt{g^2 + g^{\prime 2}}} \\
                \frac{g'}{\sqrt{g^2 + g^{\prime 2}}} & \frac{g}{\sqrt{g^2 + g^{\prime 2}}}
            \end{pmatrix} \begin{pmatrix} \wField^3 \\ B \end{pmatrix} = 
            \begin{pmatrix}
                \cos\theta_w & -\sin\theta_w \\
                \sin\theta_w & \cos\theta_w
            \end{pmatrix} \begin{pmatrix} \wField^3 \\ B \end{pmatrix}
        \end{split} \end{equation}

        Where $\theta_w \equiv \cot(\frac{g'}{g})$ is known as the \textit{weak mixing angle}.

        Returning now to Equation \ref{eq:lagV}, we now have
        \begin{equation} \begin{split}
            \Lag_v & = \frac{1}{2} (D_{\mu}^{ij} v_j)^2 \\
            & = \frac{1}{2}
                \frac{1}{\sqrt{2}} \begin{pmatrix} 0 & v \end{pmatrix}
                \left| -\frac{i}{2}
                    \begin{pmatrix} 
                        \sqrt{g^2 + g^{\prime 2}}\ A_{\mu} & g \sqrt{2}\ W^+_{\mu} \\
                        g \sqrt{2}\ W^-_{\mu} & - \sqrt{g^2 + g^{\prime 2}}\ Z^0_{\mu}
                    \end{pmatrix}
                \right|^2
                \frac{1}{\sqrt{2}} \begin{pmatrix} 0 \\ v \end{pmatrix} \\
            & = \frac{1}{2} \frac{v^2}{2} \frac{1}{4}
                \begin{pmatrix} 0 & 1 \end{pmatrix}
                \begin{pmatrix} 
                    \sqrt{g^2 + g^{\prime 2}}\ A_{\mu} & g \sqrt{2}\ W^+_{\mu} \\
                    g \sqrt{2}\ W^-_{\mu} & - \sqrt{g^2 + g^{\prime 2}}\ Z^0_{\mu}
                \end{pmatrix}
                \begin{pmatrix} 
                    \sqrt{g^2 + g^{\prime 2}}\ A_{\mu} & g \sqrt{2}\ W^+_{\mu} \\
                    g \sqrt{2}\ W^-_{\mu} & - \sqrt{g^2 + g^{\prime 2}}\ Z^0_{\mu}
                \end{pmatrix}
                \begin{pmatrix} 0 \\ 1 \end{pmatrix} \\
            & = \frac{1}{2} \frac{v^2}{2} \frac{1}{4}
                \begin{pmatrix} 
                    g \sqrt{2}\ W^-_{\mu} & - \sqrt{g^2 + g^{\prime 2}}\ Z^0_{\mu}
                \end{pmatrix}
                \begin{pmatrix} 
                     g \sqrt{2}\ W^+_{\mu} \\
                     - \sqrt{g^2 + g^{\prime 2}}\ Z^0_{\mu}
                \end{pmatrix} \\
            & = \frac{1}{2} \frac{v^2}{2} \frac{1}{4} 
                \left[ 2 g^2  W^-_{\mu} W^+_{\mu}
                + \left(\sqrt{g^2 + g^{\prime 2}}\right)^2 (Z^0_{\mu})^2 \right] \\
            & = \frac{1}{2} \left[ \left(\frac{vg}{2}\right)^2\  W^-_{\mu} W^+_{\mu}
                + \frac{1}{2} \left(\frac{v}{2}\sqrt{g^2 + g^{\prime 2}}\right)\ (Z^0_{\mu})^2 \right]
        \end{split} \end{equation}

        As in Section \ref{sec:higgs_mechanism}, the $W$ and $Z$ fields now have additional mass terms associated with their kinetic energy terms,
            with $M_W = \frac{vg}{2}$ and $M_Z = \left(\frac{v}{2}\sqrt{g^2 + g^{\prime 2}}\right) $.
        The photon field $A_{\mu}$ is notably absent in the final vev product, and thus is left massless.
        A similar procedure can then be followed to allow the Higgs Field to interact with fermions
            (albeit with complications arising from mass-mixing and chirality), which will grant mass to the fermion particles.


        % Fermion time!!
    %\subsection{Giving Mass to Fermions}
    % Ok this is seriously just the same thing but now we have to split the fermion fields into left and right and up-type and down-type
    % and split the higgs into the higgs and conjugate higgs and also insert the complex mass matrix except don't insert it for charged leptons
    % because reasons or something. And then poof your charged leptons have mass and your quarks have mass and flavour changing
        

        %Split fermions between right and left fields, assign LH to SU(2) doublets (T=+-1?), RH to SU(2) singlet (T=0), assign Y too (pg724-725);

        %Sandwitch covariant derivative terms between fermion fields, expand to get field currents (pg 725-726);

        %*Try* to expand masses and fail because of representation incompatibilities (pg 725);

        %Anamoly cancellation thing I'll probably ignore (pg 726);

        %Add higgs interaction psibar phi psi to fermion lagrangian, expand into higgs interaction, convert to fermion mass (pg 734);



        
    \subsection{The Higgs Boson and di-Higgs Interactions}

        With the critical role of the Higgs Field established, it is now time to return to Equation \ref{eq:fullHiggs},
            and investigate $\Lag_h$, the Lagrangian of the Higgs Boson itself.
        The terms involved therein provide information not only about the Higgs Field,
            but also provide insight into how the Higgs may be further studied.
        \begin{equation} \begin{split}
            \Lag & = \frac{1}{2} (D_{\mu}^{ij} (h+v)_j)^2
                + \mu^2 (h+v)^2
                - \frac{\lambda}{4} (h+v)^4 \\
            = \big( \partial^{\mu} & - \frac{ig}{2} \wField^a_{\mu} \sigma^a - \frac{ig'}{2} B_{\mu} \big) \frac{1}{\sqrt{2}}h \\
                - \mu^2 h^2
                -\sqrt{\frac{\lambda}{6}} \mu h^3
                - \frac{\lambda}{4!} h^4 \\
        \end{split} \end{equation}


        
        
        %Further work must then be done to couple the higgs boson to fermions and itself
        %You need to tie k2v, kl, and kv in to the shape of the higgs potential here


%How do we test any of this?
%From Lagrangian to Cross-Section:
%    I need to study up on exactly how you go from the lagrangian to the Feynman rules, and from there to a calcualable cross section
\section{From Theory to Experiment: The Feynman Rules and Cross-Sections}
    
    %Justify why we care about cross sections
    After all this discussion of theory, the obvious question to ask is: how can this be tested.
    The most direct physically observable effects of the equations of the Standard Model are those of \textit{cross-sections}.
    Cross-sections will be discussed in more detail in Section \ref{sec:lhc-interaction_region},
        but for now it is sufficient to state that the probability of some physical interaction taking place is directly proportional to its cross-section.
    Here I will provide a general outline for how to produce a measurable cross-section from the Lagrangian of Equation \ref{eq:higgskappas}.

    In quantum mechanics, probabilities are measured as the absolute square of amplitudes of wave functions, $\left|\braket{\psi}\right|^2$.
    The probability of a transition between different states of a wavefunction are simililarly represented
        as the absolute square of the original state, $\psi_i$, \textit{in the basis of} the final state $\psi_f$,
        written $ |\Tbraket{\psi_f}{\psi_i}|^2$.
    In Quantum Field Theory, states correspond to which particles are in existence at a given moment.
    Thus, a state of one electron and one anti-electron could be written as $\ket{e \bar{e}}$,
        and the transition of an electron-positron pair into a muon/anti-muon pair could be written
        as $\Tbraket{\mu \bar{\mu}}{e \bar{e}}$.

    The core process used in this paper to probe the Higgs' $\kappa$ values is that of Vector Boson Fusion to two Higgs Bosons.
    The initial state of this process is two incoming quarks, $\ket{q_{i1} q_{i2}}$.
    These quarks each emit a vector boson (either $W^{\pm}$ or $Z^0$), 
    which should in turn fuse into two Higgs Bosons.
    The initial quarks then continue on, significantly deflected by their emissions, and possibly flavor-changed if they emitted a charged $W$.
    The final state of this process thus consists of two Higgs and two deflected quarks, $\bra{h_1 h_2 q_{f1} q_{f2}}$.
    The transition of this process would then be written as $\Tbraket{ h_1 h_2 q_{f1} q_{f2}}{q_{i1} q_{i2}}$.
    It should be noted that there are other intermediate processes (besides VBF)
        that could produce these same initial and final states, but these will not be considered in this analysis.

    In principle, this transition process can take an indeterminate period of time.
    In the realm of high energy physics experiments though,
        the interacting particles are moving so fast that the interaction period can be thought of as occuring at a single instant in time.
    Given this context, the initial state occurs in the (comparitively) distant past ($t_i$), and the final state in the equally distant future ($t_f$).
    Since the transistion occurs at an instantaneous moment,
        I need to perform a time-translation transformation both states to place them at the moment of the transition ($t_0$).
    Using the Hamiltonian $H$ as the time translation operator,
        I can relate the initial state at $t_0$ to its time $\Delta t$ units in the future, $t_i$, by the tranformation
    \begin{equation}
        \ket{q_{i1} q_{i2} (t_i)} = e^{i\Delta tH}\ket{q_{i1} q_{i2} (t_0)}
    \end{equation}
    The same can be done to transorm the final state backwards in time
    \begin{equation}
        \bra{h_1 h_2 q_{f1} q_{f2} (t_f)}
        = \bra{h_1 h_2 q_{f1} q_{f2} (t_0)} (e^{i(-\Delta t)H})^\dag
        = \bra{h_1 h_2 q_{f1} q_{f2} (t_0)} e^{i\Delta tH}
    \end{equation}
    Putting both of these together yields
    \begin{equation} \begin{split}
        \Tbraket{ h_1 h_2 q_{f1} q_{f2} (t_f)}{q_{i1} q_{i2} (t_i)}
        &= \TbraketA{ h_1 h_2 q_{f1} q_{f2} (t_0)}{e^{i\Delta tH} e^{i\Delta tH}}{q_{i1} q_{i2} (t_0)}
        \\&= \TbraketA{ h_1 h_2 q_{f1} q_{f2} (t_0)}{e^{i2\Delta tH}}{q_{i1} q_{i2} (t_0)}
        \\&= \TbraketA{ h_1 h_2 q_{f1} q_{f2} (t_0)}{1 + iT}{q_{i1} q_{i2} (t_0)}
    \end{split} \end{equation}

    In the last step, the exponential operator is expanded as an infinite series of terms.
    The first of these terms will just be 1, corresponding to the static situation in which no interaction occurs at all.
    The sum of the remaining terms, represented as $iT$, is the part relevant for calculating the interaction probability.
    Calculation of the value $\TbraketA{ h_1 h_2 q_{f1} q_{f2} (t_0)}{iT}{q_{i1} q_{i2} (t_0)}$ involves both
        the kinematics of the incoming and outgoing particles as well as the physics involved in the interaction itself.
    As such, it is useful to fuck all this I'm just going to skip the stupid transition amplitude stuff because
        there's no clear transition from transition amplitude to cross section.




    




    %basic formalism of cross-sections calculation as vacuum to particles (is that right?) to produce matrix elements

    %individual matrix elements as different internal processes

    %feynman diagrams as representation of matrix elements.
    %explain how feynman diagrams work.
    %Explain how I'm not going to deal with loop corrections...
    %    except maybe I should briefly mention this?
    %We do use N3LO for SM...

    %Explain how feynman rules produce matrix element values for each diagram.

    %Cover the main three diagrams for VBF->HH

    %Explain how these cross-section values go up with energy. Thus LHC
    %Mention that cross-section will be explained more later.
    %Important thing now is just to note that higher cross-section = more likely that process will occur



%    things I'm thinking of talking about but I'm not sure I actually need to discuss and also I'm not sure which section to put them in:
%
%        The poincare group transformations and their conserved quantities
%            
%            Translation: $\psi(x) \rightarrow e^{-\Delta x^{\mu} \partial_{\mu}} \psi(x)$
%
%            $U(1)$ Global: $\psi(x) \rightarrow e^{i \alpha} \psi(x)$
%
%            $U(1)$ Local: $\psi(x) \rightarrow e^{i \alpha(x)} \psi(x)$
%
%            $SU(2)$ Global:  $\psi_i(x) \rightarrow  U(\epsilon)_{ij} \psi_j$; $U_{ij} \equiv e^{i \epsilon_a (h_{a,ij} / 2)}$; $a=1,2,3$
%                $h_a$ are the pauli matrices
%                %- (cheng djvu pg 97 for details)
%
%            $SU(2)$ Local:  $\psi_i(x) \rightarrow  e^{i \epsilon_a(x) (h_{a,ij} / 2)} \psi_j$
%                %- (cheng djvu pg 238 for details)
%
%            $SU(3)$ Global:  $\psi_i(x) \rightarrow  U(\epsilon)_{ij} \psi_j$; $U_{ij} \equiv e^{i \epsilon_a (\lambda{a,ij} / 2)}$; $a=1,..., 8$
%                $\lambda_a$ are 3x3 traceless hermitian matrices
%            
%
%        what does a basic transformation look like. Show how you go from (1+da/dq) to e\^(dq*da/dq).


% Ch3: LHC - DRAFT 2
\chapter{The LHC}\label{chapter:lhc}
%START:
%    Theoretically, the dihiggs production is a thing that should exist, and which we would like to observe. 
%    The higgs does not exist in an observable state naturally, so it must be artificially created.
%    For this we need the LHC.
%
%IN BETWEEN:?
%    What's the LHC? A very large proton-proton particle collider.
%
%    What's a particle collider? A machine which accelerates charged particles to very high velocities/energies using electro-magnetic fields.
%
%    Why protons? Protons are less susceceptible to synchrotron radiation. Also protons consist of strongly-interacting particles.
%
%        Why are protons less susceceptible to synchrotron radiation? TODO
%
%        Why does strong interaction matter? (does it?) TODO
%
%        Also note that protons have the downside of PDFs
%
%    Why are we smashing particles together? Because (as should have been shown in the last section?) the cross section of a particle interaction increases proportional to the particles' energies
%
%    How do we make particles go fast? Injection system + magnets on main ring, also discussion of centripital force/mag-field/radius energy limitations  TODO
%
%    How do you smash particles together? Interaction region TODO
%
%    How often can it make a higgs? or a dihiggs? This can be found by multiplying the cross-section of the desired interaction by luminosity
%
%        What is luminosity? A combination of many things:
%            tightness of beam at IR,
%            center of mass energy machine is running at,
%            how many interactions per second (interaction rate and bunch crossing),
%            how long the machine is running;
%            All of the above have changed over time, per Run 1,2,3
%
%END:
%    We have succesfully produced a di-higgs event.
%
%  ||||||||||
%  VVVVVVVVVV
%
%LAYOUT:
%    INTRO:
%        Theoretically, the dihiggs production is a thing that should exist, and which we would like to observe. 
%        The higgs does not exist in an observable state naturally, so it must be artificially created.
%        For this we need the LHC.
%
%        What's the LHC? A very large proton-proton particle collider.
%        What's a particle collider? A machine which accelerates charged particles to very high velocities/energies using electro-magnetic fields.
%        Why are we smashing particles together? Because (as should have been shown in the last section?) the cross section of a particle interaction increases proportional to the particles' energies
%
%    ACCELERATOR RING
%        How do we make particles go fast? Injection system + magnets on main ring, also discussion of centripital force/mag-field/radius energy limitations  TODO
%        Why protons? Protons are less susceceptible to synchrotron radiation. Also protons consist of strongly-interacting particles.
%            Why are protons less susceceptible to synchrotron radiation? TODO
%            Why does strong interaction matter? (does it?) TODO
%            Also note that protons have the downside of PDFs
%
%
%    INTERACTION REGION
%        How do you smash particles together? Interaction region TODO
%        How often can it make a higgs? or a dihiggs? This can be found by multiplying the cross-section of the desired interaction by luminosity
%            What is luminosity? A combination of many things:
%                tightness of beam at IR,
%                center of mass energy machine is running at,
%                how many interactions per second (interaction rate and bunch crossing),
%                how long the machine is running;
%                All of the above have changed over time, per Run 1,2,3
%
%END:
%    We have succesfully produced a di-higgs event.

\section{Introduction} TODO
    % Theoretically, the dihiggs production is a thing that should exist, and which we would like to observe. 
    % The higgs does not exist in an observable state naturally, so it must be artificially created.
    % For this we need the LHC.
    The high mass and short life-time of the Higgs Boson ensures that it cannot be readily found in nature.
    In order to study the Higgs Boson then, it must first be artificially created through extremely high-energy physical interactions.
    The Large Hadron Collider (LHC), among the largest and most complex machines ever constructed, was designed for exactly this purpose.
    Built by the European Organization for Nuclear Research (CERN, from the French \textit{Conseil Européen pour la Recherche Nucléaire}) with the goal of studying high-energy physics,
    the LHC is able to probe interaction energies well beyond that of any previous particle physics experiments.

    % What's the LHC? A very large proton-proton particle collider.
    % What's a particle collider? A machine which accelerates charged particles to very high velocities/energies using electro-magnetic fields.
    % Why are we smashing particles together? Because (as should have been shown in the last section?) the cross section of a particle interaction increases proportional to the particles' energies
    % Brief discussion of fixed vs dual-beam collider and CoM energy calulation
    % General history, size, specs TODO 
    The LHC is a proton-proton particle accelerator; indeed, it is the largest particle acclerator ever built.
    A particle accelerator is a machine which uses electromagnetic fields to accelerate charged particles to extremely high energies, with the goal of colliding these particle together.
    The reason for doing this is that the cross section of a particle interaction scales with the center-of-mass energy of the incoming particles. %TODO try to back this up with specific details eg a formula from QFT showing how the probability of an interaction depends on energy
    Older accelerators operated in a fixed-target arrangment, in which a single beam of particles was accelerated into a stationary wall of material.
    The energy of such collisions was [FIXME: insert formula for Ecom]
    Newer accelerators, including the LHC, use two independant particle beams which are collided with each.
    Due to the properties of Lorentz Boost transformations, these dual-beam colliders achieve much higher com energies of [FIXME: insert other formula here] \cite{modern_and_future_colliders}
    Construction of the LHC took place between 1995 and 2007, over 40 meters underground beneath the French/Swiss border, near Geneva Switzerland. 




\section{Accelerator Ring} TODO
    % How do we make particles go fast? Injection system + magnets on main ring, also discussion of centripital force/mag-field/radius energy limitations  TODO
    % Why protons? Protons are less susceceptible to synchrotron radiation. Also protons consist of strongly-interacting particles. TODO
    % Why are protons less susceceptible to synchrotron radiation? TODO
    % Why does strong interaction matter? (does it?) TODO
    % Also note that protons have the downside of PDFs TODO
    Gotta go fast! (Beam injection and main ring specs).
    I'm noticing there isn't much discussion on the actual beam injection, so for now I actually think I might just combine it with the main ring section.



\section{Interaction Region} TODO
    % How do you smash particles together? Interaction region TODO
    % How often can it make a higgs? or a dihiggs? This can be found by multiplying the cross-section of the desired interaction by luminosity TODO
    % What is luminosity? A combination of many things: TODO
    % tightness of beam at IR, TODO
    % center of mass energy machine is running at, TODO
    % how many interactions per second (interaction rate and bunch crossing), TODO
    % how long the machine is running; TODO
    % All of the above have changed over time, per Run 1,2,3 TODO
    Particles go smash
    - the beampipe focusing magnets,
    - beam crossing point: how do bunches cross each other and interact
    - luminosity: what is it, what determines it 
    - bunch crossing
    - interaction rate: ~1000 particles every 25 ns w/in |eta| < 2.5.


% Ch4: ATLAS - DRAFT 2
% Structure
%    intro
%    purpose
%    (maybe) helix coords (appendix?)
%    general barrel/endcap cylindrical structure
%    walkthrough of the systems, from inner to outer, discussing why they are there and their basic purpose
%    Then split off into sections for the individual subsystems


    %Discussion of radiation hardness?
    %Things in the endcap suffer from more radiation exposure than things in the barrel
    %    (you should be able to show this from the basic kinematics of the particle beams. most energy is deposited in parallel to the beams, not orthogonal to them)
    %things closer to the IR suffer more than things further away (literally just the inverse-square law)

    %The method by which these detectors work presents a logistical challenge as to their placement.
    %Specifically, the calorimeters measure particle energy through a purely destructive process.
    %Current technology to measure the energy of fast-moving particles is done through a process that stops a particle in its path, and often forces it to decay into a vast number of lower-energy particles, through a process known as "showering".
    %Once this energy measurement is complete, the measured particle has been either absorbed or completely destroyed, making further measurement of its properties impossible.
    %As such, measurement of a particle's momentum and charge must be done prior to the energy measurment. This requirement determines the location of the different sub-detectors.
    %The tracking detectors are placed the closest to the interaction region. The tracking detecor barrels are closest in $r$,
    %and the tracking endcaps closest in $z$.

        
    %TODO things I don't really know...
    %Why is the muon chamber last? Why does it get its own special chamber? does it detect other things?
    %    Because punch through all the other detectors without stopping, due to [look up the reason in that energy source]. Thus, it goes last so that muons can be id'd after everything else HAS been stopped.
    %Is their a preference in detector placement? angular resolution gets better as you move further out; does this factor into anything? is this why further out things can be designed using lower resolution tech?
    %is there any reason to prefer things closer to the IR
    %    yes, this allows detection of short lived particles like b-quarks and taus \cite{CERN-LHCC-97-016}
    %Why is the only way to measure energy destructive? is there a way to measure energy non-destructively? What does the muon chamber do? does it destroy things? does it not measure energy? If so, how?
    %Can we measure neutral particles in trackers? how
    %    we can, but only photons because they interact through EM. Things like neutrons and pi0 go through it largely undetected until the hcals
    %Do we measure particle mass with trackers (via mass/charge ratio), or just charge? Do we just get mass from energy momentum calculation?


\chapter{ATLAS}
    % Intro
    Production of new physics and particles is of little use without the ability to observe said physics.
    Herein lies the purpose of ATLAS.
    One of the two general purpose detectors at the LHC, construction of the ATLAS detector was completed on October 4, 2008.
    ATLAS is among the largest particle detectors ever built, measuring 46m long with a 25m diameter, and weighing in at 7,000 tonnes \cite{atlas_website}.

    % Purpose
    Core purpose is to accurately record the physical properties of the particle interactions which take place in the interaction region of the detector.
    Most of the particles of interest are extremely short-lived, and so their properties cannot be measured directly.
    Instead, we must detect the decay products of these particles, and reconstruct the original particles of interest after the fact.
    Accurate reconstruction of these original particles is critically dependant on measuring, as precisely as possible, the physical properties of the decay products.
    More specifically, ATLAS is designed to record the paths and decay showers of the particles which pass through the detector, in order to determine their mass, energy, momentum, and electric charge.

    % General barrel/endcap cylindrical structure
    In order to measure all the required properties, ATLAS is divided into many different subsystems.
    Each of these subsystems has a very different design and objective, but they are all constructed with roughly the same overall cylindrical geomtry.
    The reason for this design is simple kinematics.
    The LHC particle beams cross with no initial transverse momentum, which means particles are ejected without preference in the radial angle $\phi$.
    Furthermore, the extremely high longitudinal momentum of the beams results in many particles continuing along a highly "forward" (parallel to the beampipe) trajectory.
    These two properties lead naturally to a radially symmetric detector which is elongated in the forward direction; a cylinder centered on the beam axis.
    To accomodate this geometry, the various sub-detectors of ATLAS are generally split into two distinct parts, called "barrels" and "endcaps".
    The barrels are a series of radially symmetric cylindrical shells, concentric about the beampipes, meant to detect particles moving primarily in the transverse direction.
    Conversely, the endcaps are a series of flat, circular plates, stacked one behind the next along the beampipes, intended to detect more forward particles.

    % Walkthrough of the systems, from inner to outer, discussing why they are there and their basic purpose
    The various detector subsystems can be broken up into three primary groups, based on the subsystems' purpose.
    Moving out from the interaction region, the subsystems can be classified as belonging to the Inner Detector, the Calorimeters, or the Muon Spectrometer.
    The first of these, located as close in $r$ and $z$ as possible, is the Inner Detector system.
    The Inner Detector is desigined to provide momentum measurement, vertexing, and electron identification.
    It must be located so close to the interaction region in order to permit detection of short lived particles like b-quarks and taus \cite{CERN-LHCC-97-016}.
    Following immediately behind (endcap) and around (barrel) the Inner Detector is the collection of sensors comprising the Calorimetry system.
    These sensors are purpose-built to measure the energy of incoming particles and additionally provide supplementary tracking information for particle trajectories
    Their location between the IR and Muon System is based on their tertiary purpose, which is to shield the Muon system from escaping hadrons.
    Finally, at the outer edge of ATLAS, is the Muon Spectrometer, which has been built to measure the momentum of muons leaving the detector volume.
    %TODO insert overall picture of ATLAS here for reference
    


% Then split off into sections for the individual subsystems


\section{Inner Detector} %TODO
    % Purpose of subsystem
    % Basic specs
    % What mechanism is used to achieve this purpose
    % What are the individual detectors and how do they contribute to this goal
    
    The Inner Detector system is intended to provide measurement of particles' momentum, provide vertex information, and help in identifying electrons.
    The way it achieves these goals is primarily through a series of very high resolution tracking sensors, which are used to trace out the paths that particles traverse as they leave the IR. 
    Momentum and charge measurement is facilitated by using a solenoid magnet to encompass the entire Inner Detector with a 2T axial magnetic field.
    This field bends the high-momentum charged particles into a helical trajectory, allowing their momentum, mass, and charge to be determined from the shape of the particle's path.
    The entire ID system, consisting of three independent detectors, measures 5.3 m in length and 2.5 m in diameter, and is able to provide accurate tracking within $|\eta| < 2.5$ \cite{id_tdr}.
    From the innermost to outermost, the sub-detectors are the Pixel Detector, the Semiconductor Tracker (SCT), and the Transition Radiation Tracker (TRT).
    The Pixel Detector and SCT together are responsible for high resolution tracking of particle trajectories.
    Further from the IR, the TRT provides more particle tracking capability, as well the ability to help distinguish electrons. 


    \subsection{Pixel Detector and Semiconductor Tracker}
        % purpose
        Purpose is to provide high resolution position and momentum information about particles as they leave the IP,
        They ensure that any particle exiting the IR crosses at least seven detector layers (3 pixel, 4 strip), while having minimal effect on the particle trajectory and energy.


        %what kinds of particles will they detect,
        They detect any kind of ionizing radiation, which is either photons or any particle with electric charge.

        % what kind of detector tech are they,
        Both the pixel detector and semiconductor tracker are semiconductor diode-based detectors.

        % how does this tech work? what happens as particles pass through them, and how do we convert that response into something useful
        Semiconductor diode detectors function by exploiting the properties of semiconductor p-n junctions.
        These particular detectors are made using silicon.
        Silicon has four valence electrons, so a pure silicon crystal lattice will have its valence band perfectly filled, leading to a very stable structure.
        A pure semiconductor crystal lattice (in this case, silicon) can have impurities intentionally introduced to it through the process of doping.
        Doping the lattice with an element possesing only three valence electrons (e.g. Boron) will result in a number of gaps in the valence band (called "holes").
        In such a situation, known as p-type doping, the lattice will accept additional electrons to fill these holes, which will lead to an excess of negatively charged ions.
        Conversly, an element with five valence electrons can be introduced for doping, leading to an excess of electrons in the valence band.
        Known as n-type doping, such an excess results in a lattice with a propensity for shedding these excess valence electrons, which in turn leads and an excess of positive ions.
        A p-n junction can be produced by taking a single silicon wafer and n-type doping one half, while p-type doping the other.
        The junction where the two dopings meet will then see a transfer of excess valence electrons moving from the n-type side to fill the holes of the p-type side, as illustrated in figure %TODO include an illustration of this.
        As the excess "donor" electrons migrate to fill the "acceptor" holes, the area around the junction has its valence band perfectly filled, creating an area called the "depletion zone".
        Though the depletion zone has a filled valence band, it has done so at the cost of ionization; an excess of electrons now populates the p-type side, with an equal number of positive ions remaining on the n-type side.
        The depletion zone grows larger until the migration of holes and electrons is balanced by the electric potential created through this ionization.
        When equilibrium is achieved, the lattice is left with an electric potential which monotonically decreases from the n-type to the p-type side, and which spans the full width of the depletion zone. %TODO illustration of this too?
        If a voltage is the applide across the semiconductor, the width and potential difference of the depletion region can be altered.
        If the voltage is applied with the positive end of the difference at the p-type side, then the semiconductor is said to be "forward biased", and the depletion region will become smaller (and with a high enough voltage can be eliminated entirely). \cite{wiley_radiation_detection}
        If the positive end of the voltage difference is applied to the n-type side though, the semiconductor becomes "reverse biased", and the depletion region and potential difference across the junction will grow larger.
        In this reverse-biased state, the electric potential of the p-n junction becomes very effective at rapidly sweeping excess ions from the depletion region off to the edges of the semiconductor wafer, and it is this mechanism which the ATLAS semiconductor detectors exploit in order to detect particles.

        When ionizing radiation passes through an element of the Pixel Detector or SCT, it will momentarily separate electrons from their nuclei in the silicon lattice.
        Normally, such separated ions would just recombine in a matter of moments.
        Because of the electric potential in the depletion region though, these ions are further seperated, and swiftly arrive at opposite ends of the semiconductor wafer, moving at speeds of about something m/s.%TODO look this speed up in the TDR
        The very leads responsible for biasing the semiconductor are then responsible for collecting these seperated ions, which will cause a sudden jump in the circuit's current.
        The current through these semiconductor detectors is closely monitored, and these spikes are used to identify the passage of a particle through the semiconductors.


        % if there are multiple kinds of similar detector (e.g. pixel vs strip), how do these differ and (ideally) why are both in use?
        % why is the detector where it is
        % have a table describing position of layers, size of components, resolution achieved and what we actually see based on this


    \subsection{Transition Radiation Tracker (TRT)} %TODO
        % purpose
            To provide continuous tracking and aid in the identification of electrons via transition radiation \cite{ID_DTR}.
        % what kinds of particles will they detect,
            Any ionizing radiation
        % what kind of detector tech are they,
            Polyimide drift tube straws using tungsten anode wires, filled with Xe and CO2, interleaved with polypropylene fibres as the transition radiation material \cite{atlas_tdr}.

            The TRT consists of a large number of proportional drift tubes, often referred to as ``straws".
            Proportional drift tubes function in a similar way to semiconductor diode detectors, but using a gas instead of a doped semiconductor.
            The primary component of a drift tube is a cylinder filled with a gas mixture.
            In the TRT, these cylinders are 4 mm in diameter, filled with a mixture of 70\% Xenon, 27\%CO\textsuperscript{2}, and 3\% O\textsuperscript{2}.
            Ions are produced in this mixture when ionizing radiation passes through it.
            As in the semiconductor diode detectors, these ions are collected by applying an electric field through the ionization medium in order to sweep ions away into external circutry for readout.
            In a drift tube this is accomplished by maintaining an electric potential between the tube wall and a conductive wire running through the cylinder axis \cite{drift_chambers}.  %NOTE: do I need to mention avalanche multiplication?
            The wall of the tube acts as the positively charged anode to collect electrons produced by the ionization, while the axial wire is kept at high electrical potential to act as the cathode.
            In the TRT, the cathode wall is made of aluminium and the 30 $\mu$m diameter anode wire of gold-plated tungsten. \cite{trt_design}

            For the TRT barrel, there are 52,544 straws 144 cm in length, while the endcaps each contain 122,880 straws 37 cm in length.
            These straws are arranged parallel to the beampipe in the barrel, and orthogonal to the beampipe in the endcap. %TODO: maybe use atlas_tdr fig 4.2 and 4.3 as illustrations
            This arrangment is to maximize the number of straws traversed by an outgoing particle, typically 35-40 straws for $0 < |\eta| < 2$.
            Though each individual hit has a relatively low spatial resolution compared to the semiconductor detectors, the large number of hits compensates for this with a reduced statistical uncertainty.

            In addition to providing additional tracking information, the TRT provides a secondary purpose of aiding in the identification of electrons by means of detecting transition radiation. 
            Transition radiation is a phenomenon in which radiation is emitted when a charged particle crosses a boundary between two materials \cite{transition_radiation}.
            The TRT uses polypropylene as its transition radiation material, which is interleaved between the layers of straws.
            In the barrel, there are 73 layers of straws interwoven with polypropylene fibres, and in the endcaps there are 160 layers of straws with polypropylene foil between them.
            The large number of layers provide ample and repeated oppurtunity for charged particles crossing the TRT to encounter transition layers and emit identifying radiation.

        % why is the detector where it is
        % have a table describing position of layers, size of components, resolution achieved and what we actually see based on this


\section{Calorimetry} %TODO also include table 1.3 of atlas_tdr
    % Purpose of subsystem
    % Basic specs
    % What mechanism is used to achieve this purpose
    % What are the individual detectors and how do they contribute to this goal

    The ATLAS Calorimeter system is designed to measure the energy of particles emerging from the Interaction Region.
    The entire collection of sub-detectors extends from the immediate outer edge of the Inner Detector region, out to a radius of 4.25 m, and longitudinally out to 6.12 m.
    Together the various systems achieve complete measurment coverage out to $|\eta| < 4.9$.

    There are a number of different ways to measure the energy of a particle, but in ATLAS this is done exclusively using the class of calorimeters known as \textit{sampling} calorimeters.
    Fundamentally, particle calorimeters work by impeding the path of a particle with some material, such that the particle is forced to interact with that material and deposit energy into it.
    This interaction must be such that it produces a measurable response, proportional to the deposited energy.
    A sampling calorimeter functions by sampling the energy deposited.
    Two materials, one which actually measures energy, called the active material, and one which is just there to get in the way, called inactive.
    Inactive material is typically something very dense and heavy.
    The active and inactive materials are arranged in alternating layers, so that the active material gets a snapshot of the way the particle is depositing energy across the entire length of the detector.\cite{energy_measurement}

    The Calorimetry system is split among three different subsystems, designed to measure two distinct kinds of particles.
    The innermost of these subsystems is the Electromagnetic Calorimeter (ECal), designed primarily to measure (anti) electrons and photons \cite{calorimetry_lecture}.
    These particles readily interact electromagnetically in the calorimeter material, rapidly losing energy and permitting the ECal to be more compact.
    Surrounding the ECal are the Hadronic Calorimeter (HCal) systems.
    These detectors, as their name suggests, detect hadrons, such as neutrons or pions.
    Not only are these particles heavier, but many of them are electrically neutral and can therefore only be stopped through repeated nuclear interactions \cite{energy_measurement}.
    Consequently, the HCal systems are significantly thicker and use more dense inactive materials that that of the ECal.
    % atlas_tdr figure 5.2 beautifully illustrates just how much more dense/thick the hcals are than the ecals
    Additionally, they serve the crucial role of preventing hadrons from escaping into the Muon Spectrometer.
    The final subsystem of the Calorimetry system is the Forward Calorimeter (FCal).
    The FCal records, across different sections, the energy of electromagnetic and hadronic particles.
    It is constructed very close to the beampipe, and serves to extend the calorimetric measurement coverage out to an $|\eta|$ of 4.9.


    \subsection{Electromagnetic Calorimeters}
        Purpose is to detect electrons and photons at high resolution.
        Both barrel and encaps use lead as their inactive material, and Liquid Argon as the active material.
        The liquid argon calorimeters, used in the ECal and elsewhere, detects ionization from incident particles via capacitive coupling to electrodes placed outside the active medium.
        Uses a unique "accordian" design in the layout of the material layers, in order to provide complete coverage in $\phi$.
        Also uses a "Presampler", due to an excess of material in front of the ecal.
        Presampler needed to account for energy lost in this material


    \subsection{Hadronic Calorimeter}
        Serves to detect hadrons, and to prevent "punch-through" into the Muon System.
        The Hadronic Endcap Calorimeter (HEC) uses Liquid Argon for the active material, like the ECals, but uses copper plates for the inactive material.
        The HCal Barrel, referred to as the Tile Calorimeter, uses steel plates as the inactive material, and scintillating tiles as the active material.


    \subsection{Forward Calorimeter}
        Purpose is to extend calorimetry coverage as far as $|\eta| < 4.9$.
        Divided into three parts: 1 ECal, and 2 HCals
        All parts use LAr for active medium, but ECal uses copper while Hcals use Tungsten for inactive medium
        Exposed to extremely high radiation flux, necessitating materials that are both very dense and very radiation-resistant \cite{Lar_cal_tdr}.



\section{Muon Spectrometer}  %TODO
    The Muon Spectrometer has the purpose of providing track position and momentum measurements for particles (mosty muons) exiting the ATLAS detector.The Muon barrels start at a radii of 5 m from the beam axis, extending out to 10 m. The endcaps start at a $|z|$ of roughly 7.4 m, and proceed to an extent of 21.5 m. The immense size of the muon system poses a challenge, as it must provide tracking across the entire volume. It's distance from the interaction point ameliorates this issue though, as it permits the Muon detectors to operate at much lower spatial resultions than the inner detectors, while still retaining similar anngular resolution.

    \subsection{Toroid magnets}
        Three large air-core toroids meant to deflect muons.
        Barrel provides 1.5-5.5 Tm (Tesla-meters???) of bending power in $0<|\eta|<1.4$,
        and endcaps provide 1-7.5 Tm in $1.6<|\eta|<2.7$.
        The power is lower in the region where the fields overlap ($1.4<|\eta|<1.6$)

    \subsection{Precision Track Chambers}
        The Precision Track Chambers are designed to provide high resolution measurements of track position and momentum for particles escaping the ATLAS detector. Split across two technologies, the Monitored Drift Tube Chambers (MDT's) and Cathode-Strip Chambers (CSC's).
        The MDT's are drift chambers filled with Ar/CO2 gas using a tungsten-rhenium wire for charge collection. These are used in both the barrel and endcap regions, covering the range $|\eta| < 2.7$. The MDT consists of three endcap and three barrel layers, with a notable exception in the endcap region $2 < |\eta| < 2.7$. Within this $\eta$ range, for the first layer, the muon track density exceeds the resolution capabilities of the MDT's. As such, the first layer of the endcap in this range is replaced with CSC's. The CSC's are muliwire chambers using orthogonal cathode plane strips. These chambers can read both coordinates of a track simultaneosly, preventing the ghosting issues that the MDT would suffer in this higher-flux region.

    \subsection{Trigger Chambers}
        Meant to provide rapid information on muon track multiplicity and energy range.
        Also provides additional track information for higher level triggers.
        Provides acceptance in range $|\eta| < 2.4$.
        The barrel and end-cap use two different technologies in order to address the different issues present in their respective regions.
        The barrel uses Resistive Plate Chambers (RPC's), which have higher temporal resolution. The endcap consists of Thin Gap Chambers (TGC's), designed to withstand higher radiation levels while still delivering high enough time resolution to tag beam-crossings.


% Ch5: Data - DRAFT 4
\chapter{Data Collection and Simulation} \label{chapter:data}

\section{Introduction}
    The amount of data output by the ATLAS detector is immense and overwhelming.
    Reading out every single bunch crossing would require phenomenal bandwidth and would take centuries to process.
    To counteract this overabundance of data, ATLAS relies on an on-site bunch crossing filtering system to drastically reduce its throughput.
    Known as the ATLAS trigger system, this critical piece of infrastructure constitutes the last step of data taking, and the first step of physics analysis.
    Having such a system in place is not by itself sufficient however, without guidance as to how it should be utilized.
    Throwing out events without reason will remove useful events as often as garbage events.
    In order to make informed decisions as to how filtering should be done,
        data is not only collected from ATLAS, but is generated through Monte Carlo simulation processes.
    These simulated data samples provide insight into how different physics processes may present in ATLAS,
        allowing for greater efficiency in what data is kept, and what is removed.


\section{Trigger System}

    The trigger system is a series of hardware and software level algorithms designed to quickly identify bunch crossing ``events'' which may be of interest to physics analysis, while discarding the rest.
    Referred to as ``online'' analysis, these algorithms perform event selection live, in parallel to the ATLAS machine running and taking data.
    The trigger system processes all ATLAS events immediately after readout, ultimately reducing the 40 MHz bunch crossing rate to a data output rate of 1 kHz.
    The data which survives this rapid selection is read out to disk and distributed to individual research teams for more sophisticated ``offline'' analysis later.
    
    Triggering is achieved by running events through two sequential trigger systems.
    All events first go through the hardware-based Level 1 Trigger (L1) before being run through the more sophisticated (and slower) software-based High Level Trigger (HLT).
    Both of these triggers involve a plethora of different measurements on various aspects of the events, such as total transverse energy, transverse momentum, jet multiplicity, and opening angles between jets.
    All of these measurements are factored into whether or not an event passes the trigger selection.

    Each of the various kinematic properties checked by the triggers have multiple threshold values that can determine a ``pass''.
    For example, a jet $p_T$ trigger can have thresholds at 30, 45, or 55 GeV, among others.
    A ``trigger chain'' is a combination of several different such kinematic conditions, each with their own thresholds.
    A bunch crossing is ultimately accepted and read out to disk for further analysis offline if it is able to pass all the conditions of a trigger chain.
    There are hundreds of different trigger chains, each permitting different combinations of kinematics.
    An event is read out if it passes any one of these trigger chains, and is labeled in data with all the trigger chains it passes.
    The trigger chains used in ATLAS were decided upon before the beginning of the Run 2 data taking period, based on input from various analysis teams.
    This predefined list of trigger chains comprise what is known as the ``trigger menu'', and ultimately defines what kinds of physics processes ATLAS analyses have access to.
    The following sections describe broadly how the two levels of the trigger system work, and later chapters wil focus on which trigger menu items are used in the di-Higgs analysis specifically .

    \begin{figure}[h]
        \includegraphics[width=\linewidth,height=\textheight,keepaspectratio]{trigger/facilities}
        \caption{General layout of buildings and facilities at LHC Point 1, site of ATLAS \cite{trigger_tdr}}
        \label{fig:facilities}
    \end{figure}


    \subsection{Level 1 Trigger}

        One bunch-crossing every 25 ns is a blistering pace to operate at.
        The first layer of the trigger system, L1, therefore must run entirely through hardware-level gate logic.
        The goal of this system is to reduce the event rate from the raw 40 MHz bunch-crossing rate, down to a more manageable rate of 100 kHz \cite{trigger_run2}.
        To reduce latency as much as possible, all the electronics comprising L1 are located as close as possible to ATLAS itself, specifically in the USA15 underground chamber \cite{trigger_tdr} (see figure \ref{fig:facilities}).
        As yet another consequence of the high frequency L1 must operate at, it exclusively uses information from the ATLAS calorimeters and Muon Trigger Chamber for its decisions.
        %(utilizes detector buffer memory to keep up).

        The Muon Trigger aspect of L1 is based entirely on the trigger-dedicated Muon Trigger Chamber, described in section \ref{sec:muon-trigger_chamber}.
        Its purpose is to make trigger decisions primarily based on muon $p_T$ and track multiplicity in the Trigger Chamber \cite{trigger_run1}.
        In the calorimeter-based trigger, the selection algorithm first requires a reduction in data resolution.
        The full granularity of the ATLAS calorimeters is too complex to analyze in the 25 ns L1 has to process each bunch-crossing.
        Instead, the various sensors of the calorimeters are clustered together into ``trigger towers'', each with a resolution of 0.1x0.1 in $\Delta \eta \times \Delta \phi$.
        The way towers are clustered and used varies between the different L1 calorimeter trigger modules: the CPM, JEM, and CMX.

        \begin{figure}[h]
            \includegraphics[width=\linewidth,height=\textheight,keepaspectratio]{trigger/trigger_towers}
            \caption{Structure of trigger towers and Regions of Interest \cite{L1_calo_run1}}
            \label{fig:trigger_towers}
        \end{figure}


        The Cluster Processor Module (CPM) exclusively uses the Barrel Calorimeters to function, and is primarily meant for rapid identification of electrons/photons and taus/hadrons.
        For either case, the CPM's first step is to check all possible 4x4 ``windows'' of trigger towers, identifying windows containing an isolated ``Region of Interest'' (RoI).
        Here, an RoI is defined as a 2x2 cluster of towers with an $E_T$ sum that is a relative maximum compared to surrounding towers.
        This 2x2 RoI is the center of the 4x4 window (see figure \ref{fig:trigger_towers}).
        Windows are considered as passing the CPM trigger if the RoI satisfies an isolation requirement, meaning that the 12 towers surrounding that core fall \textit{below} a predefined $E_T$ ``isolation threshold'' value.
        Electrons and photons are then separated from taus and hadrons by the fact that the latter group penetrates into the HCal barrel, while the former group stays highly contained to the ECal.

        Expanding out, the Jet/Energy Processing Module (JEM, or sometimes JEP) makes use of the calorimeter barrels and endcaps, as well as the FCAL, though it does not distinguish between the ECal and HCal.
        The JEM further reduces the granularity under consideration, with a basic unit of data collection being 2x2 collections of trigger towers called ``jet elements'', resulting in a minimum resolution of 0.2x0.2 in $\Delta \eta \times \Delta \phi$.
        Like the CPM, the JEM runs its trigger conditions on windows of multiple jet elements that must be based around a 2x2 RoI core (which is a local $E_T$ maximum).
        Unlike the CPM, these windows can vary in size.
        Primarily, the JEM is intended to perform hit multiplicity counting as well as assist the Extended Cluster Merger Modules (CMX) in carrying out the final jet multiplicity and $E_T$ sums \cite{L1_calo_run1}\cite{trigger_run2}.
        When these conditions, alongside those performed in the Muon trigger, are completed, the event is passed along to the HLT.


\FloatBarrier
    \subsection{High Level Trigger}

        After the L1 Trigger has reduced the event rate to 100 kHz, the software-based High Level Trigger is used to further reduce the event rate to the final output of \textasciitilde 1 kHz.
        Located in the SCX1 building (figure \ref{fig:facilities}) at the surface of P1 (again to minimize latency), the HLT uses data from all detector elements to completely reconstruct the event as it occurred in ATLAS, and performs its selections based on this reconstructed event.
        Different physics processes and particles, known as physics ``signatures'', are reconstructed in different ways, and have different triggers based around them. 
        The main signatures used in ATLAS are: minimum bias signatures, electron/photons (Egamma), muons, jets, taus, missing transverse energy (MET), b-jets (as in jets from bottom quarks), and B-physics (as in B-hadrons).
        The process of reconstruction and the trigger algorithms applied to these reconstructed signatures are based on offline software algorithms which have been repurposed for online use.
        These offline algorithms are quite technical in nature, and are discussed further in Chapter \ref{chapter:reconstruction}.
        However, once events have successfully passed both L1 and the HLT, they are finally distributed off-site for analysis by different physics groups.

        \begin{figure}[h]
            \includegraphics[width=\linewidth,height=\textheight,keepaspectratio]{trigger/data_delivered}
            \caption{Amount of data collected, in terms of integrated luminosity,
                through the Run2 data taking period\cite{data_quality}
                [TODO Steve this plot doesn't make any sense.
                This seems to imply that the trigger hardly throws away any events.
                Is that right?]}
            \label{fig:data_delivered}
        \end{figure}


\FloatBarrier
\section{Monte-Carlo Simulation} \label{sec:mcsim}
    
    Currently, Monte-Carlo simulation is the most effective method of making predictions
        for how theoretical parameters should affect experimental observations in particle physics.
    The process as used for this analysis is a complex one, using an entire suite of programs to achieve a fully simulated output prediction.
    This set of programs, referred to as a \textit{production chain}, consists of three distinct software frameworks.
    Their ultimate goal is to produce a faithful reproduction of how a collection of VBF \to HH processes would appear,
        were they to occur in data collected from ATLAS.
    The chain begins with pure theoretical simulation from MadGraph,
        the output of which is fed to a program called Pythia8,
        which in turn has its output run through the Geant4 simulation framework.
    Output from Geant4 is made to take the same form as if it came from the ATLAS hardware,
        at which point it can be processed in the same manner as real data and studied to make analysis decisions.


    Madgraph is technically a meta-code, a program that creates a program,
        with the created program designed to simulate physics in the manner described by Madgraph.
    To do this, Madgraph needs a theory model, which consists of a lagrangian of the desired physics
        alongside input parameters such as coupling values and particle masses.
    The feynman rules are derived from the given lagrangian,
        which are in turn used to create the process's matrix elements (see Section \ref{sec:feyn_rules}).
    Feynman rules alone are sufficient to then produce tree-level calculations automatically.
    For NLO calculations, additional counterterms must be supplied by hand
        so that matrix element computations appropriately converge during loop integrations. 
    With these features supplied, MadGraph generates code specific to the supplied model,
        which in turn calculates and computes the matrix elements of the requested process\cite{madgraph}.
    A large number of events are produced using Monte Carlo techniques,
        each involving a different configuration of four momenta for the associated particles.
    These events represent a different point in the phase-space of the process,
        and a weight is assigned to each corresponding to its probability as calculated from the matrix element computation.
    The final collection of events is then output for use by later stages of the simulation process.

    [cite TODO good christ the madgraph docs are garbage and now I have to cite a WEB FORUM post by some student who was ALSO commenting on how aweful the documentation for this stuff is... https://answers.launchpad.net/mg5amcnlo/+question/291534]

    Pythia linked to madgraph through ``Les Houches Event'' (LHE) files

    Pythia specialties are isr, fsr, multiparton interactions, fragmentation, and showering

    




    %MadGraph is used for producing simulated particle interactions based on the Feynman diagrams of the process.
    %Using a configuration of particle properties and interaction vertices,
    %    MadGraph is able to construct a tree-level formula akin to Equation \ref{eq:tree_level_invamp}.
    %It does this by using the provided configuration to prepare a list of Feynman rules,
    %    compose all tree-level diagrams that could result from those rules,
    %    and then perform the appropriate matrix element squared expansion.
    %Given a desired initial and final particle state,
    %    MadGraph can produce PDFs for the four-momentum of the final-state particles.
    %A collection of \textit{truth-level} simulated events can then be generated from MadGraph
    %    by sampling from these kinematic PDFs.

    %Although MadGraph does a fine job of generating output particles with kinematics resulting from a production process,
    %    in practice such particles rarely remain in such a state for very long.
    %Most particles, given any appreciable amount of time to propagate, will undergo some form of decay or radioactive emission.
    %Such processes occur randomly, and will often occur repeatedly
    %    (e.g.\ the products of a decay process can themselves undergo further decay).
    %Pythia8 is used to simulate these decay chains.
    %It uses a variety of built-in physics processes to inform the behavior of high-energy collision particle products.
    %This simulation is done entirely in a vacuum though, and one last program is required for full simulation.

    Geant4 is responsible for the final step of event simulation,
        which is to simulate the propagation of particles through the physical hardware of ATLAS,
        and then simulate the detector response to this propagation.
    A full three-dimensional simulation of ATLAS has been constructed within the Geant4 framework,
        modeling the exact dimensions and material composition of every component,
        be it an active detector material or an inactive support structure.
    The particles and showers output by Pythia8 are propagated through the materials of the ATLAS machine by Geant4,
        with further modeling of showering and scattering effects.
    The detectors' response to the simulated particles are recorded, with the detector output stored in the same format as that produced by the ATLAS hardware.



% Ch6: Reconstruction - DRAFT 4
\chapter{Event Reconstruction and Selection}

    Oh god how do I even start with this crap. These two are so tightly linked and there's no obvious way to speparate them AHHHHHH.
    
%    Once a bunch crossing event has cleared the Level 1 Trigger system, the process of event reconstruction begins.
%    An event in ATLAS is, initially, nothing but a collection of electrical signals emitted from the various detectors.
%    Event reconstruction is the process wherein detector readings are aggregated together into meaningful patterns,
%        which are interpreted as physical objects and processes.
%


\section{Object Classification}

    Ok let's just get the basic objects contructed first, since nothing else (triggers, later reconstruction, final selection)
        make sense until I have them.

    This entails ... what? For VBF 4b I think we have:

    Tracks: Helix fitting, I don't think I really know how this works but that's fine I'll figure it out...

    Primary Vertex

    Jets:
        what is a jet.

    Jet flavour tagging.
        b-jets used for higgs,
        light jets used for VBF IS quarks.
    
    ... is that it? I'll look around to see if there's anything else.

    I should explain how online and offline reco differs.
    \cite{cell_clustering}
    \cite{bjet_id_and_performance}
    


    %Reconstruction is performed for an event twice.
    %It is first done rapidly, using coarse measurments and calculations,
    %    for the purpose of allowing the aforementioned HLT to trigger events for readout.
    %These events which pass the trigger are passed off to a massive global computer cluster reffered to as the GRID.
    %No longer bound by the stringent time contrainsts of the ``online'' running environment,
    %    the GRID is able to devote vast amounts of time and computing power to a second, 
    %    more comprehensive, reconstruction of each event.
    %For both the ``online'' (HLT) and ``offline'' (GRID) environments, event reconstruction is carried out by the same suite of software,
    %    called \textit{Athena}.
    





\section{HH4b VBF Analysis}
    
    I think I'm just going to do everything else here? It's weird, because I'm going backwards to triggers,
        and then forward to the final analysis stuff. But at least I have all the complicated object algorithms and stuff listed I guess?

    %Truthfully, reconstruction and selection are not distinct stages of the analysis, but rather are interwoven with each other.
    %The first step of selection really occured before any signficant reconstruction had even taken place, in the L1 Trigger system.
    %Yet the final stage of event reconstruction -- the reconstruction of the di-Higgs system --
    %    will be performed \textit{after} almost every other stage of selection.

    Triggers used in this analysis and why (do we have any plots showing why we use these triggers?)

    Trigger bucketing strategy?

    I need to discuss the kinematics of VBF and 4b in order to justify the ->

    kinematic cuts
        basic jet multiplicity reqs,
        4b-tagged, central
        2 non-btagged jets with min-mjj for VBF

    Then there's the minDR pairing of b's to reconstruct HH

    Then cut on dihiggs and full system kinematics

    Then make sure the higgss fall within the "signal region" (their individual masses are approximately 125 GeV)

    Pretty sure that's it?



%...and how we wittle the abundance of events down to a manageable subset.
%If I stick to this format, there's still a bit of event reconstruction done here.
%
%HH4b Resolved Reco and Selection (literally just go through resolved recon...).
%    Final stage is the signal region selection


% Ch7: Selection - DRAFT 4
\chapter{Event Selection} \label{ch:selection}
    
    I think I'm just going to do everything else here? It's weird, because I'm going backwards to triggers,
        and then forward to the final analysis stuff. But at least I have all the complicated object algorithms and stuff listed I guess?

    %I need to discuss the kinematics of VBF and 4b in order to justify the ->

    %Truthfully, reconstruction and selection are not distinct stages of the analysis, but rather are interwoven with each other.
    %The first step of selection really occured before any signficant reconstruction had even taken place, in the L1 Trigger system.
    %Yet the final stage of event reconstruction -- the reconstruction of the di-Higgs system --
    %    will be performed \textit{after} almost every other stage of selection.


    \section{Triggers}

        \input{tables/selection/nr-triggers-table}

        Triggers used in this analysis and why (do we have any plots showing why we use these triggers?)

        Maybe I should just take that old trigger explaining slide of Mathew's
            and literally just lay out how to interpret what each of these triggers does.

        Trigger bucketing strategy?

    \section{Analysis Cuts} \label{sec:analysis_cuts}

        %Do I need to show proof of all our cut steps? Like, surely there's a motivation for each of them?

        Two VBF Initial Scatter (IS) quark jets:
            light jets (u,d, or maybe charm);
            high pt;
            wide opening angle;
            high mjj;
            can be central or forward.

        Four b-jets:
            decay products of higgss;
            all central; %TODO: why are interesting things always central? can i show this mathematically?
            also high pt;
            b/b-bar products are expected to have very low opening angle between them.

        Overall:
            Zero pt initially, so expected low pt for vector sum of all jets combined
            

        Make note of the fact that the anti-$k_t$ algorithm is set with $\Delta R = 0.4$.

    kinematic cuts
        basic jet multiplicity reqs,
        btag jets with DL1r 77\% working point,
        4b-tagged, central
        also 2b and 3b1f, which are reserved for background estimation
        2 non-btagged jets with min-mjj for VBF

    Then there's the minDR pairing of b's to reconstruct HH

    Then cut on dihiggs and full system kinematics

    Then make sure the higgss fall within the "signal region" (their individual masses are approximately 125 GeV)


%...and how we wittle the abundance of events down to a manageable subset.
%If I stick to this format, there's still a bit of event reconstruction done here.
%
%HH4b Resolved Reco and Selection (literally just go through resolved recon...).
%    Final stage is the signal region selection


% Ch8: Background Estimation - DRAFT 4
\chapter{Background Modelling} \label{chapter:background}

\section{Data-Driven Background Estimate}

    Selection does not remove all non-signal events. (not even close...),
    Indeed, most of the remaining data events are almost certainly not signal events.
    Since these background events cannot be effectively differentiated from data,
        a technique must be used to estimate how much of the oberved data is coming from background.
    Specifically, the analysis employs a method to predict the kinematic shape (in \mhh)
        of the background events within the 4b Signal Region,
        without observing that set of events.

    For the \vbfproc process, the background consists almost entirely of QCD jet events.
    The usual approach of simulating the background using Monte-Carlo techniques is not feasible,
        as such simulations are innacurate when trying to model such complex, many-jet processes.
    Instead, the yield from background in the SR data is predicted based on the background yield in different subsets of the data.
        
    The fundamental assumption for this data-driven background estimate
        is that background events with only two b-tagged jets (2b data)
        will have kinematics very similar to the background events with 4 b-tagged (4b) jets.
    Thus, the kinematic distribution of the 2b signal region can be used as an estimate of the background in the 4b signal region.
    Unfortunately, the kinematics of the 2b and 4b data are not expected to match completely,
        so the 2b signal data must undergo a "reweighting" process,
        which adjusts the kinematic distribution to match that of 4b.
    This reweighting takes the form of a function $R$,
        which for each event $i$ takes a series of kinematic variables $x_i$ as an argmuent,
        and returns a reweight value $r_i$ which scales that event's contribution to kinematic distributions:
        $r_i = R(x_i)$.
    Deriving this function, and determining the appropriate inputs for it, is the an undertaking of its own.
    %The set of variables used for this function are discussed in Section \ref{sec:vbf_bgdNNRW}.

\section{Neural Network Training and Uncertainty}

    The reweighting function is derived using machine learning techniques.
    A neural network (NN) is trained to identify how to reweight 2b data such that the final result looks like 4b data.
    In order to improve stability in the reweighting function,
        an ensemble of 100 networks is trained, and the median of their calculations is used (called the ``Nominal Estimate'').
    Each neural network instance $j$ produces a reweighting factor $w_{ij}$, and has a normalization factor $\alpha_j$ associated with it.
    This normalization is such that the total yield of all 2b reweighted events matches that of the 4b yield in the same region.
    So for $N$ total 2b events and $N'$ total 4b events:
        \begin{equation}
        \sum_{i=1}^{N} \alpha_j w_{ij} = N'
        \end{equation}

    The nominal estimate $\tilde{w}$ constructed from the median of these networks will not neccesarily satisfy this same relation,
        so it is given its own seperate normalization $\tilde \alpha$:
        \begin{equation}
        \sum_{i=1}^{N} \tilde \alpha \tilde w_i = N'
        \end{equation}

    \begin{figure}[!htbp]
        \subfloat[Data before reweighting]{
            \includegraphics[width=0.5\linewidth,height=\textheight,keepaspectratio]{background/NR-MAY21-crypto-m_hh-control-no_rw-allyr-3b1f-dEtahhcat_inclusive}
        }
        \subfloat[Data after reweighting]{
            \includegraphics[width=0.5\linewidth,height=\textheight,keepaspectratio]{background/NR-MAY21-crypto-m_hh-control-NN-allyr-3b1f-dEtahhcat_inclusive}
        }
        \caption{
            The \mhh kinematic distribution of the 2b VS 4b regions, before and after reweighting the 2b region.
        }
        \label{fig:data_mhh_reweight}
    \end{figure}

    The problem with this training approach, is that the 4b signal region cannot be used to train against.
    Instead, the Neural networks are trained using the Control Region data,
        and are optimized to reweight the 2b CR data to resemble the 4b CR data.
    Because the reweighting function is trained on a different kinematic region than the signal region,
        a systematic uncertainty is associated with it.
    This is calculated by training a different ensemble of NNs on yet another kinematic region (Validation).
    The difference between the median of these validation-trained NNs and the nominal estimate
        produces a ``shape uncertainty'' for the reweighting.
    The shape uncertainty is symmetrized around the nominal estimate to produce the final systematic uncertainty
        associated with the reweighting.
    A comprehensive error for the background estimate is produced by combing the shape uncertainty with the statistical error
        associated with the limited sample size of the training data.


\section{VBF Background Reweighting Variables} \label{sec:vbf_bgdNNRW}

    The Neural Network used to reweight the 2b-Tagged Control-Region data into the 4b Signal region,
        uses a different set of variables when trained for the VBF process than it does for ggF.
    This list of variables was selected by first producing a set of plots showing how strongly correlated different variables are to \mhh
        (Figure \ref{fig:mhh_corr}).

    \begin{figure}[!htbp]
        \subfloat[$\deta \leq 1.5$]{
            \includegraphics[width=0.5\linewidth,height=\textheight,keepaspectratio,ext=.pdf,type=pdf]{background/correlation_control2b_detahh-LTE1.5_hist_m_hh.pdf}
        }
        \subfloat[$\deta > 1.5$]{
            \includegraphics[width=0.5\linewidth,height=\textheight,keepaspectratio,ext=.pdf,type=pdf]{background/correlation_control2b_detahh-GT1.5_hist_m_hh.pdf}
        }
        \caption{
            The Pearson Correlation Coefficients associated with \mhh for the 2b Control Region data.
            The regions in which $\deta \leq 1.5$ and $\deta>1.5$ are separately displayed.
            The higher up on the chart a variable is, the more strongly correlated it is to \mhh,
                with \mhh itself at the top with a coefficient of 1.
            Note that most of the VBF-specific variables, such as vbf\_mjj, are very poorly correlated to \mhh and thus not preffered.
        }
        \label{fig:mhh_corr}
    \end{figure}

    Based on their high correlation to \mhh in both \deta regions, 23 variables were then selected for further study.
    These 23 variables were plotted against each other in a correlation matrix.
    A final set of seven variables was selected based (Table \ref{tab:vbf_NNRW_vars}) on which variables were \textit{least} correlated to each other,
        in order to avoid variables carrying redundant information (Figure \ref{fig:vbf_corr_matrix}).

    \begin{figure}[!htbp]
        \subfloat[$\deta \leq 1.5$]{
            \includegraphics[width=0.5\linewidth,height=\textheight,keepaspectratio,ext=.pdf,type=pdf]{background/correlation_control2b_detahh-LTE1.5_matrix.pdf}
        }
        \subfloat[$\deta > 1.5$]{
            \includegraphics[width=0.5\linewidth,height=\textheight,keepaspectratio,ext=.pdf,type=pdf]{background/correlation_control2b_detahh-GT1.5_matrix.pdf}
        }
        \caption{
            The Pearson Correlation Coefficients matrix associated with variables in the 2b Control Region data.
            The regions in which $\deta \leq 1.5$ and $\deta>1.5$ are separately displayed.
            Bins with more saturated (i.e.\ more blue or more red) colors indicate strong correlation between those two variables.
            All variables are of course 100\% correlated to themselves, hence the deep red line running down the diagonal.
            The most highly preffered variables are those with correlation coefficients near 0 for as many other variables as possible.
            If a variable (i.e.\ m\_max\_dj) is selected, any other variables with strong correlations to it (i.e.\ pT\_h2)
                are then strongly disfavored for later selection.
        }
        \label{fig:vbf_corr_matrix}
    \end{figure}


    \begin{table}[!htbp] \centering \footnotesize
    \caption{Final Set of Neural Network Variables}
    \label{tab:vbf_NNRW_vars}
    \begin{tabular}{ |l|l|l| }
        \hline
        \textbf {Variable} & \textbf {Internal Name} & \textbf {Description} \\
        \hline
        $M_{max \Delta j}$ & \code{m\_max\_dj}         & 
            Take the di-jet mass of the six possible pairings\\
            && of the four higgs’ candidate jets;\\
            && this is the maximum di-jet mass of those pairings \\ 
        \hline
        $M_{min \Delta j}$ & \code{m\_min\_dj}         & 
            As above, but the minimum \\
        \hline
        $E_{H1}$           & \code{E\_h1}              & 
            Energy of the leading-$p_T$ reconstructed Higgs \\
        \hline
        $E_{H2}$           & \code{E\_h2}              & 
            Energy of the sub-leading-$p_T$ reconstructed Higgs \\
        \hline
        Xwt-tag            & \code{X\_wt\_tag}         & 
            $\log\left(X_{Wt}\right)$, where $X_{Wt}$ is the variable used for the top veto \\
        \hline
        $\eta_i$           & \code{eta\_i}             & 
            Average $\eta$ of the the four Higgs decay jets \\
        \hline
        Pairing Score 2    & \code{pairing\_score\_2 } & 
            This changes depending on the Higgs pairing algorithm used. \\
            &&The current algorithm used is minDR. \\
            &&With four jets, there are three possible pairings of the jets. \\
            &&The pairings are ranked by the $\Delta_R$ of the leading-$p_T$ Higgs candidate. \\
            &&Pairing Score 2 is the $\Delta-R$ of the second-smallest of these pairings. \\
        \hline
    \end{tabular} \end{table}



%\section{Statistical Error} %TODO I don't even want to bother with this until the group figures out what it's doing
%
%    Now to talk about statistical error in the Neural network estimate
%
%    Poisson stat errors: stat error on bin i is 
%        \begin{equation}
%        \sigma_i = \sqrt{\sum_{j \in i} w_j^2}
%        \end{equation}
%
%    Bootstrap uncertainty: A bin-by-bin statistical error resulting from the variations in each of the networks' reweighting functions.
%    Naively, this would mean using the average of all networks, with an error calculated as the standard deviation of all histograms of all the networks, for a given variable of interest.
%    Slightly less naively (to make us more robust to outliers),
%        we would use the median of all networks, and the IQR values of the networks as the error.
%    In reality, it's impractical to retain the information of all networks for all histograms.
%    We can still use the median value, but the error has to be calculated another way.
%    Specifically, the error is calculated by storing a small amount of extra information regarding the IQR of the bootstraps.
%    Whenever a kinematic distribution is produced, \textbf{two} histograms are generated:
%        the nominal histogram, and the \textit{varied} histogram.
%    The difference between the values of the nominal and varied histograms, per bin,
%        are taken as the error on that bin for the nominal histogram.
%
%    The nominal event weights are calculated as
%        \begin{equation}
%        \tilde w_i = \textrm{median}(\alpha_1 w_{1,i}, ..., \alpha_{100} w_{100,i})
%        \end{equation}
%        with the normalization $\tilde \alpha$ calculated as described above.
%    So if the yield of normalized nominal weights is denoted
%        \begin{equation}
%        \tilde Y \equiv \sum_{i=1}^{N} \tilde \alpha \tilde w_i = \textrm{yield of 4b control region}
%        \end{equation}
%    then the values of the bins of the nominal histogram, $H_j$ are:
%        \begin{equation}
%        H_j = \tilde \alpha \sum_{i \in j} \tilde w_i
%        \end{equation}
%
%    A "weight variation" factor is calculated per event as 
%        \begin{equation}
%        V^w_i = \frac{1}{2}\textrm{IQR}(w_{1,i}, w_{2,i}, ..., w_{100,i})
%        \end{equation}
%    Likewise, a "normalization variation" factor is calculated as
%        \begin{equation}
%        V^{\alpha} = \frac{1}{2}\textrm{IQR}(\alpha_{1}, \alpha_{2}, ..., \alpha_{100})
%        \end{equation}
%    The sum of the varied weights is 
%        \begin{equation}
%        Y_v = \sum_{i=1}^{N} \left( \tilde w_i + V^w_i \right)
%        \end{equation}
%    And the ratio of the nominal yield to the varied yield is $R_Y \equiv \tilde Y / Y_v$
%
%    The "varied" histogram $H'$ is then calculated based on these variation factors as:
%        \begin{equation} \begin{split}
%        H'_j &= \left[ R_Y \sum_{i \in j} \tilde w_i + V^w_i \right] + V^{\alpha} H_j
%        \\H'_j &= \sum_{i \in j} \left[ R_Y \left( \tilde w_i + V^w_i \right)
%            + V^{\alpha} \tilde \alpha \tilde w_i \right]
%        \end{split} \end{equation}
%
%    The error of the nominal histogram bin $j$ then is
%        \begin{equation} \begin{split}
%        \sigma_j &= | H'_j - H_j |
%        \\&= \left| \sum_{i \in j} \left[
%            R_Y \left( \tilde w_i + V^w_i \right)
%            + V^{\alpha} \tilde \alpha \tilde w_i \right] 
%            -\tilde \alpha \sum_{i \in j} \tilde w_i \right|
%        \\&= \left| \sum_{i \in j} \left[
%            R_Y \left( \tilde w_i + V^w_i \right)
%            + V^{\alpha} \tilde \alpha \tilde w_i 
%            - \tilde \alpha \tilde w_i
%            \right] \right|
%        \\&= \left| \sum_{i \in j} \left[
%            R_Y \left( \tilde w_i + V^w_i \right) 
%            + \tilde \alpha \tilde w_i (V^{\alpha}-1) \right] \right|
%        \end{split} \end{equation}
%


% Ch9: Signal Modeling - DRAFT 5
\chapter{Signal Modeling}

How do we produce a signal hypothesis that we can test?
Mostly just a bunch of software.

\section{Monte-Carlo Simulation}

    Need to cover MadGraph, Pythia, and Geant4.

\section{Signal Combination}

    The cross-section and kinematic distributions of the di-Higgs production processes depend fundamentally on a number of coupling constants between the Higgs and other particles.
    Of particular interest in this analysis are \kl for the Higg's self-coupling and \kvv for the \HHVV coupling.
    The values of \kl and \kvv have loose experimental constraints \cite{EXOT-2016-31} \cite{HDBS-2018-18-witherratum} \cite{ATLAS-CONF-2019-049}, requiring analysis across a wide range of these coupling values.
    Unfortunately, MC generation is computationally expensive and time consuming.
    As such, only a handful of MC simulation samples for only a handful of coupling values are actually produced, and a sample combination technique is employed to model the signal hypothesis across the coupling parameter space.

    The process of combining a few samples in such a way as to model the entire parameter space of coupling constants is based on exploiting the underlying mathematics of the differential cross-section formula.
    By expanding the squared term of the sum of all Feynman diagram contributions, the cross-section can be expressed as a function of its coupling values \cite{ATLAS-CONF-2019-049}.
    In the case of VBF the differential cross-section depends on the three diagrams of Figure \ref{fig:tree_level_vbfhh} as:

    %\begin{equation} \label{eqn:ggf_hh_feynSum}
    %\frac{d\sigma(\kl, \kt)}{d \mhh}  = | A(\kt, \kl) |^2 = | \kl \kt M_{\bigtriangleup}(\mhh) + \kt^2 M_{\Box}(\mhh) |^2
    %\end{equation}

    %\begin{equation} \label{eqn:ggf_hh_feynExpand}
    %    = \kl^2 \kt^2 |M_{\bigtriangleup}(\mhh)|^2 +
    %    \kl \kt^3 [ M_{\bigtriangleup}^*(\mhh) M_{\Box}(\mhh) + M_{\Box}^*(\mhh) M_{\bigtriangleup}(\mhh) ] +
    %    \kt^4 |M_{\Box}|^2
    %\end{equation}

    %\begin{equation} \label{eqn:ggf_hh_feynSimple}
    %    = \kl^2 \kt^2 a_1(\mhh) + \kl \kt^3 a_2(\mhh) + \kt^4 a_3(\mhh)
    %\end{equation}


    The $a_i$ matrix element expansion values have a dependence on \mhh, which is not trivially derivable as an analytic function.
    Instead, for a given \kl, its cross-section in each \mhh bin can be mathematically determined by solving a set of linear equations for $a_1$, $a_2$, and $a_3$,
        using three different cross-section values (in the same \mhh bin) for three different \kl values.
    In practice, the cross-section values of these three \textit{basis} samples are represented by the yields from MC simulation, binned by their true \mhh.
    Once the $a_i$ values are obtained, event weights for every \kl values can be derived for one of the simulated samples (e.g.\ the SM sample),
        by taking the ratio between the true \mhh distributions of the target coupling and the SM.
    By applying these weights to the SM sample, one obtains a reweighted sample for the targeted \kl value.

    In order to obtain a reweighted distribution at reconstruction-level, the truth \mhh value is stored in the SM sample, and the weights are applied to the events remaining after reconstruction and selection.
    In this approach, an assumption is made that the acceptance varies approximately as a function of only \mhh.
    This assumption is validated on a fourth sample by comparing the distributions from the reweighted sample and a MC sample generated at some other point in \kl.

    The VBF HH process is modelled in a similar manner to the ggF process, but with some key differences.
    Most notably, the sample combination for VBF is not done at truth level.
    The kinematics of the VBF process involve both the Higgs pair and the VBF initial scatter jets.
    As such, the acceptance cannot be adequately described by only one variable (e.g.\ \mhh), rendering the truth re-weighting approach untenable.
    Instead, all of the basis VBF MC samples are run through reconstruction and selection.
    The signal distribution modelling is then performed by directly combining the reconstructed \mhh distributions of post-selection signal samples.

    The VBF to HH process involves 3 diagrams (Figure \ref{fig:tree_level_vbfhh}). 
    The squared-sum matrix element expansion then produces six terms, which are a function of three couplings (\kvv, \kl, \kv).
    The full modelling equation for VBF thus requires the combination of six different samples (see this section after I copy in the appendix).
    This particular combination of samples was chosen first, based on the samples available and, second, by an optimization procedure described in right here TODO.

    \begin{equation}
    \label{eqn:vbf_hh_6term_chosen}
    \begin{split}
        \frac{d\sigma}{d\mhh}(\kvv, \kl, \kv) =
            \left(\frac{68 \kappa_{2V}^{2}}{135} - 4 \kappa_{2V} \kappa_{V}^{2} + \frac{20 \kappa_{2V} \kappa_{V} \kappa_{\lambda}}{27} + \frac{772 \kappa_{V}^{4}}{135} - \frac{56 \kappa_{V}^{3} \kappa_{\lambda}}{27} + \frac{\kappa_{V}^{2} \kappa_{\lambda}^{2}}{9}\right) \times \frac{d\sigma}{d\mhh}{\left(1,1,1 \right)} \\
            + \left(- \frac{4 \kappa_{2V}^{2}}{5} + 4 \kappa_{2V} \kappa_{V}^{2} - \frac{16 \kappa_{V}^{4}}{5}\right) \times \frac{d\sigma}{d\mhh}{\left(\frac{3}{2},1,1 \right)} \\
            + \left(\frac{11 \kappa_{2V}^{2}}{60} + \frac{\kappa_{2V} \kappa_{V}^{2}}{3} - \frac{19 \kappa_{2V} \kappa_{V} \kappa_{\lambda}}{24} - \frac{53 \kappa_{V}^{4}}{30} + \frac{13 \kappa_{V}^{3} \kappa_{\lambda}}{6} - \frac{\kappa_{V}^{2} \kappa_{\lambda}^{2}}{8}\right) \times \frac{d\sigma}{d\mhh}{\left(1,2,1 \right)} \\
            + \left(- \frac{11 \kappa_{2V}^{2}}{540} + \frac{11 \kappa_{2V} \kappa_{V} \kappa_{\lambda}}{216} + \frac{13 \kappa_{V}^{4}}{270} - \frac{5 \kappa_{V}^{3} \kappa_{\lambda}}{54} + \frac{\kappa_{V}^{2} \kappa_{\lambda}^{2}}{72}\right) \times \frac{d\sigma}{d\mhh}{\left(1,10,1 \right)}  \\
            + \left(\frac{88 \kappa_{2V}^{2}}{45} - \frac{16 \kappa_{2V} \kappa_{V}^{2}}{3} + \frac{4 \kappa_{2V} \kappa_{V} \kappa_{\lambda}}{9} + \frac{152 \kappa_{V}^{4}}{45} - \frac{4 \kappa_{V}^{3} \kappa_{\lambda}}{9}\right) \times \frac{d\sigma}{d\mhh}{\left(1,1,\frac{1}{2} \right)} \\
            + \left(\frac{8 \kappa_{2V}^{2}}{45} - \frac{4 \kappa_{2V} \kappa_{V} \kappa_{\lambda}}{9} - \frac{8 \kappa_{V}^{4}}{45} + \frac{4 \kappa_{V}^{3} \kappa_{\lambda}}{9}\right) \times \frac{d\sigma}{d\mhh}{\left(1,-5,\frac{1}{2} \right)}
    \end{split}
    \end{equation}

    Discussion here will focus on the individual one-dimensional \kvv and \kl scans.
    Note that although \ref{eqn:vbf_hh_6term_chosen} involves six samples, the equation simplifies down to only three terms (and thus needs only three samples to combine) when \kvv and \kv are set to the Standard Model value of 1.
    In theory the same can be done to produce a 1D equation for \kvv (by setting \kl and \kv to 1), though this particular basis does not permit such a simplification.
    For \kl, the simplified three-term equation takes the form

    \begin{equation}
    \label{eqn:vbf_hh_3term_kl}
    \begin{split}
        \frac{d\sigma}{d\mhh}(\kvv=1, \kl, \kv=1) =
        \left(\frac{\kappa_{\lambda}^{2}}{9} - \frac{4 \kappa_{\lambda}}{3} + \frac{20}{9}\right) \times
            \frac{d\sigma}{d\mhh}{\left(1,1,1 \right)} \\
        \left(- \frac{\kappa_{\lambda}^{2}}{8} + \frac{11 \kappa_{\lambda}}{8} - \frac{5}{4}\right) \times
            \frac{d\sigma}{d\mhh}{\left(1,2,1 \right)} \\
        \left(\frac{\kappa_{\lambda}^{2}}{72} - \frac{\kappa_{\lambda}}{24} + \frac{1}{36}\right) \times
            \frac{d\sigma}{d\mhh}{\left(1,10,1 \right)}
    \end{split}
    \end{equation}



    This procedure has been successfully validated in its ability to accurately model a large range of coupling values (see \ref{fig:vbf_hh_validation}) and shown to produce reasonable distributions well beyond the Standard Model coupling values (\ref{fig:vbf_hh_preview}).

    \begin{figure}
    	\centering
        \subfloat[Validation against MC with $\kl=0$]{
            \includegraphics[width=0.32\linewidth,height=\textheight,keepaspectratio]{signal/reco_mHH_cvv1p00cl0p00cv1p00}
        }
        \subfloat[Validation against MC with $\kvv=0$]{
            \includegraphics[width=0.32\linewidth,height=\textheight,keepaspectratio]{signal/reco_mHH_cvv0p00cl1p00cv1p00}
        }
        \subfloat[Validation against MC with $\kvv=3$]{
            \includegraphics[width=0.32\linewidth,height=\textheight,keepaspectratio]{signal/reco_mHH_cvv3p00cl1p00cv1p00}
        }
        \caption{
            The six-term linear combination of samples is combined for various coupling values.
            The combined distribution shape (in blue) is compared against a Monte-Carlo sample (in green) which was generated for the same coupling values.
            The combination approach shows good agreement with the generated sample, indicating accurate modelling of the signal shape.
        }
        \label{fig:vbf_hh_validation}
    \end{figure}



    \begin{figure}
    	\centering
        \subfloat[Combination at  \kvv = -2]{
            \includegraphics[width=0.44\linewidth,height=\textheight,keepaspectratio]{signal/preview_reco_mHH_new_cvv-2p00cl1p00cv1p00}
        }
        \subfloat[Combination at  \kl = -9]{
            \includegraphics[width=0.44\linewidth,height=\textheight,keepaspectratio]{signal/preview_reco_mHH_new_cvv1p00cl-9p00cv1p00}
        }
        \caption{
            The six-term linear combination of samples is combined for coupling values very far from the Standard Model.
            There are no MC simulated samples to compare these points too, but the combined distributions at these points are smooth and well-behaved,
                indicating reasonable modelling of the distribution even at distant couplings.
        }
        \label{fig:vbf_hh_preview}
    \end{figure}


\section{VBF Full 3D Combination} \label{app:vbf3dcombination}

    The full cross section for the VBF to HH process involves three diagrams,

    \begin{equation}
    \label{eqn:vbf_hh_6term_amp}
    \sigma(\kvv, \kl, \kv) = |A|^2 = | \kv \kl M_s + \kv^2 M_t + \kvv M_x |^2
    \end{equation}

    Expanding the absolute square of the amplitude $A$ then yields six terms:

    \begin{equation}
    \label{eqn:vbf_hh_6term_generic}
    \sigma = \kv^2 \kl^2 a_1 + \kv^4 a_2 + \kvv^2 a_3 + \kv^3 \kl a_4 + \kv \kl \kvv a_5 + \kv^2 \kvv a_6
    \end{equation}

    As discussed earlier, this requires the combination of six different Monte-Carlo samples in order to model the signal hypothesis at any arbitrary point in \kvv, \kl, \kv space.
    In theory, with infinite events, the only requirement for these samples is that they are linearly independent of each other.
    In practice, the final samples are statistically limited, and different combinations of variations yield different statistical power.
    The 6-sample combination used in this analysis (\ref{tab:vbf_hh_6term_varlist} and \ref{eqn:vbf_hh_6term_chosen}) has been chosen specifically for its ability to avoid mismodelling errors.
    The method used to determine the overall performance of a basis is to check the number of negative bins generated
        in the \mhh distribution across all points in the two-dimensional \kvv,\kl space, as seen in \ref{fig:vbf_hh_6term_nWeight_grid}.
    As negative bin weights are unphysical, they indicate poor signal modelling.
    Thus, identifying a basis which minimizes the presence of negative weights ensures stable modeling of the signal hypothesis at all points in the coupling space (\ref{fig:vbf_hh_6term_validation} and \ref{fig:vbf_hh_6term_preview}).

    \begin{table}[] \centering
    \caption{6-Term VBF Combination Sample Variations}
    \label{tab:vbf_hh_6term_varlist}
    \begin{tabular}{ |l|l|l| }
        \hline
        \textbf {$\kappa_{2V}$} & \textbf {$\kappa_\lambda$} & \textbf {$\kappa_V$} \\
        \hline
            1   &   1 & 1   \\
            1.5 &   1 & 1   \\
            1   &   2 & 1   \\
            1   &  10 & 1   \\
            1   &   1 & 0.5 \\
            0   &  -5 & 0.5 \\
        \hline
    \end{tabular} \end{table}

    \begin{figure}
        \includegraphics[width=\linewidth,height=\textheight,keepaspectratio]{signal/negative_weights_base}
        \caption{
            Frequency of negative bin weights in \mhh distribution across \kvv,\kl range.
            Brighter regions indicate more negative-weighted bins, and suggest less stable signal modelling.
            This particular combination of samples was chosen for how much of the space is ``dark",
                with darker regions indicating generally stable signal modelling.
            The table of coupling values in the upper right corner indicates the 6 MC samples
                (highlighted on the plot with cyan dots) used in the combination.
        }
        \label{fig:vbf_hh_6term_nWeight_grid}
    \end{figure}


    \begin{figure}
        \subfloat[Validation \kv = 1.5]{
            \includegraphics[width=0.5\linewidth,height=\textheight,keepaspectratio]{signal/reco_mHH_cvv1p00cl1p00cv1p50}
        }
        \subfloat[Validation \kl = 0]{
            \includegraphics[width=0.5\linewidth,height=\textheight,keepaspectratio]{signal/reco_mHH_cvv1p00cl0p00cv1p00}
        }
        \caption{
            Validation of 6-term combination against MC generated at \kv = 1.5 and \kl = 0.
            The combination shows good agreement to the generated MC distributions.
        }
        \label{fig:vbf_hh_6term_validation}
    \end{figure}



    \begin{figure}
        \subfloat[Combined \mhh distribution at \kvv = 2.5, \kl = -10]{
            \includegraphics[width=0.5\linewidth,height=\textheight,keepaspectratio]{signal/preview_reco_mHH_new_cvv2p50cl-10p00cv1p00}
        }
        \subfloat[Combined \mhh distribution at \kvv = 2.0, \kl = -10]{
            \includegraphics[width=0.5\linewidth,height=\textheight,keepaspectratio]{signal/preview_reco_mHH_new_cvv2p00cl-10p00cv1p00}
        }\\
        \subfloat[Combined \mhh distribution at \kvv = 0, \kl = 13]{
            \includegraphics[width=0.5\linewidth,height=\textheight,keepaspectratio]{signal/preview_reco_mHH_new_cvv0p00cl13p00cv1p00}
        }
        \caption{
            \mhh distribution produced by the 6-term combination at points far from the SM.
            The combination produces smooth, well-behaved distributions at these points,
                suggesting the signal is well-modelled in these regions.
        }
        \label{fig:vbf_hh_6term_preview}
    \end{figure}



\section{Solidarity}
    
    Should this go here?


% Ch10: Results - DRAFT 3
\chapter{Results} \label{chapter:results}

%Need to discuss how we set put data, the background estimate, and the signal model together
%    and make a claim as to the compatibility of the hypothesis with the data.
%Largely, this means I need to finally understand how pyhf actually works and what the hell the limit framework is doing.
%
%
%
%    I'm not sure I actually have to explain Baye's theorem here...
%    it doesn't appear to be obviously used, or if it is, we're implicitly using the "Uniform" Prior
%    Ask Steve
%
%    L gives probability of seeing the data we have, based on the given model.
%    We need the probability that the given model is the one responsible for the data we see.
%    i.e. we have P(data|model), but we need P(model|data)
%
%    To do this, we need Baye's rule, which comes from the basic 'anding' of probabilities:
%    P(a \& b) = P(a)*P(b|a) , or P(a \& b) = P(b)*P(a|b) . Thus
%    P(a)*P(b|a) = P(b)*P(a|b) 
%    P(b|a) = P(a|b) P(b) / P(a)
%    So we need P(model|data) = P(data|model) * P(model) / P(data)
%
%    For our data we have to use the extended L, which accounts for poisson stats.
%    Basic test is to test mu*S+B for what value of mu is compatible with data
%
%        
%
%

%\section{Statistical Mathematics}
%
%    Statistics is a powerful tool in science,
%        but one which can be very misleading if not used carefully.
%    The oft-used quote comparing lies and statistics exists for a reason;
%        both can lead to incorrect scientific conclusions.
%    But whereas a lie can be caught by a simple slip of the tongue,
%        it can take scientists years to discover a slight (intentional or not) mishandling of statistics.
%    Unfortunately, the use of statistical methods is not optional.
%    In this section, I want to explain why statistics are required for this analysis,
%        as well as describe the basic mathematics and techniques utilized to obtain results.
%
%    To begin, let me propose a far simpler experiment than the one described in this analysis.
%    I have a coin, which I suspect may be weighted to one side.
%    How can I test this?
%    basic binomial distribution (coin flip) allows obvious p-test. 
%    Show of whether or not theory is compatible with data.
%    ?? Can show how to set basic limits in absence of enough data


What is a test statistic and why do we need it?
%    I have a single 6-sided die.
%    I suspect that this die is not a fair die,
%        and that it is actually weighted to land with the ``4'' side facing up more often than other sides.
%    How can I test this?

Description of how "test statistic" q~ is constructed:

    Likelihood function

        L = product[ for each category:
            product[for each bin: poissons]
            * product[ nuissance params ] 
        ]

        
    lambda = L(mu, theta vary-opt) / L(mu-opt, theta-opt);
    t = -2*ln(lambda);
    q~ = t with edge cases;

%\section{VBF \to HH \to 4b Limit-Setting Framework}
%
%    Discuss sources of error, assumptions, categorization, concept of mu values;
%    basically all the complicated stuff the limit framework is doing
%
Discussion of how the error is calculated for the shape is a particular point here...

Discuss how this formula for L specifically functions in this analysis
    (emphasis on explaining the bits in parentheses):

        L = product[ for each category (2: eta hi and lo) 
            product[for each bin: poissons]
            * product[ nuissance params (4 for bgd shape error, 1 for norm error) ] 
        ]

    
%\section{Final Interpretation}
%
%    Finally, I need to reveal what our final results actually are, and what they mean.
%    This will mostly just be plots of mu values, limit values, and my 2D exclusion plots.
%    I can discuss the shapes and stuff here, as well as talk about any mismatches between expected and observed results.
%    Be ready to just put in filler results, since we probably won't have unblinded by the time I get here.
%

Single mu vs q~ PDF constructed from monte-carlo distros

C-PDF showing where mu-model resides for Psb, Pb, and finally Ps=Psb/(1-Pb)
CLs = CLs+b/CLb explained in \cite{Barlow:2019svl} (pg. 192)

In practice, the q~ CPDF distros are estimated using an asymptotic approximation method\cite{asymptotic_formulae_for_likelihood}.
(because the MC method is way too slow)

SM mu scan plot of signal p-value.
If you do this using one of the other stats (t, or t~ or something),
    I think you can use it to show that pure discovery of the HH process is impossible right now,
    and therefore justify putting limits on the couplings instead.
Should also probably dig up the "sensitivity" metric and show how bad that is here as well.

multi k2v-scan mu plots

1D SM k2v plot

Multi-kl 1D k2v plots

2D k2v/kl plot

Multi-dimensional slice plots


% Ch11: Conclusion and Future Studies - DRAFT 2
\chapter{Conclusion}\label{chapter:conclusion}

%Summarize thesis
This thesis used 126 $\textit{fb}^{-1}$ of data from Run 2 of ATLAS
    to perform a search of the \hhproc process decaying to 4 bottom quarks,
    in order to better constrain the \kl and \kvv Higgs self-coupling constants.
The analysis made extensive use of the b-jet trigger system to identify this final state,
    alongside a b-jet pairing algorithm capable of accurately reconstructing the di-Higgs system.
An efficient selection algorithm was also employed to remove a majority of background events,
    though a sizable background still remained.
This multi-jet and \ttbar background was estimated using a sophisticated data-driven approach,
    which takes advantage of machine-learning algorithms specifically designed around the VBF environment.
The di-Higgs signal itself was then modeled using six Monte-Carlo simulations.
These six models were extended across the entirety of the possible \kvv/\kl/\kvv coupling parameter space
    through a linear combination technique
    which operated by reverse-engineering the fundamental QFT cross-section equation for the HH process.
Although this method initially met with severe instabilities in its modeling performance,
    I was able to derive a novel optimization procedure able to specify
    the exact Monte-Carlo simulation needed to stabilize the combination.
This additional MC sample was then used to provide an accurate hypothesis description across
    the three-dimensional \kvv/\kl/\kv parameter space,
    and so establish exclusion limits across these coupling values. 

%signal
Though the di-Higgs process itself is still too elusive for discovery with the amount of data collected so far,
    significantly tighter constraints can be placed on the \kvv and \kl coupling constants.
Indeed, for the first time constraints have been placed on the full
    3D coupling space observable via the VBF production mode of the di-Higgs process.
Courtesy of the MC sample linear combination technique,
    \kl has been constrained to between $ -19 < \kl < 22$ $( -24 < \kl < 27)$ for all values of \kvv with a fixed \kv=1,
    and between $-9.65 < \kl < 12.9$ $(-12.84 < \kl < 15.94)$ for $\kvv=\kv=1$.
Likewise, the extrema of \kvv have been established between $-0.75 < \kvv < 2.75$ $(-1 < \kvv < 3)$
    if \kl is permitted to fluctuate beyond its SM value while \kv is fixed to 1,
    and fixing both \kl and \kv to 1 produces limits of $0.09 < \kvv < 1.99$, $(-0.08 < \kvv < 2.16)$.

\begin{figure}
    \centering
    \begin{subfigure}{0.48\textwidth} 
        \includegraphics[width=\linewidth,height=\textheight,keepaspectratio]{conclusion/mu_limits_fast_HL-LHC_k2v}
        \caption{}%$\mu$ Scan Plot for \kvv = 1}
        \label{fig:mulimits_kvv_future}
    \end{subfigure}
    \begin{subfigure}{0.48\textwidth}
        \includegraphics[width=\linewidth,height=\textheight,keepaspectratio]{conclusion/mu_limits_fast_HL-LHC_kl}
        \caption{}%$\mu$ Scan Plot for \kvv = 3}
        \label{fig:mulimits_kl_future}
    \end{subfigure}\\
    \begin{subfigure}{0.48\textwidth} 
        \includegraphics[width=\linewidth,height=\textheight,keepaspectratio]{conclusion/xsec_limits_fast_HL-LHC_k2v}
        \caption{}%$\mu$ Scan Plot for \kvv = 1}
        \label{fig:xseclimits_kvv_future}
    \end{subfigure}
    \begin{subfigure}{0.48\textwidth}
        \includegraphics[width=\linewidth,height=\textheight,keepaspectratio]{conclusion/xsec_limits_fast_HL-LHC_kl}
        \caption{}%$\mu$ Scan Plot for \kvv = 3}
        \label{fig:xseclimits_kl_future}
    \end{subfigure}
    \caption{
        Figures of scans across the \kvv and \kl scaling factors
            for a hypothetical 3000 \ifb of data from the HL-LHC.
    }
    \label{fig:mu_xsec_future}
\end{figure}

%Emphasize results
Looking towards future studies, prospects for ATLAS High-Luminosity LHC (HL-LHC) are very exciting, due to the abundance of expected data.
One of the primary limiting factors in this analysis was its highly restricted sample size.
Integrated luminosity from HL-LHC is expected to reach approximately 3000 \ifb.
Rerunning the limit framework with artificially scaled statistical yields produces vastly tighter constraints,
    shown in Figs. \ref{fig:mu_xsec_future} and \ref{fig:limit_slices_future}.
These projected limits 
    ($0.3<\kvv<1.9$, $-4.5<\kl<8.0$)
    indicate much tighter constraints even with yields alone.
Using a full log-likelihood scan it may even be possible to detect the di-Higgs process directly.
Looking farther out in the future, Higgs physics is not limited to only the LHC or even hadronic colliders.
The hypothetical Future Circular Collider (FCC) \cite{Taliercio:2022maa} can operate as both a hadronic and leptonic collider,
    with the possibility to study the Higgs self-coupling in both configurations.
Purely leptonic colliders, such as the planned International Linear Collider (ILC) \cite{Apresyan:2022tqw} 
    or theoretical high-energy muon colliders \cite{Chiesa:2021qpr},
    could also serve as leptonic probes of the Higgs couplings.
Any of these future experiments could serve to further understanding of the Higgs and its self-couplings.
A number of advanced techniques were pioneered in this Run 2-based analysis,
    which will surely be expanded upon in the HL-LHC and beyond.
Therefore, this thesis constitutes a significant contribution to the fundamental study of the Standard Model,
    not only from the far-improved limits provided directly,
    but also by virtue of the techniques developed herein,
    which will facilitate much more efficient use of the data collected in future studies.


\begin{figure}
    \centering
    \begin{subfigure}{0.48\textwidth} 
        \includegraphics[width=\linewidth,height=\textheight,keepaspectratio]{conclusion/limit_slice_HL-LHC_kv_1p0}
        \caption{}
        \label{fig:limit_slice_kv_1p0_future}
    \end{subfigure}
    \begin{subfigure}{0.48\textwidth}
        \includegraphics[width=\linewidth,height=\textheight,keepaspectratio]{conclusion/limit_slice_HL-LHC_kl_1p0}
        \caption{}
        \label{fig:limit_slice_kl_1p0_future}
    \end{subfigure}\\
    \begin{subfigure}{0.48\textwidth} 
        \includegraphics[width=\linewidth,height=\textheight,keepaspectratio]{conclusion/limit_slice_HL-LHC_k2v_1p0}
        \caption{}
        \label{fig:limit_slice_k2v_1p0_future}
    \end{subfigure}
    \caption{
        Two Dimensional Limit Exclusion plots for a hypothetical 3000 \ifb of data.
        The $\kappa$ scaling factors corresponding to regions outside the limit boundary
            can be excluded based on the yield of the observed data.
    }
    \label{fig:limit_slices_future}
\end{figure}


%Predict future results



%\input{src/printer_calibration.tex} TODO: you should remember to use this eventually (thanks Matthew!)

%\StartAppendix
%    \input{src/appendix_A.tex}


% Glossary
% Check with specific department on the style to use
%\clearpage
%\singlespacing%
%\setglossarystyle{list}
%\printglossary[title=GLOSSARY,toctitle=GLOSSARY]
%\doublespacing%

% Bibliography goes below
% Check with specific department on the appropriate
% bibliography style to use

%\nocite{*}
\bibliographystyle{bib/JHEP}
\raggedright
\bibliography{
    bib/introduction.bib,
    bib/analysis.bib,
    bib/atlas.bib,
    bib/ATLAS.bib,
    bib/ATLAS-errata.bib,
    bib/ATLAS-SUSY.bib,
    bib/ATLAS-useful.bib,
    bib/CMS.bib,
    bib/ConfNotes.bib,
    bib/Extra.bib,
    bib/lhc.bib,
    bib/PubNotes.bib,
    bib/reconstruction.bib,
    bib/selection.bib,
    bib/theory.bib,
    bib/trigger.bib,
    bib/results.bib
}

\end{thesis}
\end{document}
